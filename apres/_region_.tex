\message{ !name(apres_quali.tex)}% Gerar um texto com o conteúdo da apresentação
% \documentclass[class=article]{beamer}
\documentclass[10pt,aspectratio=1610,lualatex]{beamer}
\mode<presentation>
%\documentclass[dvips]{beamer}
%\documentclass[notes]{beamer}
 \usepackage{beamerbaserequires}
 \usepackage{beamerbaseoptions}
% \usepackage{beamerbasearticle}
\usepackage{appendixnumberbeamer}
\usepackage[portuguese,brazil,brazilian]{babel}
\usepackage[utf8]{luainputenc}
%\usepackage{xunicode}% 
%\usepackage{xltxtra}%
%\usepackage{metalogo}%
%\usepackage{xkeyval}%
%\usepackage{polyglossia}
%\usepackage{lmodern}
\usepackage{ifluatex}
%\usepackage{times}
\usepackage{ae}
%\usepackage{helvet}
\usepackage{bm}
\usepackage[no-math]{fontspec}
\usepackage{pifont}
\usepackage{FiraSans}
\usepackage{amsfonts}
\usepackage{amsmath}
% \usepackage{unicode-math}
\usepackage{amsthm}
%\usepackage{cmbright}
%\usepackage{cmbright}
\usepackage[T1]{fontenc}
\usepackage{amssymb}
\usepackage{color}
\usepackage{animate}
\usepackage{multimedia}
\usepackage{media9}
\usepackage{beamerthemesplit}
\usepackage{booktabs}
\usepackage{parskip}
\usepackage{microtype}
\usepackage{setspace}
\usepackage{pdflscape}
\usepackage{hyperref}
% \hypersetup{
%   colorlinks=true,    % false: boxed links; true: colored links
%   linkcolor=blue,     % color of internal links
%   citecolor=blue,     % color of links to bibliography
%   filecolor=magenta,  % color of file links
%   urlcolor=blue,
%   bookmarksdepth=4
%   pdftitle={Mostra PG 2017}, 
%   pdfauthor={Sabrina Tigik Ferr\~ao},
%   pdfsubject={Mostra PG 2017},
%   pdfkeywords={plasma}{plasma kinetic theory}{collisions}
%   {quasilinear theory}{weak turbulence theory} 
% }
\usepackage{url}
\usepackage{graphicx}
\usepackage{caption}
\usepackage{ifxetex}
\usepackage{xspace}
\usepackage{xcolor}
% short arrows
\usepackage{stmaryrd}
\usepackage{units}
\usepackage{calc}
\usepackage{tikz}
\usepackage{etoolbox}
\usepackage[scale=2]{ccicons}
\usepackage{pgffor}
\usepackage{pgfplots}
\usepackage{pgfopts}
\usepackage{pgfplotsthemetol}
\usepgfplotslibrary{dateplot}
\pgfplotsset{compat=newest}
%\usefonttheme{professionalfonts}
% \useinnertheme{metropolis}
% \useoutertheme{metropolis}
% \usefonttheme{metropolis}

% \useoutertheme{progressbar}
% \useinnertheme{progressbar}
% \usefonttheme{progressbar}

% \usetheme{AnnArbor}
% \usetheme{Antibes}
% \usetheme{Arkyenell}
% \usetheme{bars}
% \usetheme{Berlin}
% \usetheme{BerlinFU}
% \usetheme{Bergen}
% \usetheme{Berkeley}
% \usetheme{Boadilla}
% \usetheme{boxes}
% \usetheme{CambridgeUS}
% \usetheme{classic}
% \usetheme{Copenhagen}
% \usetheme{Cuerna}
% \usetheme{Darmstadt}
% \usetheme{default}
% \usetheme{DetlevCM}
% \usetheme{Dresden}
% \usetheme{epyt}
% \usetheme{EastLansing}
% \usetheme{Frankfurt}
% \usetheme{Goettingen}
% \usetheme{Hannover}
% \usetheme{Ilmenau}
% \usetheme{Jacobs}
% \usetheme{JLTree}
% \usetheme{JuanLesPins}
% \usetheme{lankton-keynote}
% \usetheme{lined}
% \usetheme{Luebeck}
% \usetheme{Madrid}
% \usetheme{Malmoe}
% \usetheme{Marburg}
 \usetheme[progressbar=frametitle,numbering=fraction]{metropolis}
% \usetheme{Montpellier}
% \usetheme{PaloAlto}
% \usetheme{PhnomPenh}
% \usetheme{Pittsburgh}
% \usetheme{progressbar}
% \usetheme{Rochester}
% \usetheme{shadow}
% \usetheme{sidebar}
% \usetheme{Singapore}
% \usetheme{split}
% \usetheme{Szeged}
% \usetheme{TorinoTh} %não funciona direito
% \usetheme{tree}
% \usetheme{Warsaw}

% \setbeamercovered{transparent}

%% Color themes
% \usecolortheme{Arkyenell}
% \usecolortheme{beaver}
% \usecolortheme{bluesimplex}
% \usecolortheme{bluedawn}
% \usecolortheme{crane}
% \usecolortheme{dolphin}
% \usecolortheme{dove}
% \usecolortheme{ETII} % BR colors
% \usecolortheme{fly}
% \usecolortheme{goeagles}
% \usecolortheme{hohenheim}
% \usecolortheme{lily}
% \usecolortheme{MedStarColors}
% \usecolortheme{metropolis}
% \usecolortheme{monarca}
% \usecolortheme{orchid}
% \usecolortheme{penn}
  \usecolortheme{progressbar}
% \usecolortheme[snowy]{owl}
% \usecolortheme{rose}
% \usecolortheme{seagull}
% \usecolortheme{seahorse}
% \usecolortheme{solarized}
% \usecolortheme{spruce}
% \usecolortheme{torinoth} 
% \usecolortheme{whale}
% \usecolortheme{wolverine}
% %% dark color themes %%
% \usecolortheme[overlystylish]{albatross}
% \usecolortheme{albatross}
% \usecolortheme{beetle}
% \usecolortheme[cautious]{owl}
% \usecolortheme{cormorant}
% \usecolortheme{frigatebird}
% \usecolortheme{magpie}
% \usecolortheme[dark]{solarized}

%% Fonte padrão do LaTeX para equações
  \usefonttheme{professionalfonts}

%  \usefonttheme{default}

% Os arquivos .sty que definem a fonte e o esquema de cores usados
% devem estar na mesma pasta deste arquivo.

 %% Controla a espessura da linha de progresso
 \makeatletter
  \setbeamertemplate{progress bar in head/foot}{
  \nointerlineskip
  \setlength{\metropolis@progressinheadfoot}{%
    \paperwidth * \ratio{\insertframenumber pt}{\inserttotalframenumber pt}%
  }%
  \begin{beamercolorbox}[wd=\paperwidth]{progress bar in head/foot}
    \begin{tikzpicture}
      \draw[bg, fill=bg] (0,0) rectangle (\paperwidth, 0.4pt);
      \draw[fg, fill=fg] (0,0) rectangle (\metropolis@progressinheadfoot, 0.4pt);
    \end{tikzpicture}%
  \end{beamercolorbox}
}

\setbeamertemplate{progress bar in section page}{
  \setlength{\metropolis@progressonsectionpage}{%
    \textwidth * \ratio{\insertframenumber pt}{\inserttotalframenumber pt}%
  }%
  \begin{tikzpicture}
    \draw[bg, fill=bg] (0,0) rectangle (\textwidth, 0.4pt);
    \draw[fg, fill=fg] (0,0) rectangle (\metropolis@progressonsectionpage, 0.4pt);
  \end{tikzpicture}%
}
\makeatother

\newcommand{\darken}[1]{%
  \begingroup
  \setbeamercolor{background canvas}{bg=normal text.fg}
  \setbeamercolor{section title}{fg=normal text.bg}
  #1\endgroup
}

%% Cor do texto:
% \definecolor{fgblue}{rgb}{0, 0.349, 0.584} % cores da ufrgs
% \definecolor{fgblue}{rgb}{0.2, 0.3, 0.5} %  cores do tema "Progress bar"
% \setbeamercolor{normal text}{fg=fgblue}
%  \setbeamercolor{tilte text}{fg=fgblue}

%\input{titlepage}
% Aqui é definido que as subseções (as bolinhas do Ilmenau) não
% irão aparecer na headline
% \setbeamertemplate{headline}
% {%
%   \begin{beamercolorbox}{section in head/foot}
%   \vskip2pt\insertsectionnavigationhorizontal{\textwidth}{}{}\vskip2pt
%   \end{beamercolorbox}
% }
% % O trecho abaixo foi inserido para que o tema usado (Ilmenau) mostre o
% % slide atual/o número de slides, já que eu eliminei a barra de prgresso
% % nativa do tema, por ocupar muito espaço.
% \defbeamertemplate*{footline}{myminiframes theme}
%   {%
%     \begin{beamercolorbox}[colsep=1.5pt]{upper separation line foot}
%     \end{beamercolorbox}
%     \begin{beamercolorbox}[ht=2.5ex,dp=1.125ex,%
%       leftskip=.3cm,rightskip=.3cm plus1fil]{author in head/foot}%
%       \leavevmode{\usebeamerfont{author in head/foot}\insertshortauthor}%
%       \hfill%
%       {\usebeamerfont{institute in head/foot}\usebeamercolor[fg]{institute in head/foot}\insertshortinstitute}%
%     \end{beamercolorbox}%
  %   \begin{beamercolorbox}[ht=2.5ex,dp=1.125ex,%
  %     leftskip=.3cm,rightskip=.3cm plus1fil]{title in head/foot}%
  %     {\usebeamerfont{title in head/foot}\insertshorttitle\hfill
  %       \insertframenumber/\inserttotalframenumber}%<-here
  %   \end{beamercolorbox}%
  %   \begin{beamercolorbox}[colsep=1.5pt]{lower separation line foot}
  %   \end{beamercolorbox}
  % }
  % \makeatother

% % Isso não foi usado, mas serve para remover completamente a headline,
% % para o caso de equações grandes, que extrapolam os limites da headline.
% \makeatletter
%     \newenvironment{withoutheadline}{
%         \setbeamertemplate{headline}[default]
%         \def\beamer@entrycode{\vspace*{-\headheight}}
%     }{}
% \makeatother

%\setbeamertemplate{navigation symbols}{}
%\logo{\includegraphics[scale=0.02]{images/tree04}}
%%%%%%%%%%%%%%%%%%%%%%%%%%%%%%%

\title{Evolução temporal de processos fracamente turbulentos
  na presença de interações colisionais}
\subtitle[]{Exame de Qualificação ao Doutorado} 
\author[Sabrina Tigik Ferrão]{Doutoranda: Sabrina Tigik Ferrão\\
  Orientador: Prof. Dr. Luiz Fernando Ziebell}
\institute[IF-UFRGS]{Universidade Federal do Rio Grande do Sul\\
  Instituto de Física, sala M206 \\ e-mail: sabrina.tigik@ufrgs.br}
\titlegraphic{\vspace{5.5cm} \hfill
  \includegraphics[width=0.3\textwidth]{ufrgs.png}}

\begin{document}

\message{ !name(apres_quali.tex) !offset(308) }
\section{Teoria de turbulência fraca}
% \begin{frame}
%   \frametitle{Processos não lineares}
%   \begin{itemize}
%     \item Dependendo do tipo de fenômeno que se está estudando, o estado
%     estacionário atingido pela aproximação quaselinear pode continuar
%     evoluindo sob a ação de processos não lineares;
%     \vspace{0.4cm}
%     \item O formalismo responsável por essa análise não linear é conhecido
%     como \emph{teoria de turbulência fraca};
%     \vspace{0.4cm}
%     \item Este é o próximo passo na cadeia de aproximações perturbativas para
%     o estudo de instabilidades cinéticas em um plasma.
%     %\vspace{0.4cm}    
%  \end{itemize}
% \end{frame}

\begin{frame}
  \frametitle{Teoria de turbulência fraca}
  \begin{itemize}
    \item Na presença de uma instabilidade, a intensidade do
    campo elétrico é amplificada;
    \vspace{0.3cm}
    \item Com o aumento da amplitude dos campos perturbativos,
    os processos que envolvem a interação entre as diferentes
    oscilações presentes no plasma começam a ganhar importância;
    \vspace{0.3cm}
    \item Para descrever essas interações, é necessário que
    incorporemos efeitos não lineares de segunda ordem na
    descrição cinética do plasma;
    \vspace{0.3cm}
    \item Se a intensidade das instabilidades estiver dentro do
    limite em que $\mathcal{E}_{fluc} \ll \mathcal{E}_{kin}$, a
    condição de crescimento lento é satisfeita;
    \vspace{0.3cm}
    \item Nesse caso, a análise não linear ainda pode ser feita
    através de um método perturbativo;
    \vspace{0.3cm}
    \item Este método é conhecido como \emph{teoria de turbulência
      fraca}.
  \end{itemize}
\end{frame}



\begin{frame}
  \frametitle{Ponto de partida}
  \begin{itemize}
    \item Estamos interessados em descrever ondas eletrostáticas
    propagando em um plasma
    \begin{itemize}
      \item homogêneo,
      \item totalmente ionizado,
      \item não magnetizado;
      % \item e, no momento, sem interação colisional;
    \end{itemize}
    \vspace{0.3cm}
    \item Na ausência de colisões, a equação de movimento é dada
    pela equação de Vlasov
    \begin{equation}
  \label{vlasov-nl}
  \frac{\partial f_a({\bf r}, {\bf v},t)}{\partial t}
  +{\bf v \cdot \nabla}f_a({\bf r}, {\bf v},t)
  +\frac{e_a}{m_a}{\bf E}({\bf r},t){\bf \cdot}
  \frac{\partial f_a({\bf r},{\bf v},t)}{\partial \bf v}=0,
    \end{equation}
    onde o campo elétrico é dado pela forma diferencial da lei de Gauss
    \begin{equation}
   {\bf \nabla \cdot E}({\bf r},t)
   =4\pi \hat n \sum_a e_a\int d^3 v f_a({\bf r}, {\bf v},t). 
  \end{equation}
  \end{itemize}
\end{frame}

\begin{frame}
  \frametitle{Abordagem perturbativa}
  \begin{itemize}
    \item Para oscilações de baixa amplitude, podemos escrever
    \begin{equation*}
      %\label{pert-nl}
      \begin{split}
	f_a({\bf r}, {\bf v},t)&=F_a({\bf v})
	+\delta f_a({\bf r}, {\bf v},t),\\
      {\bf E}({\bf r},t)&=\delta {\bf E}({\bf r},t);
    \end{split}
    \end{equation*}
    \vspace{0.4cm}
    \item Substituindo as expressões acima na equação de Vlasov
    e levando em conta que $F_a$ é homogênea, ficamos com
    \begin{equation}
      \frac{\partial F_a}{\partial t}-\frac{e_a}{m_a}\
      \delta {\bf E \cdot} \frac{\partial F_a}{\partial \bf v}
      +\frac{\partial \delta f_a}{\partial t}+{\bf v \cdot}
      \frac{\partial \delta f_a}{\partial {\bf r}}
      +\frac{e_a}{m_a}\ \delta {\bf E \cdot}
      \frac{\partial \delta f_a}{\partial \bf v}=0.
      \label{turb_ave_vlasov}
    \end{equation}
  \end{itemize}
\end{frame}

\begin{frame}
  \frametitle{Formalismo não linear}
  \begin{itemize}
    \item Tomando a média da Eq. \eqref{turb_ave_vlasov}, obtemos uma
    equação para a evolução de $F_a$
    \begin{equation}
      \frac{\partial F_a}{\partial t}
      =-\frac{e_a}{m_a}\left< \delta {\bf E \cdot}
      \frac{\partial \delta f_a}{\partial \bf v}\right>;
      \label{form_par}
    \end{equation}
    \pause 
    \item E, subtraindo \eqref{form_par} de \eqref{turb_ave_vlasov},
    obtemos uma equação para as flutuações
    \begin{equation}
      \frac{\partial \delta f_a}{\partial t}
      +{\bf v \cdot } \frac{\partial \delta f_a}{\partial {\bf r}}
      +\frac{e_a}{m_a}\ \delta {\bf E \cdot}
      \frac{\partial F_a}{\partial t}
      +\frac{e_a}{m_a}\left[\delta {\bf E \cdot}
      \frac{\partial \delta f_a}{\partial \bf v}
      -\left< \delta {\bf E \cdot}
      \frac{\partial \delta f_a}{\partial \bf v}\right>\right]=0;
      \label{flut}
    \end{equation}
    %\vspace{0.2cm}
    \item Os termos entre colchetes, na equação acima, são descartados
    na teoria quaselinear;
    \vspace{0.4cm}
    \item No formalismo não linear, contudo, eles são mantidos.
  \end{itemize}
\end{frame}

\begin{frame}
  \frametitle{Transformadas de Fourier-Laplace}
  \begin{itemize}
    \item O procedimento padrão consistem decompor as flutuações
    em termos de suas transformadas de Fourier-Laplace, com relação
    à escala de tempo rápida das oscilações, mas supondo lenta evolução
    temporal para as amplitudes:
    \begin{equation*}
      \begin{split}
	\delta f_a({\bf r}, {\bf v},t)
	&=\int d^3k \int_L d \omega\,\delta f^a_{{\bf k}, \omega}({\bf v}, t)
	e^{i({\bf k \cdot r}-\omega t)},\\
	\delta f^a_{{\bf k}, \omega}({\bf v}, t)
	&=\frac{1}{(2 \pi)^4}\int d^3r \int_0^{\infty} dt\,
	\delta f_a({\bf r},{\bf v},t) e^{-i({\bf k \cdot r} - \omega t)},\\
	\delta {\bf E}({\bf r},t)
	&=\int d^3k \int_L d\omega\, \delta{\bf E}_{{\bf k},\omega}(t)
	e^{i({\bf k \cdot r}-\omega t)},\\
	\delta{\bf E}_{{\bf k},\omega}(t)
	&=\frac{1}{(2 \pi)^4}\int d^3r\int_0^{\infty} dt\,
	\delta{\bf  E}({\bf   r},t) e^{-i({\bf k \cdot r} - \omega t)};
      \end{split}
    \end{equation*}
    % onde a integração se dá ao longo do caminho $L$, com um prolongamento
    % de $\omega = -\infty +i\sigma$ até $\omega=\infty+i\sigma$, em que
    % $\sigma >0$ e $\sigma \rightarrow 0$.
    \item Para os termos não lineares, onde temos o produto de duas
    funções, a transformada de Fourier-Laplace é dada pela convolução
    dessas funções
    \begin{equation*}
      \frac{1}{(2\pi)^4}\int d^3r \int dt\,
      \delta f_a({\bf r}, {\bf v},t)\,
      \delta{\bf E}({\bf r},t)\, e^{-i({\bf k \cdot r}-\omega t)}
      =\int d^3k' \int d\omega '\,\delta f^a_{{\bf k-k'},\omega-\omega '}\,
      \delta{\bf E}_{{\bf k'},\omega '}.
    \end{equation*}
  \end{itemize}
\end{frame}

\begin{frame}
  \frametitle{Equações cinéticas não lineares}
  \begin{itemize}
    \item  Das transformadas, obtemos o conjunto de equações
    hierárquicas da teoria de turbulência fraca;
    \pause
    \item Composto pela equação cinética formal para as
    partículas do tipo $a$
    \begin{equation}
      \frac{\partial F_a}{\partial t}=-\frac{e_a}{m_a}
      \frac{\partial}{\partial {\bf v}}
      {\bf \cdot}\int d^3k \int d\omega \int d^3k'\int d\omega '\,
      \left<\delta{\bf E}_{{\bf k'},\omega '}\,
      \delta f^a_{{\bf k},\omega}\right>
      e^{i({\bf k+k'}){\bf \cdot r}-i(\omega+\omega ')t};
      \label{eq-cin-par}
    \end{equation}
    \pause
    \item Pela equação para a evolução da função de distribuição
    perturbativa das partículas
    \begin{equation}
      \begin{split}
	\left(\omega-{\bf k \cdot v}+i\frac{\partial}{\partial t}\right)
	\delta f^a_{{\bf k},\omega}
	&=-i \frac{e_a}{m_a} \delta {\bf E}_{{\bf k},\omega}{\bf \cdot}
	\frac{\partial F_a}{\partial \bf v}-i\frac{e_a}{m_a}
	\frac{\partial}{\partial \bf v} {\bf \cdot} \int d^3k'
	\int d\omega '\\
	\times \bigr[\delta {\bf E}_{{\bf k'},\omega '}
	&\delta f^a_{{\bf k-k'},\omega-\omega '}
	-\left<\delta{\bf E}_{{\bf k'},\omega '}
	\delta f^a_{{\bf k-k'},\omega-\omega '}\right>\bigr];
      \end{split}
   \label{ev-par-per}
    \end{equation}
  \pause
  \item E pela forma diferencial da lei de Gauss para as
  flutuações do campo elétrico
  \begin{equation}
    {\bf k \cdot}\delta{\bf E}_{{\bf k},\omega}=-4\pi \hat n i
    \sum_a e_a\int d^3v\ \delta f^a_{{\bf k}, \omega}.
    \label{poisson-flut}
  \end{equation}
\end{itemize}
\end{frame}

\begin{frame}
  \frametitle{Aproximação de dois tempos}
  \begin{itemize}
      % \item Vemos que a derivada temporal de lenta escala de tempo,
      % $i(\partial/\partial t)$, foi  mantida no lado esquerdo da
      % equação (\ref{ev-par-per});
      % \pause
      \item Na aproximação de dois tempos, propomos que o fator
      $i(\partial/\partial t)$ seja ``absorvido'' em uma nova
      definição para a frequência angular, $\omega \rightarrow
      \omega+i \partial/\partial t$;
      \vspace{0.25cm}
      \pause
      \item Dessa forma, a equação para a evolução da função de
      distribuição perturbativa pode ser resolvida iterativamente
      até terceira ordem de $\delta {\bf E}_{{\bf k},\omega}$;
      \vspace{0.25cm}
      \pause
      \item Uma vez obtida a solução iterativa desejada, ela é
      inserida na Eq. \eqref{poisson-flut};
      \vspace{0.25cm}
      \pause
      \item Por fim, são tomadas as médias de ensemble apropriadas,
      supondo que existam fases aleatórias associadas com as flutuações;
      \vspace{0.25cm}
      \pause
      \item Como resultado desse procedimento, obtemos a equação não
      linear para o balanço espectral, que forma a base da teoria de
      turbulência fraca.
  \end{itemize}
\end{frame}

\begin{frame}\vspace{-0.6cm}
  % [noframenumbering]
  % \frametitle{Equação do balanço espectral}
  \begin{eqnarray}
    &&\frac{i}{2}\frac{\partial \epsilon({\bf k},\omega)}{\partial \omega}
       \frac{\partial \left<\delta E^2\right>_{{\bf k},\omega}}{\partial t}
       +\mbox{Re}\,\epsilon({\bf k},\omega)\left<\delta E^2 \right>_{{\bf k},\omega}
       +i\mbox{Im}\,\epsilon({\bf k},\omega)\left<\delta E^2 \right>_{{\bf k},\omega}
       \nonumber\\
    &&\qquad \qquad \qquad -\frac{2}{\pi}\frac{1}{k^2\epsilon^*({\bf k},\omega)}
       \sum_ae_a^2\int d{\bf v}\,\delta(\omega-{k\cdot v})F_a({\bf v})
       \nonumber\\
    &&=-2\int d{\bf k'} \int d\omega '\
       \biggl\{ \left[\chi^{(2)}({\bf k'},\omega '|{\bf k-k'},
       \omega-\omega ')\right]^2
       \biggl[\frac{\left<\delta E^2\right>_{{\bf k - k'},\omega-\omega '}}
       {\epsilon({\bf k'},\omega ')}
       \nonumber\\
    &&+\frac{\left<\delta E^2\right>_{{\bf k'},\omega '}}
       {\epsilon({\bf k - k'},\omega-\omega ')}\biggr]
       -\bar \chi^{(3)}({\bf k'},\omega|-{\bf k'},-\omega '|{\bf k},\omega)
       \left<\delta E^2\right>_{{\bf k'},\omega'}\biggr\}
       \left<\delta E^2\right>_{{\bf k},\omega}\nonumber\\
    &&+2\int d{\bf k'} \int d\omega '
       \frac{|\chi^{(2)}({\bf k'},\omega '|{\bf k-k'},\omega-\omega ')|^2}
       {\epsilon^*({\bf k},\omega)}
       \left<\delta E^2\right>_{{\bf k'},\omega '}
       \left<\delta E^2\right>_{{\bf k-k'},\omega-\omega '}
       \label{nlebe}\\
    &&-\frac{4}{\pi}\int d{\bf k '} \int d\omega '\
       \frac{1}{k^2|\epsilon({\bf k'},\omega ')|^2}
       \biggl[ \frac{[\chi^{(2)}({\bf k'},\omega '|{\bf k-k'},\omega-\omega ')]^2}
       {\epsilon({\bf k-k'},\omega-\omega ')}
       \left<\delta E^2 \right>_{{\bf k},\omega}
       \nonumber\\
    &&-\frac{|\chi^{(2)}({\bf k'},\omega '|{\bf k-k'},\omega-\omega ')|^2}
       {\epsilon^*({\bf k},\omega)}
       \left<\delta E^2 \right>_{{\bf k-k'},\omega-\omega '}\biggr]
       \sum_ae_a^2\int d{\bf v}\, \delta(\omega '-{\bf k \cdot v})F_a({\bf v})
       \nonumber\\
    &&-\frac{4}{\pi}\int d{\bf k '} \int d\omega '\
       \frac{1}{|{\bf k - k'}|^2|\epsilon({\bf k-k'},\omega-\omega ')^2}
       \biggl[ \frac{[\chi^{(2)}({\bf k'},\omega '|{\bf k-k'},\omega-\omega ')]^2}
       {\epsilon({\bf k'},\omega ')}\left<\delta E^2 \right>_{{\bf k},\omega}
       \nonumber\\
    &&-\frac{|\chi^{(2)}({\bf k'},\omega '|{\bf k-k'},\omega-\omega ')|^2}
       {\epsilon^*({\bf k},\omega)}
       \left<\delta E^2 \right>_{{\bf k'},\omega '}\biggr]\sum_ae_a^2\int d{\bf v}\,
       \delta(\omega-\omega '-{\bf (k-k')\cdot v})F_a({\bf v})
       \nonumber
  \end{eqnarray}
\end{frame}

\begin{frame}
  \frametitle{Na equação para o balanço espectral:}
  \begin{itemize}
    \item No lado esquerdo da equação estão as expressões que correspondem
    à aproximação quaselinear e no lado esquerdo estão os termos não lineares;
    \vspace{0.5cm}
    \item Nos denominadores temos a função resposta dielétrica linear
    $\epsilon({\bf k},\omega)$, que é dada por
    \begin{equation*}
      \epsilon({\bf k},\omega)=1+\sum_a\chi_a({\bf k},\omega);
    \end{equation*}
    %\vspace{0.1cm}
    \item Sendo $\sum_a \chi_a({\bf k},\omega)$ a susceptibilidade dielétrica
    linear, $\sum_a \chi_a^{(2)}({\bf k}_1,\omega_1|{\bf k}_2,\omega_2)$ a
    susceptibilidade não linear de segunda ordem e
    $\sum_a \bar \chi_a^{(3)}({\bf k}_1,\omega_1|{\bf k}_2,\omega_2|{\bf k}_3,\omega_3)$
    a susceptibilidade não linear de terceira ordem.
  \end{itemize}
\end{frame}

\begin{frame}
  \frametitle{Equação cinética generalizada para as partículas}
  \begin{itemize}
    \item Para a equação cinética das partículas, é usado um procedimento
    similar ao aplicado para obter a equação para o balanço espectral, mas
    com relação à Eq. \eqref{eq-cin-par};
    \vspace{0.4cm}
    \item Com isso pode-se chegar a uma equação cinética generalizada formal
    para as partículas, composta por vários termos não lineares de acoplamento
    entre as ondas;
    \vspace{0.4cm}
    \item Neste trabalho, no entanto, vamos usar apenas os termos correspondentes
    à fricção e difusão quaselinear no espaço de velocidades:
    \begin{equation*}
      \begin{split}
	\frac{\partial F_a}{\partial t}
	&=\frac{\pi e_a^2}{m_a^2}\int d{\bf k}\,\int d\omega\,
	\left( \frac{\bf k}{k}\cdot
	  \frac{\partial}{\partial {\bf v}}\right)
	\delta(\omega -{\bf k \cdot v})\\
	&\times\left[\mbox{Im} \frac{m_a \epsilon({\bf k},\omega)}
	  {2\pi^3k \left|\epsilon ({\bf k},\omega) \right|^2 } F_a
	  +\left<\delta E^2\right>_{{\bf k},\omega}\left(\frac{\bf k}{k}
	    \frac{\partial F_a}{\partial \bf v}\right)\right].
      \end{split}
   %\label{part}
    \end{equation*}
  \end{itemize}
\end{frame}

\begin{frame}
  \frametitle{Equação cinética das ondas para automodos lineares}
  \begin{itemize}
    \item O procedimento padrão para lidar com a expressão geral para
    o balanço espectral e obter as equações cinéticas para os automodos,
    consiste em supor amplificação lenta das ondas, ou seja:\,
    $|\mbox{Im}\, \epsilon({\bf k},\omega)|\ll|\mbox{Re}\, \epsilon({\bf k},\omega)|$;
    \vspace{0.2cm}
    \item O que significa que o processo de crescimento das ondas não
    interfere na dinâmica do plasma;
    \vspace{0.2cm}
    \item Dessa forma, a parte imaginária da Eq. \eqref{nlebe} leva à equação
    cinética das ondas, enquanto a parte real leva à equação de dispersão
    das ondas;
    \vspace{0.2cm}
    \item  Com isso, na forma generalizada, os modos normais de oscilação
    são determinados pela resposta linear do plasma, enquanto as interações
    onda-partícula e onda-onda são descritas pelas equações cinéticas não
    lineares das ondas e das partículas;
    \vspace{0.2cm}
    \item Então, o que temos aqui é uma teoria que descreve interações
    não lineares que envolvem os automodos lineares do plasma. 
    % \item Lembrando que, no nosso caso, vamos usar a aproximação quaselinear
    % para a equação cinética das partículas.
  \end{itemize}
\end{frame}

\begin{frame}
  \frametitle{Modos normais de oscilação}
  \begin{itemize}
    \item As relações de dispersão lineares para os modos normais lineares
    são as soluções da seguinte equação
    \begin{equation}
      0=\mbox{Re}\ \epsilon({\bf k},\omega)
      \left<\delta E^2 \right>_{{\bf k},\omega};
      \label{real-lin}
    \end{equation}
    \item Supondo que a solução da equação acima seja dada por
    $\omega=\omega^{\alpha}_{{\bf k},\omega}$, onde $\alpha$ leva em
    conta os diferentes modos normais de oscilação do plasma;
    \vspace{0.2cm}
    \item Com isso, podemos escrever a amplitude espectral das ondas:
    \begin{equation}
      \left<\delta E^2\right>_{{\bf k},\omega}
      =\sum_{\alpha}\bigl[I^{+\alpha}_{\bf k}
      \delta(\omega-\omega_{\bf k}^{\alpha})
      +I^{-\alpha}_{\bf k} \delta(\omega+\omega_{\bf k}^{\alpha})\bigr],
      \label{wave-dens}
    \end{equation}
    onde $\pm$ denota a direção de propagação de $\alpha$;
    \vspace{0.2cm}
    \item Sendo que os modos normais $\alpha$, são os modos de
    oscilação que satisfazem
    \begin{equation}
      \label{eigenmode}
      \epsilon({\bf k,\pm\, \omega_{{\bf k},\omega}^\alpha}) \approx 0.
    \end{equation}
  \end{itemize}
\end{frame}

\begin{frame}
  \frametitle{Equação cinética para ondas de Langmuir $(L)$}
  \begin{itemize}
    \item
    \begin{equation}
  \label{L}
  \begin{split}
\frac{\partial I_{\bf k}^{\sigma L}}{\partial t}
&=\frac{\omega_{pe}^2}{k^2}\int d{\bf v}\;
\delta(\sigma\omega_{\bf k}^L-{\bf k}\cdot{\bf v})
\biggl(\hat{n}\,e^2\,F_e({\bf v})+\pi\,(\sigma\omega_{\bf k}^L)\,
\;{\bf k}\cdot\frac{\partial F_e({\bf v})}{\partial{\bf v}}
\,I_{\bf k}^{\sigma L}\biggr)\\
&-\,\pi\sigma\,\omega_{\bf k}^L
\,\frac{e^2}{2T_e^2}\sum_{\sigma ',\sigma ''}\int d{\bf k'}\;
\frac{\mu_{{\bf k}-{\bf k}'}^S
\,({\bf k}\cdot{\bf k}')^2}{k^2\,k'^2\,|{\bf k}-{\bf k}'|^2}
\biggl(\sigma '\omega_{{\bf k}'}^L\,
\frac{I_{{\bf k}-{\bf k}'}^{\sigma ''S}}{\mu_{{\bf k}-{\bf k}'}^S}
I_{\bf k}^{\sigma L}\\
&+\,\sigma ''\omega_{{\bf k}-{\bf k}'}^L
\,I_{{\bf k}'}^{\sigma 'L}\,I_{\bf k}^{\sigma L}
-\sigma\omega_{\bf k}^L I_{{\bf k}'}^{\sigma 'L}
\frac{I_{{\bf k}-{\bf k}'}^{\sigma ''S}}
{\mu_{{\bf k}-{\bf k}'}^S}\biggr)
\;\delta(\sigma\omega_{\bf k}^L-\sigma '\omega_{{\bf k}'}^L
-\sigma ''\omega_{{\bf k}-{\bf k}'}^S)\\
&+\,\sigma\omega_{\bf k}^L\,\frac{e^2}{m_e^2\,\omega_{pe}^2}
\sum_{\sigma '}\int d{\bf k'}\int d{\bf v}\;
\frac{({\bf k}\cdot{\bf k}')^2}{k^2\,k'^2}\;\delta[\sigma\omega_{\bf k}^L
-\sigma '\omega_{{\bf k}'}^L-({\bf k}-{\bf k}')\cdot{\bf v}]\\
&\times\biggl[\frac{\hat{n}\,e^2}{\omega_{pe}^2}
\bigl(\sigma\omega_{\bf k}^LI_{{\bf k}'}^{\sigma 'L}
  -\sigma '\omega_{{\bf k}'}^L I_{\bf k}^{\sigma L}\bigr)
+\pi\,\frac{m_e}{m_i}\,I_{{\bf k}'}^{\sigma 'L}
I_{\bf k}^{\sigma L}\;({\bf k}-{\bf k}')
\cdot\frac{\partial F_i({\bf v})}{\partial{\bf v}}\biggr].
  \end{split}
\end{equation}
  \end{itemize}
\end{frame}

\begin{frame}
  \frametitle{Equação cinética para ondas íon-acústicas $(S)$}
  \vspace{-0.9cm}
  % \begin{itemize}
  %   \item 
    \begin{equation}
  \label{S}
  \begin{split}
\frac{\partial}{\partial t}\frac{I_{\bf k}^{\sigma S}}{\mu_{\bf k}^S}
&=\mu_{\bf k}^S\,\frac{\omega_{pe}^2}{k^2}
\int d{\bf v}\;\delta(\sigma\omega_{\bf k}^S-{\bf k}\cdot{\bf v})
\biggl[\hat{n}\,e^2\,[F_e({\bf v})+F_i({\bf v})]\\
&+\,\pi\,(\sigma\omega_{\bf k}^L)\,\biggl({\bf k}\cdot
\frac{\partial F_e({\bf v})}{\partial{\bf v}}
+\frac{m_e}{m_i}\;{\bf k}\cdot
\frac{\partial F_i({\bf v})}{\partial{\bf v}}\biggr)
\,\frac{I_{\bf k}^{\sigma S}}{\mu_{\bf k}^S}\biggr]\\
&-\,\pi\sigma\omega_{\bf k}^L\,\frac{e^2}{4T_e^2}
\sum_{\sigma ',\sigma ''}\int d{\bf k'}\;
\frac{\mu_{\bf k}^S[{\bf k}'\cdot({\bf k}-{\bf k}')]^2}{k^2k'^2|{\bf k}-{\bf k}'|^2}
\biggl(\sigma '\omega_{{\bf k}'}^L\,
I_{{\bf k}-{\bf k}'}^{\sigma ''L}
\frac{I_{\bf k}^{\sigma S}}{\mu_{\bf k}^S}\\
&+\,\sigma ''\omega_{{\bf k}-{\bf k}'}^L
\,I_{{\bf k}'}^{\sigma 'L}\frac{I_{\bf k}^{\sigma S}}{\mu_{\bf k}^S}
-\sigma\omega_{\bf k}^LI_{{\bf k}'}^{\sigma 'L}
I_{{\bf k}-{\bf k}'}^{\sigma ''L}\biggr)
\delta(\sigma\omega_{\bf k}^S-\sigma '\omega_{{\bf k}'}^L
-\sigma ''\omega_{{\bf k}-{\bf k}'}^L)\\
&+\frac{e^2}{m_e^2\omega_{pe}^2}\sigma\omega_{\bf k}^L\sum_{\sigma '}\int d{\bf k'}
\frac{\mu_{\bf k}^S\mu_{{\bf k '}({\bf k}\cdot{\bf k}')^2}
{\lambda_{De}^4k^4k'^4}\int d{\bf v}\;
\delta[\sigma\omega_{\bf k}^S-\sigma '\omega_{{\bf k}'}^S
-({\bf k}-{\bf k}')\cdot{\bf v}]\\
&\times\,\biggl[\frac{\hat{n}\,e^2}{\omega_{pe}^2}W_{{\bf k},{\bf k}'}
\biggl(\sigma\omega_{\bf k}^L\,\frac{I_{{\bf k}'}^{\sigma 'S}}{\mu_{{\bf k}'}^S}
-\sigma '\omega_{{\bf k}'}^L\,\frac{I_{\bf k}^{\sigma S}}{\mu_{\bf k}^S}\biggr)
\,[F_e({\bf v})+F_i({\bf v})]\\
&+\frac{m_e}{m_i}\frac{\pi({\bf k}\cdot{\bf k}')^2}{k^2\,k'^2}
\biggl(W_{{\bf k},{\bf k}'}+\sigma\,\sigma '\frac{k'}{k}\biggr)
\,\frac{I_{{\bf k}'}^{\sigma 'S}}{\mu_{{\bf k}'}^S}
\frac{I_{\bf k}^{\sigma S}}{\mu_{\bf k}^S}({\bf k}-{\bf k}')
\cdot\frac{\partial F_i({\bf v})}{\partial{\bf v}}\biggr].
\end{split}
\end{equation}
  % \end{itemize}
\end{frame}

\begin{frame}
  \frametitle{Evolução temporal das partículas}
  \begin{equation*}
  \begin{split}
    \frac{\partial F_a({\bf v})}{\partial t}&=\frac{\pi e_a^2}{m_a^2}
    \sum_\sigma\sum_{\alpha=L,S}\int d^3k\ \biggl(\frac{\bf k}{k}
    \cdot\frac{\partial}{\partial{\bf v}}\biggr)\ \mu_{\bf k}^\alpha
    \delta(\sigma\omega_{\bf k}^\alpha-{\bf k}\cdot{\bf v})\\
    &\times\biggl(\frac{m_a}{4\pi^2}\frac{\sigma\omega_{\bf k}^L}{k}
    F_a({\bf v})+\frac{I_{\bf k}^{\sigma\alpha}}{\mu_{\bf k}^\alpha}
    \frac{\bf k}{k}\cdot\frac{\partial F_a({\bf v})}{\partial{\bf v}}\biggr);
  \end{split}
  \end{equation*}
\end{frame}

% \begin{frame}
%   \frametitle{Evolução temporal das ondas de Langmuir}
% \end{frame}

\begin{frame}\vspace{-0.6cm}
  \begin{displaymath}
    \begin{split}
        \mbox{NL}
 =&-2\int d{\bf k'} \int d\omega '\
     \biggl\{ \left[\chi^{(2)}({\bf k'},\omega '|{\bf k - k'},
     \omega-\omega ')\right]^2\\
  &\times\biggl[\frac{\left< \delta E^2 \right>_{{\bf k-k'},\omega-\omega '}^0
     +\sum_{\sigma ''}\sum_{\gamma} I_{\bf k -k'}^{\sigma '' \gamma}
     \delta(\omega-\omega ' -\sigma '' \omega_{\bf k -k'}^\gamma)}
     {\epsilon({\bf k'},\omega ')}\\
  &+\frac{\left< \delta E^2 \right>_{{\bf k'},\omega '}^0
  +\sum_{\sigma '}\sum_{\beta}
  I_{\bf k'}^{\sigma ' \beta}\delta(\omega '-\sigma ' \omega_{\bf k'}^\beta)}
     {\epsilon({\bf k-k'},\omega-\omega ')}\biggr]
     -\bar \chi^{(3)}({\bf k'},\omega|-{\bf k'},-\omega '|{\bf k},\omega)\\
     &\times\biggl[ \left< \delta E^2 \right>_{{\bf k'},\omega '}^0
	+\sum_{\sigma '}\sum_{\beta}I_{\bf k'}^{\sigma ' \beta}
	\delta(\omega '-\sigma ' \omega_{\bf k'}^\beta) \biggr]\biggr\}
	\sum_{\sigma}\sum_{\alpha}
  I_{\bf k}^{\sigma \alpha}\delta(\omega-\sigma \omega_{\bf k}^\alpha)\\
  &+2\int d{\bf k'} \int d\omega '
  \frac{|\chi^{(2)}({\bf k'},\omega '|{\bf k-k'},\omega-\omega ')|^2}
     {\epsilon^*({\bf k},\omega)}\biggl[ \left< \delta E^2 \right>_{{\bf k'},\omega '}^0
  +\sum_{\sigma '}\sum_{\beta}
     I_{\bf k'}^{\sigma ' \beta}\delta(\omega '-\sigma ' \omega_{\bf k'}^\beta) \biggr]\\
  &\times \biggl[ \left< \delta E^2 \right>_{{\bf k - k'},\omega-\omega '}^0
     +\sum_{\sigma ''}\sum_{\gamma} I_{\bf k -k'}^{\sigma '' \gamma}
     \delta(\omega-\omega ' -\sigma '' \omega_{\bf k -k'}^\gamma) \biggr]
    \end{split}
  \end{displaymath}
 \end{frame}

\begin{frame}[noframenumbering]
   \begin{displaymath}
     \begin{split}
&-\frac{4}{\pi}\int d{\bf k '} \int d\omega '\
\frac{1}{k^2|\epsilon({\bf k'},\omega ')|^2}
\biggl\{ \frac{[\chi^{(2)}({\bf k'},\omega '|{\bf k-k'},\omega-\omega ')]^2}
{\epsilon({\bf k - k'},\omega-\omega ')}\sum_{\sigma}\sum_{\alpha}
   I_{\bf k}^{\sigma \alpha}\delta(\omega-\sigma \omega_{\bf k}^\alpha)\\
  &-\frac{|\chi^{(2)}({\bf k'},\omega '|{\bf k-k'},\omega-\omega ')|^2}
   {\epsilon^*({\bf k},\omega)}
     \biggl[ \left< \delta E^2 \right>_{{\bf k - k'},\omega-\omega '}^0
     +\sum_{\sigma ''}\sum_{\gamma}I_{\bf k -k'}^{\sigma '' \gamma}
     \delta(\omega-\omega ' -\sigma '' \omega_{\bf k -k'}^\gamma)\biggr]\biggr\}\\
     &\times\sum_ae_a^2\int d{\bf v}\,
     \delta(\omega '-{\bf k \cdot v})F_a({\bf v})\\
-&\frac{4}{\pi}\int d{\bf k '} \int d\omega '\
\frac{1}{|{\bf k - k'}|^2|\epsilon({\bf k-k'},\omega-\omega ')^2}
\biggl\{ \frac{[\chi^{(2)}({\bf k'},\omega '|{\bf k-k'},\omega-\omega ')]^2}
   {\epsilon({\bf k'},\omega ')}\\
  &\times\sum_{\sigma}\sum_{\alpha}
  I_{\bf k}^{\sigma \alpha}\delta(\omega-\sigma \omega_{\bf k}^\alpha)
-\frac{|\chi^{(2)}({\bf k'},\omega '|{\bf k-k'},\omega-\omega ')|^2}
{\epsilon^*({\bf k},\omega)}\\
&\times \biggl[\left< \delta E^2 \right>_{{\bf k'},\omega '}^0
   +\sum_{\sigma '}\sum_{\beta}I_{\bf k'}^{\sigma ' \beta}
   \delta(\omega '-\sigma ' \omega_{\bf k'}^\beta)\biggr]\biggr\}
   \sum_a e_a^2\int d{\bf v}\,
   \delta(\omega-\omega '-{\bf (k-k')\cdot v})F_a({\bf v}).
     \end{split}
   \end{displaymath}
 \end{frame}


\message{ !name(apres_quali.tex) !offset(736) }

\end{document}