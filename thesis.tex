% \documenclass: mandatorio, indica o tipo/formato de documento
% \documentclass[12pt,a4paper]{abntex2}
\documentclass[12pt,a4paper,ruledheader]{report}
\usepackage[inner=2.0cm,outer=2.0cm,top=3.0cm,bottom=2.0cm]{geometry}
\usepackage{enumerate}
\usepackage[inline]{enumitem}
%\usepackage{chngcntr}
\usepackage[toc]{appendix}
\usepackage{pdflscape}
%\usepackage{rotating}
\usepackage[portuguese,brazilian,english]{babel}
%\usepackage[portuguese,brazil,brazilian]{babel}
%\usepackage[utf8]{inputenc} %comment when using luatex
\usepackage[utf8]{luainputenc} %comment when using pdftex
\usepackage{lmodern}
\usepackage[T1]{fontenc}
% \usepackage{ae}
\usepackage{fontspec}
%\usepackage{lxfonts}
\usepackage{nomencl}
%\makenomenclature
%\renewcommand{\rmdefault}{lmss}
%\renewcommand{\rmdefault}{ptm}
\renewcommand{\sfdefault}{lmss}
%\renewcommand{\sfdefault}{ptm}
%\renewcommand{\ttdefault}{ptm} 
\usepackage{amsmath}
\usepackage{amsfonts}
\usepackage{mathrsfs}
\usepackage{amssymb}
\usepackage{bm}
% melhoras na justificação
\usepackage{microtype}
\usepackage{setspace}
\usepackage{float}
\onehalfspacing
\usepackage{units}
\usepackage{parskip}
\usepackage{stmaryrd}
\usepackage{appendix}
\usepackage{epigraph}
%angulos em graus
%\usepackage{gensymb}
%símbolo de grau (não funciona com package gensymb
%um dos dois tem de ser comentado)
\newcommand{\degree}{\ensuremath{^\circ}}
\usepackage{indentfirst}
\usepackage{graphicx}
\usepackage{pdfpages} 
\usepackage{color}
\setlength{\parindent}{1.25cm}
%\setlength{\parskip}{0.2cm}
%\setlength\parindent{0pt}
%\setmarks{chapter}
\usepackage{listofsymbols}
\usepackage{fancyhdr}
% \usepackage{titlesec}
% \usepackage{ifthen}
% \usepackage{calc}
% \usepackage{abntex2abrev}
% \usepackage{setspace}
\usepackage{url}
\usepackage{cite}
% \usepackage[english,hyperpageref]{backref}
% Paginas com as citações na bibl
% \usepackage[num]{abntex2cite}
%\setlength{\headheight}{15.2pt}
\usepackage[breaklinks=true]{hyperref}
% \makeatletter
\hypersetup{
  colorlinks=true,    % false: boxed links; true: colored links
  linkcolor=blue,     % color of internal links
  citecolor=blue,     % color of links to bibliography
  filecolor=magenta,  % color of file links
  urlcolor=blue,
  bookmarksdepth=4
  pdftitle={Thesis}, 
  pdfauthor={Sabrina Tigik Ferr\~ao},
  pdfsubject={Thesis},
  pdfkeywords={plasma}{plasma kinetic theory}{collisions}
  {quasilinear theory}{weak turbulence theory} 
}
% \makeatother
\usepackage{cleveref}
\def\calE{{\cal E}}
\def\calL{{\cal L}}
\def\bk{{\bf k}}
\def\bq{{\bf q}}
\def\bv{{\bf v}}
\def\bu{{\bf u}}
\usepackage{makeidx}
%\maketitle
\makeindex
\definecolor{teal}{rgb}{0.0, 0.5, 0.5}
\definecolor{darkcyan}{rgb}{0.0, 0.55, 0.55}
\definecolor{darkblue}{rgb}{0.34,0.49,0.6}
\definecolor{darkgrayishcyan}{rgb}{0.275,0.38,0.35}
\definecolor{aurometalsaurus}{rgb}{0.43, 0.5, 0.5}
\definecolor{arsenic}{rgb}{0.23, 0.27, 0.29}
\definecolor{skobeloff}{rgb}{0.0, 0.48, 0.45}
\definecolor{skyblue}{rgb}{0.53, 0.81, 0.92}
\definecolor{raspberry}{rgb}{0.89, 0.04, 0.36}
\definecolor{amaranth}{rgb}{0.9, 0.17, 0.31}
\definecolor{softred}{rgb}{0.92,0.345,0.46}
\definecolor{darkelectricblue}{rgb}{0.33, 0.41, 0.47}
\definecolor{slategray}{rgb}{0.44, 0.5, 0.56}
\definecolor{cadet}{rgb}{0.33, 0.41, 0.47}
\definecolor{lapislazuli}{rgb}{0.15, 0.38, 0.61}
\definecolor{airforceblue}{rgb}{0.36, 0.54, 0.66}
\definecolor{cerulean}{rgb}{0.0, 0.48, 0.65}
\definecolor{chromeyellow}{rgb}{1.0, 0.65, 0.0}
\definecolor{deepsaffron}{rgb}{1.0, 0.6, 0.2}
\definecolor{fulvous}{rgb}{0.86, 0.52, 0.0}
\definecolor{pumpkin}{rgb}{1.0, 0.46, 0.09}
\definecolor{lilac}{rgb}{0.78, 0.64, 0.78}
\begin{document}

\frenchspacing 
\begin{titlepage}
  \begin{center}
  \centering{
    \textsc{\Large \sffamily {Universidade Federal do Rio Grande do Sul}}\\
    \textsc{\Large \sffamily {Programa de Pós-Graduação em Física}}\\
    \vspace{0.15cm}
    \textsc{\large  Doctoral Thesis}}\\
\end{center}

\vspace{6cm}

\vfill
\begin{center}
\renewcommand{\thefootnote}{\fnsymbol{footnote}}
\setcounter{footnote}{2}
{\Large \sffamily \textbf{Time evolution of weakly turbulent processes in
    the presence of collisional interactions in astrophysical plasmas}\footnote[1]{
    Project financed by the by the Conselho Nacional de
    Desenvolvimento Científico e Tecnológico (CNPq) and the
    Coordenação de Aperfeiçoamento Pessoal de Nível Superior (CAPES).}}\\
{\large \sffamily (Evolução temporal de processos fracamente turbulentos
  na presença de efeitos colisionais em plasmas astrofísicos)}\\
\vspace{1.5cm}
\Large Sabrina Tigik Ferrão\\
\end{center}
\vspace{3cm}
\vfill
\hfill
\parbox{.5\textwidth}{Doctoral Thesis prepared under the supervision of
  Professor Luiz Fernando Ziebell, presented to the Physics Graduation
  Program of the Instituto de F\'{\i}sica at UFRGS, in partial fullfillment
  of the requirements for obtaining the title of Doctor in Sciences.}\\
\vfill
\vspace{1cm}
\begin{center}
  { Porto Alegre\\ \today}
\vspace{0.5cm}
\renewcommand{\thefootnote}{\arabic{footnote}}
\end{center}

\end{titlepage}

%%% Local Variables:
%%% mode: latex
%%% TeX-master: "thesis"
%%% End:

\pagenumbering{gobble}
%\input{folhadeaprovacao}
\newpage
\pagestyle{empty}
\begin{center}
  \textbf{\large Acknowledgments}
\end{center}
  
First of all, I acknowledge the dedication of my supervisor, Luiz Fernando
Ziebell, whose guidance, encouragement, and patience (tons of) made this
work possible. Further, I would like to state my sincere appreciation for
the trusting he always demonstrated to have in my capabilities for developing
high-quality research and for his availability to discuss my work, giving me
insightful advice. 

In the academic scope, I would like to thank Peter Yoon for the years of
productive collaboration and for receiving me with great hospitality during
my stay at the University of Maryland.

I also want to express my earnest appreciation to Dr.~Adolfo Viñas, who
dedicated a part of his time to teach me PIC simulation and discuss the
subtleties of the kinetic theory of plasmas during my time in the University
of Maryland. As special mentions, I would like to thank Jeferson Arenzon,
Renato Pakter, and Rudi Gaelzer for their willingness to help me with my
postdoc applications when I randomly appeared at their offices.

From the Graduate Program in Physics, I am grateful to Liane Thier Ruschel for
the kind and efficient assistance in dealing with bureaucratic procedures and
the most diverse document requests. I also thank Mari Nunes for her patience
and goodwill in dealing with everyday trouble of grad students.

On the personal side, I thank Felipe for his comprehension, patience, support
and encouragement during my undergrad and graduate years. I am also grateful
to Gabriel for his love and refined sense of humor that cheers me up even in
the saddest days.

Agradeço especialmente à minha mãe, Diná, por ter me criado para ser uma
mulher livre e a não aceitar qualquer papel na sociedade que não seja o
que eu considere o melhor para mim. \emph{Mãe, essa tese é dedicada a ti.
  Sem teu exemplo, eu não estaria onde estou. Obrigada por tudo!}

From my friends and colleagues from UFRGS, I am grateful to Nicole for the
companionship that ranges from hard-working weekends to life celebrations.
\emph{Nicole, grad school would be terribly boring without you.} I also thank
Amanda for showing me how to be prepared for life, Demétrius for the cheerful
annoyance, and Vinícius for the light talks and for sharing his delicious
bread. A special thanks to Thales, my example of plasma physicist, and to
Fernanda, whose intelligence and dedication will always be my inspiration. 
Special mentions to colleagues that were part of this journey and deserved
to be remembered: Alexandre, Gustavo, João and Larissa.

From Maryland, I am especially grateful to Jacek for the infinite talks, solid
companionship and (mutual) support.\emph{ Jacek, you made my time in CP much
  better. Dziękuję bardzo!}






  
  
  

%\end{quote}

%%% Local Variables:
%%% mode: latex
%%% TeX-master: "thesis"
%%% End:

%\newpage
\pagestyle{empty}
\chapter*{}
\vspace{17.5cm}
\vfill
\hfill
\parbox{.6\textwidth}{
{\it``So long, and thanks for all the fish.''}}
\begin{flushright}
(Douglas Adams)
\end{flushright}


%\input{resumo}
%\input{param}{\pagestyle{empty}}
%\pagenumbering{none}
\begin{abstract}
  The weak turbulence theory has been an important theoretical tool
  for the study of nonlinear kinetic instabilities in plasmas. For
  a long time, this theory treated exclusively of the study of
  oscillatory processes and its influence in the plasma dynamics. The
  long-lasting timescale of nonlinear processes, however, suggests
  that collisional processes might have some effect in the late plasma
  dynamics, acting alongside nonlinear collective effects. In a recent
  work [P. H. Yoon et al., Phys. Rev. E 93:033203 (2016)], collisional
  effects and collective processes were systematically incorporated,
  starting from first principles, in the weak turbulence theory equations,
  considering electrostatic oscillations. The outcome of this innovative
  approach was a formal mathematical expression for the collisional
  damping rate for Langmuir and ion-sound waves, and the discovery of a new
  fundamental  process of emission of electrostatic fluctuations, in the
  ion-sound and Langmuir frequency range, caused by binary interactions of
  particles, named \emph{electrostatic bremsstrahlung}. In the present study
  we introduce the first numerical analysis of these two new equations and
  discuss the relevance of these numerical results in the solar physics scope.
  The first work to be discussed concerns the collisional damping equation, 
  and in it we compare the results of numerical integration of the rigorous
  expression with the damping rate calculated with the widely applied
  \emph{Spitzer formula}. We show that the Spitzer approximation highly
  over-estimates the intensity of collisional attenuation of plasma waves. 
  The lack of relevance of the collisional damping rate gets further 
  demonstrated when we compare it with the collisionless (Landau) damping
  rate. In the second work to be discussed, we analyze the so-far ignored
  electrostatic bremsstrahlung effect. We show that the presence of
  electrostatic bremsstrahlung emission modifies the Langmuir spectrum,
  which in turn alters the shape of the initial electron velocity
  distribution, assumed to be Maxwellian. After a long time-evolution period
  (numerical integration), the system seems to arrive to a new quasi-steady
  state, in which the shape of the electron velocity distribution resembles
  the shape of a core-halo distribution function, i.e., composed by a
  Maxwellian core and a suprathermal tail. The outcomes of both analyses are
  unprecedented; the prospects and possibilities for further studies on this
  subject are promising. In the end, we also present an extra result,
  indirectly related to the main subject of this doctoral project, regarding
  the analysis of the complete set of electromagnetic weak turbulence equations,
  in the presence of a core-halo velocity distribution function. 
\end{abstract}
\newpage
\begin{otherlanguage}{brazilian}
  % \selectlanguage{portuguese}
  \begin{abstract}
    A teoria de turbulência fraca tem sido uma importante ferramenta
    teórica para o estudo de instabilidades cinéticas não lineares
    em plasmas. Por um longo tempo, esta teoria tratou exclusivamente
    do estudo de processos oscilatórios e de sua influência na dinâmica
    do plasma. Entretanto, a escala de tempo de longa duração dos
    processos não lineares sugere que processos colisionais podem ter
    algum efeito na dinâmica do plasma, atuando em conjunto com os
    efeitos dos processos coletivos não-lineares. Em um trabalho recente
    [P. H. Yoon et al., Phys. Rev. E 93:033203 (2016)], efeitos colisionais
    e processos coletivos foram sistematicamente incorporados, partindo
    de primeiros princípios, nas equações da teoria de turbulência fraca,
    considerando oscilações eletrostáticas. O resultado dessa abordagem
    inovadora foi uma expressão matemática formal para a taxa de amortecimento
    colisional para ondas de Langmuir e íon-acústicas, e a descoberta de um
    novo processo fundamental de emissão de flutuações eletrostáticas, na
    faixa de frequência das ondas de Langmuir e das ondas íon-acústicas,
     causado por interações binárias entre partículas do plasma, nomeado como
    \emph{bremsstrahlung eletrostático}. Neste estudo, introduzimos as
    primeiras análises numéricas relativas a essas duas novas equações e
    discutimos a relevância desses resultados numéricos no contexto da física
    solar. O primeiro trabalho a ser discutido se refere à nova equação para o 
    amortecimento colisional, e nele comparamos o resultado da integração
    numérica dessa expressão, com a largamente usada \emph{fórmula de Spitzer}.
    Com isso conseguimos mostrar que a aproximação de Spitzer superestima
    em muito a intensidade da atenuação colisional das ondas de plasma. Além
    disso, a falta de relevância da taxa de amortecimento colisional fica
    demonstrada quando a comparamos com a taxa de amortecimento não colisional
    (de Landau). No segundo trabalho, analizamos o até então ignorado efeito
    de \emph{bremsstrahlung} eletrostático. Mostramos que a presença da
    emissão de \emph{bremsstrahlung} eletrostático modifica o espectro das ondas
    de Langmuir que, por sua vez, altera a forma da função de distribuição
    inicial dos elétrons, suposta Maxwelliana. Após um longo tempo de evolução
    temporal (integração numérica), o sistema parece chegar a um novo estado
    quase-estacionário, no qual a forma da função de distribuição de velocidades
    dos elétrons lembra a forma de uma função de distribuição núcleo-halo, ou
    seja, composta por um núcleo Maxwelliano e uma cauda supratérmica. Os
    resultados de ambas análises são sem precedentes; as perspectivas e as
    possibilidades para novos estudos neste assunto são promissoras. No final é
    também apresentado um resultado extra, indiretamente relacionado ao assunto
    principal desse projeto de trabalho de doutorado, relativo à análise do
    conjunto completo de equações eletromagnéticas da teoria de turbulência
    fraca, na presença de uma função de distribuição de velocidades núcleo-halo. 
   \end{abstract}
\end{otherlanguage}
% \input{resumo}
\pagenumbering{roman}
\tableofcontents{\thispagestyle{empty}}
\listoffigures{\thispagestyle{empty}}
\pagestyle{fancy}
\pagenumbering{arabic}
\fancyhf{}
\renewcommand{\headrulewidth}{0.5pt}
\lhead{\nouppercase{\it \leftmark}}
\rhead{\thepage}
\setcounter{page}{0}
\chapter{Introduction}
\label{cha:intro}

In the context of plasma kinetic theory, processes involving waves
and kinetic instabilities, the so-called \emph{collective processes},
are almost exclusively associated with situations in which the effects
of collisional dissipation can be neglected, what happens when the
studied phenomenon evolves in a much shorter time-scale than the
collisional relaxation time of the plasma in question \cite{akhi1967}.
In such cases, the plasma dynamics may be described by the Vlasov-Maxwell
system, a complex set of coupled equations whose solution invariably
will depend on some degree of approximation. In the presence of small
amplitude oscillations in a fully ionized plasma, we can make use
of perturbation theory in order to solve such complicated system
\cite{klimo}. From this method, one may obtain simpler approximations
like the linear theory and the quasilinear formalism, and also
more complex formulations, like the weak turbulence theory, which
takes into account low-order nonlinearities on the plasma dynamics.

With the lowest order of this chain of perturbative approximations, the
linear theory, it is possible to obtain the dispersion relations, which
are mathematical expressions that not only help us on identifying the
different oscillatory modes that may be excited in plasmas, but also
wholly characterize these oscillations. Valuable information like
how these waves propagate and the frequencies that will resonate with
the plasma particles, resulting in damping or amplification of these
oscillations \cite{bitt,chen,gurnett2017}, are given by the dispersion
relations. This information, however, is static. There is no way to
know, based solely on the linear formulation, any further development
of the system, how these waves will evolve in time from a given initial
state, for instance. Thus, to clarify if these oscillations will grow
until their amplitudes become too large to be treated by a perturbative
method, or if they will rapidly increase and saturate, or if they will
be completely damped, or if they will decrease a little and then saturate, 
a certain degree of nonlinearity must be taken into account. That is the
purpose of the quasilinear approximation.

In the quasilinear theory, low-order nonlinear terms are kept in
the equation for the time evolution of the velocity distribution
function. The velocity distribution will then evolve in time through
a process of diffusion in velocity space, in which the diffusion
coefficient depends on the spectral energy density of the waves.
The dispersion relation is formally given by the same expression
obtained under the linear theory, except that now it depends on a
time-dependent velocity distribution function, which must vary in a
much slower time-scale than the period of the plasma oscillations.
Under such condition, the gradual changing on the velocity space
shape slowly modifies the shape of the wave spectrum, whose new form
will alter the velocity distribution function and so on, until a new
quasi-stationary equilibrium state is attained \cite{gurnett2017}.
However, depending on the phenomenon being described, this saturation
state reached under the quasilinear approximation might keep on evolving
if higher order nonlinearities are taken into account \cite{akhi2}.
This kind of nonlinear analysis is given by the
\emph{weak turbulence theory}, which is the next step in this chain
of perturbative approximations.

Mostly developed in the period between the late $1950$s and early
$1970$s \cite{Kadomtsev1965,Tsytovich1967,saggal,tsynlep,david,
  tsyitotp,akhi2,tsytotp,mel}, the weak turbulence theory has
become a valuable theoretical tool for the analysis of nonlinear
phenomena in plasmas since then, being applied in several studies
until nowadays \cite{Grognard1982,McClements87b,Hanssen1991,
  Edney2001,Kontar2001,Kontar2002,THD13}. Its complex formulation
is constituted by a set of coupled kinetic equations, which describe
the time evolution of the velocity distribution functions of the
plasma particles and the time evolution of the spectral intensities of
the plasma wave modes. The development of the weak turbulence theory
was resumed in a relatively recent series of papers, where the author,
starting from first principles, has extended the original formalism in
order to include new effects. In the initial work, it was considered
only the propagation of electrostatic oscillations, taking into account
the effects of both wave-wave and wave-particle interactions \cite{Yoon00}.
Later, the formalism was expanded, and discrete particle effects,
related to spontaneous emission and spontaneous scattering processes,
were included~\cite{Yoon05a}. In a further extension, the effects of
the propagation of electromagnetic waves were incorporated into the
revised formulation \cite{Yoon06,Yoon2012b}.

The equations of this revised version of the weak turbulence theory
have been the subject of extensive studies involving nonlinear analysis
and two-dimensional numerical integration applied to the time evolution
of the Langmuir turbulence \cite{ZGPY08,Ziebell2012,YZGLW12,
  ZYGP14a,ZYGP14b} in a fully ionized and unmagnetized plasma, and in
other studies that also include the emission of electromagnetic waves
\cite{ZYSGP14c,ZYPGP15,ZPYGP16}. Some studies using a one-dimensional
approach have also been made \cite{ZiebellGY01,GaelzerZY02,GZVYR08}.
The Langmuir turbulence is a by-product  of the instability generated
when an energetic electron beam interacts with a background plasma. The
electron beam alters the electron velocity distribution triggering the
so-called \emph{bump-in-tail instability}. Besides being recurrently used
in textbooks as an illustrative example of the quasilinear diffusion process
\cite{akhi2,chen,gurnett2017}, the bump-in-tail instability also has an
essential role in the study of type II and type III radio bursts
\cite{Emslie1984,Hannah2009,ZVST11,Hannah2011,KK12,Reid2014,BNKR14}. In
this context, the relevance of a more accurate description of nonlinear
kinetic instabilities in plasmas, represented here by the beam-plasma
interaction, becomes clear.

The next step towards a more complete description of weakly-turbulent
processes in plasmas is the inclusion of collisional effects in the
formalism. So far, the effects of binary collisions have been neglected
under the time-scale argument, mentioned in the first paragraph. This
assumption can be considered quite precise for the evolution of the
bump-in-tail instability under the quasilinear approximation. In such
regimen, the plasma quickly saturates in a quasi-equilibrium state and
the time evolution entirely ceases. However, when nonlinear processes
are considered in the dynamics, one cannot be sure if this argument
still holds. Basically, for the system to evolve inside the weak
turbulence theory limitations, the incident beam must be tenuous, and
the oscillations must have low amplitude (after all, the turbulence
ought to be weak). Under such conditions, the time evolution of the
system occurs in a time interval that is way longer than the oscillation
period of the plasma waves being studied. In fact, there are studies that
show that nonlinear effects keep acting in the system in a time-scale that
goes far beyond the saturation time of the quasilinear instability,
to the extent that an asymptotic analysis of the new quasi-equilibrium
state of the turbulent process becomes relevant \cite{Yoon2012c,ZYGP14a,
  ZYGP14b,ZYSGP14c,ZYPGP15}.

The hypothesis about the probable relevance of binary collisions in the
nonlinear dynamics of the beam-plasma interaction was numerically tested
in \cite{Tigik2016a} (see \autoref{master}). In this work, a linearized
form of the Landau collision integral was added to the particle kinetic
equation, and the weak turbulence theory's complete set of self-consistent,
electrostatic equations was numerically integrated in two-dimensions.
It was shown that collisions indeed do affect the plasma dynamics in a
time-scale that is very close to the time-scale of action of nonlinear
effects. Beyond the purely theoretical conjecture, combining collective
processes and collisional effects have some important applications in
solar physics. In the solar hard X-ray emission analysis, for instance, the
models used for interpreting the spectra measured by the Reuven Ramaty High
Energy Solar Spectroscopic Imager (RHESSI)\footnote{For more information
 about the mission see: \url{https://hesperia.gsfc.nasa.gov/rhessi3/}.}
that employ quasilinear/nonlinear wave instabilities with some kind of
collisional dissipation (collisional damping for the waves, binary
collisions for the particles) are usually empirically adapted merely by
adding an \emph{ad hoc} expression into the particles and wave kinetic
equations to fit the observed data \cite{PK95,Kohl1998,Veronig05,Brown2006,
  Kontar2011,KKDSSK11}. However, there is no rigorous theory to compare
and support such models. Moreover, though the formalism in \cite{Tigik2016a}
makes use of the full Landau collision integral in the particle kinetic
equation, it does not take into account collisional damping effects in the
wave equation, which is widely used in these empirical models. The point
is: until that moment, there was not a proper theoretical tool to deal
with the combination of collective processes and collisional interaction.

In a recent paper \cite{YZKS16}, it was presented the first rigorous
theory that, starting from first principles, incorporates collisional
effects into the already well established weak turbulence formalism.
The authors used the same standard weak turbulence perturbative method
but, instead of keeping only the collective eigenmodes in the linear
and nonlinear wave-particle interactions, they also kept the effects
of the non-collective fluctuations emitted by thermal particles. The
outcome is a complete nonlinear description, from first principles, of
the propagation of electrostatic oscillations in the presence of both
wave and particle collisional dissipation and a new equation describing
a hitherto unknown process. This new effect depicts the emission of
electrostatic radiation, in the eigenmode frequency range, caused by
particle scattering. Since this is a form of braking radiation, the
authors named it as \emph{electrostatic bremsstrahlung}.


\section{Applications to astrophysical plasmas}
\label{appE}
Astrophysical plasmas are observed in a vast range of densities, temperatures,
magnetic fields, ionization degree, and scale lengths\cite{peratt2014}. Starting
in the outer layers of Earth's atmosphere, passing through the solar wind in the
interplanetary space, to the stellar interiors and atmospheres, accretion disks
and molecular clouds, it is estimated that something around $99 \sim 99.9\%$ of the
observable matter in the universe is in the plasma state. Under cosmic conditions,
even the weakly ionized gas of the neutral hydrogen regions around galaxies or in
the atmosphere of cold stars have a strong reaction to electromagnetic fields,
exhibiting the characteristic collective behavior of plasmas and, therefore, are
also considered plasmas\cite{tsypa,may,peratt2014}.

As soon as the pervasiveness of the plasma state in space came as a fact,
plasma physics began to be recognized as an essential component in the
multidisciplinary framework of astrophysics \cite{Alfven1950,Parker1956,
  Alfven1961,Gailitis1964,Alfven1971}. At the same time, the analysis and
interpretation of observational data started to have increasing relevance
in the progress of theoretical, computational, and experimental plasma
physics research \cite{YZKS16,Donnert2014,Pezzi2018,Howes2018,Peterson2019}.
In this context of mutual collaboration, the solar and interplanetary
plasma research has a pivotal contribution. The accessibility to direct,
in situ measurements, highly resolved remote sensing and high-resolution
spectroscopy, turned the solar system into a space plasma laboratory.
Processes like magnetic reconnection, particle acceleration, shocks, and
turbulence can be closely observed by space probes, and the understanding
acquired may be extended or adapted to address similar phenomena observed
in remote astrophysical environments \cite{NAP10477}. 

The theory developed during this graduate research analyses the combined
action of nonlinear kinetic instabilities and collisional processes in
the context of solar physics. To contextualize our work within the solar
plasma research, in Subsections~\ref{solar-physics},~\ref{sun} and
\ref{solar-corona} we present a brief review of the current research in
solar and space physics and the phenomena related to this work. The
corresponding paper is pointed in a footnote.

\subsection{Solar and space physics}
\label{solar-physics}
The domain of solar and space physics is the \emph{heliosphere}, a cavity
in the interstellar medium, created by the constant supersonic flow of solar
plasma that permeates the interplanetary space. This region encloses the
solar system and extends for approximately $140\unit{AU}$\footnote{An AU
  (astronomical unit) is defined as the mean distance between the Sun and
  Earth, $1\unit{AU}\approx 150\times 10^6\unit{km}$.} in the surrounding
galactic space. Its outer limit is the \emph{heliopause}, an interface
zone where the outgoing pressure of the solar plasma balances the incoming
pressure from the interstellar medium. This comet-shaped plasma bubble
\cite{Dialynas2017}, also has an inner boundary, the \emph{termination shock},
which delimits the sphere of absolute influence of the solar plasma. Beyond
the termination shock, at roughly $100\unit{AU}$ from the Sun, extending
out to the heliopause, lies the \emph{heliosheath}. In this region, the
solar wind particles start being subjected to increasing external pressure,
transitioning to a subsonic, turbulent flow that blends and interacts
directly with the matter in the local interstellar medium \cite{NAP13060}.
In \autoref{figE1}, we have a depiction of the regions and boundaries
described above, and the current location of both Voyager missions, that
are already outside the heliosphere, in the interstellar space.
\begin{figure}[h]
  \begin{center}
    \includegraphics[width=0.892\textwidth]{solar_system_env.png}
    \caption{Main boundaries of the heliosphere and the location of both
      Voyager missions.}
      {\footnotesize Credit:}
      \href{https://voyager.jpl.nasa.gov/news/details.php?article_id=112}
      {\footnotesize NASA/JPL-Caltech}{\footnotesize , accessed at 08/08/2019.}
    \label{figE1}
  \end{center}
\end{figure}


The ultimate goal of heliophysics can be summarized as the search for a
better understanding of the solar dynamics and how it affects the Earth,
other planetary bodies, and the interstellar medium. The heliosphere is
a coupled system, where different interacting elements that might look
unrelated at first sight are connected by underlying plasma processes
\cite{NAP13060}. A great example of this interconnection is the relation
between the appearing of sunspots and the occurrence of non-recurrent
geomagnetic storms, which are the most intense space weather disturbances.
Evidence of such relationship started to appear at the end of the $19$th
century. However, the correct hypothesis supporting this correlation
(coronal mass ejections associated with solar flares), which depended
on advances in plasma physics theory, came along only in $1929$
\cite{chapman1929cosmical,Koskinen2011}.

\subsection{The Sun and the near-Earth environment}
\label{sun}
A better comprehension of the Sun and the dynamic space around our planet
goes beyond the pure scientific curiosity. This accessible space physics
laboratory can indeed lead to discoveries that can be applied and extended
to remote cosmic systems. However, the primary reason behind the international
effort (see Figures \ref{nasa} and
\ref{esa}) on understanding the complex physics of the Sun and interplanetary
space is to unlock means of predicting the occurrence and intensity of solar
activity, and anticipate how the near-Earth environment will be affected.
Reliable forecasting of the space weather is essential for protecting
electronic equipment, radio communications, GPS signal and humans in space
from the effects of the interaction between the dense, magnetized plasma of
coronal mass ejections with the geomagnetic field, and also from the
energetic radiation emitted by solar flares\footnote{The emission of hard X-rays
  and type III radio bursts, both generated during solar flares, are addressed in
  \cite{Tigik2016a} (see \autoref{master}).}. Extreme coronal mass ejections
can also cause stronger geomagnetic disturbances that may affect terrestrial
infrastructure such as electric power grids \cite{NAP13060}.
\begin{figure}[hb]
  \begin{center}
    \includegraphics[width=0.892\textwidth]{hso_fleet_chart_april_2019.pdf}
    \caption{NASA's present and future heliophysics system observatory. The number
      between parenthesis under the probe's name is the total number of spacecrafts
      that compose the mission.}
      {\footnotesize Credit:}
      \href{https://science.nasa.gov/heliophysics}
      {\footnotesize NASA}{\footnotesize , accessed at 10/08/2019.}
    \label{nasa}
  \end{center}
\end{figure}
\begin{figure}[ht]
  \begin{center}
    \includegraphics[width=0.892\textwidth]{ESA_s_fleet_of_Solar_System_explorers_2019.jpg}
    \caption{ESA's past, present, and future science missions for studying the solar system.}
      {\footnotesize Credit:}
      \href{https://www.esa.int/spaceinimages/Images/2019/02/ESA_s_fleet_of_Solar_System_explorers}
      {\footnotesize ESA}{\footnotesize , accessed at 10/08/2019.}
    \label{esa}
  \end{center}
\end{figure}

\subsection{The solar corona}
\label{solar-corona}
The source of the tenuous magnetized plasma permeating the heliosphere, and
also the origin of solar particles and radiation that may cause disruptive
space weather around Earth, is the outermost layer of the Sun: the solar
corona \cite{Cranmer2017,Klein2017}. The solar corona can be easily recognized
as the bright halo that appears around the hidden Sun during solar eclipses.
\autoref{corona} shows two beautiful photographs of the corona, taken in two
different solar eclipses, August 21 $2017$ and July 2 $2019$. In both pictures,
one can see the moon covering the bright photosphere, allowing us to have a
glimpse of the thin, but ultra-hot coronal plasma.

Counter-intuitively, the solar corona is, at the same time, the most tenuous
and the hottest outer layer of the Sun. At the top of the chromosphere, where
the temperature might reach $20,000\,\unit{K}$, in a relatively narrow interface,
the \emph{solar transition region}, the plasma temperature rises abruptly to
$\sim 10^6\,\unit{K}$ \cite{BenzA2002}. It has been almost eighty years since
the extreme coronal temperature came as the only reasonable explanation for the
origin of the spectral lines observed in the solar corona \cite{Hunter1942},
in a classical case of an answer that brings a more troublesome question. 
Since then, several mechanisms have been proposed \cite{Parker1988,Scudder92b,
  Vinas2000,PP14}, not only to comprehend the heating of the solar corona,
but also the acceleration of the solar wind. Another unsolved problem that is
directly connected to the inverted temperature profile of the solar corona and
particle acceleration in the Sun, is regarding where and how the non-Maxwellian
velocity distribution functions with suprathermal tails, observed in the solar
wind, are formed\footnote{The generation of suprathermal electrons in the lower
  corona and its consequences to the coronal heating problem are tackled in
  \cite{Tigik2017a} (see \autoref{sec:gen-supr}).}.

\begin{figure}[h!]
  \begin{center}
    \includegraphics[width=0.892\textwidth]{corona2017.png}
    \includegraphics[width=0.892\textwidth]{corona2019.png}
    \caption{Solar corona captured during the total solar eclipse of 2017 (top panel)
      and 2019 (lower panel). The reddish contour at some regions is the chromosphere,
      and the reddish arcs are solar prominences, chromospheric plasma structures that
      penetrate the coronal region.
      {\footnotesize Credit:}
      \href{https://hdr-astrophotography.com}
      {\footnotesize Nicolas Lefaudeux}{\footnotesize , accessed at 15/08/2019}.}
    \label{corona}
  \end{center}
\end{figure}

\vspace{0.5cm}

\section{Thesis organization}
\label{sec:thesis-organization}
Considering the context presented above, the working proposal of this PhD
Thesis is to put forward the first numerical analysis of the new weak
turbulence theory for collisional plasmas \cite{YZKS16} and explore its
applications in the solar physics scope. In \Cref{cha:kin-theo}, we make
a brief review of the basic plasma kinetic theory for electrostatic
oscillations. \Cref{cha:weak-turb} revises the procedure for obtaining
the weak turbulence equations for Langmuir and ion-sound waves and
the quasilinear equation for the particle dynamics.

The main subject of this thesis is discussed in \Cref{noneigenmode}. In
\Cref{sec:inc-non} we show the methodology used in the formulation of this
new generalization and the outcome of its inclusion in the quasilinear
wave equation, and in the particle kinetic equation. \Cref{sec:non-non}
describes the inclusion of the noneigenmode fluctuations in the nonlinear
wave equations and the resulting terms, which are the main concern of
the present analysis. In \Cref{main}, we contextualize the present work
regarding the theory discussed in the previous chapter and present the
results obtained in appended papers. \Cref{cha:final}, summarizes the
results and discuss some remarking conclusions obtained in this analysis.
At the end of this chapter, we also elaborate on the perspectives for this
new generalized formalism and its relevance for the next generation of
solar physics research.
% LFZ190822: Suponho que aqui devia ser "... a brief discussion on the
% conclusions obtained and on the perspectives ..." ?
% Sabrina: Melhorou?
% and its relevance for the next generation of solar physics research are
% also deliberated.

In the end, we have four appendices. \Cref{appA} depicts an improved
approximation for the second order susceptibility of the electrostatic
bremsstrahlung for Langmuir waves. In \Cref{appB} we discuss the
similarities of the asymptotic equilibrium state of the Langmuir waves
spectrum, considering a combination of an inverse power-law kappa
distribution and a Maxwellian distribution, in the presence and the
absence of the electrostatic bremsstrahlung effect. The discussion in
\Cref{appB} is related to the article appended in \Cref{appC}, which
is parallel research and is only slightly connected to the main subject
of this work. \Cref{master} shows the paper that summarizes the results
of my Master’s dissertation, which may be considered the starting point
of this doctoral research. % \Cref{appE} explores relevant aspects of
% current solar physics research and how they are connected to this work.
% It was written to be more like a light, informative reading, than a
% deep academic digression.


\chapter{Kinetic theory of plasmas}
\label{cha:kin-theo}
Plasmas are ionized gases that exhibit collective behavior. Composed of
positive and negative charges and neutral particles, a plasma is said to
be \emph{neutral} when the quantity of particles with opposite charges is
the same~\cite{klimon}, and \emph{nonneutral} when it is constituted - or
have a significant excess - of one kind of charge~\cite{DavidsonNonNeutral}.
For the present work, we assume a fully ionized hydrogen plasma, which is
overall neutral and does not contain any neutral particle. In such a case,
half the total number of particles is given by protons and the other half
by electrons.

Electrons and ions are sources of electric and magnetic fields. Due to the
long-range nature of these fields, particles can interact simultaneously, at
distance, with several other particles in the system. The effective range of
these interactions is determined by the \emph{Debye length}
\begin{equation}
  \label{debye}
  \lambda_{De}=\sqrt{\frac{k_BT_e}{4\pi n e^2}},
\end{equation}
where $T_e$ is the electron temperature\footnote{Due to the high mobility,
  electrons are the main responsible for maintaining the Debye shielding.
  There are specific situations where this is not true~\cite{chen}, but
  they are outside of the scope of the current study.}, $k_B$ the Boltzmann
constant, $n$ the plasma density and $e$ the electron charge. This means
that a specific charge will interact effectively only with other charges
that are inside of a virtual sphere with radius given by $\lambda_{De}$,
being almost completely shielded from the influence of particles outside
the \emph{Debye sphere}. The self-consistent interaction between charged
particles and electric and magnetic fields, creates fluctuations in the
local neutrality of the plasma, giving rise to the series of oscillations
and complex wave phenomena that characterize the plasma’s collective
behavior~\cite{chen}. Describing the precise state of a plasma, in a given
time, requires the knowledge of the $6N$ phase-space coordinates of each
particle, and also the microscopic amplitudes of the electric and magnetic
fields at each point in space~\cite{klimon}. Such exact representation is
clearly a non-practical approach for a many-body system like the plasma.
However, this microscopic formulation can be approximated to a macroscopic
description by taking the average of the microscopic quantities and making
suitable assumptions to truncate the correlation functions at an appropriate
point~\cite{klimon}. This will lead to the plasma kinetic equations, which is
the first of a hierarchy of approximations that also includes the multi-fluid
and magnetohydrodynamic theories. The application of each one of these
formulations will depend on the context and the scale of the phenomena being
described~\cite{chen}. In this work we are interested in small-scale processes
that depend on velocity-space properties, requiring a kinetic description of
the plasma.

Based on the concepts and methods of statistical mechanics, together with the
Maxwell equations, the kinetic theory is the most fundamental description of
the plasma dynamics, providing a formal, self-consistent evolution of plasma
processes. As in the case of neutral gases, the kinetic formalism of plasmas
takes into account the variations of the velocity distribution function
$f({\bf r},{\bf v},t)$ in the phase-space, for each particle species.
Statistically, this means that the product between the distribution function
and the volume element in the hexa-dimensional phase-space
$f_\alpha({\bf r},{\bf v},t)d^3r d^3v$ represents the probability of finding
a particle of species $\alpha$ in a volume element $d^3r$ around ${\bf r}$
and in a velocity volume element $d^3v$ around ${\bf v}$, at the instant $t$
\cite{klimon}.

Like in the case of non-ionized gases, the statistical description of the
plasma dynamics is given by the Boltzmann equation
\begin{equation}
\frac{\partial f_\alpha({\bf r},{\bf v},t)}{\partial t}
+{\bf v \cdot \nabla_r}f_\alpha({\bf r},{\bf v},t)
 +\frac{{\bf F}}{m_\alpha} {\bf\cdot \nabla_v}f_\alpha({\bf r},{\bf v},t)
=\left(\frac{\partial f_\alpha ({\bf r},{\bf v},t)}{\partial t}\right)_{\rm coll}.
\label{boltz}
\end{equation}

Above, on the left-hand side, the differential vector operators $\nabla_{\bf r}$
and $\nabla_{\bf v}$ represent, respectively, the gradients in ($x,y,z$) space
and ($v_x, v_y,v_z$) velocity coordinates. The quantity ${\bf F}$ is the acting
force in the system, and  $m_{\alpha}$ is the mass of the particles of kind
$\alpha$, which, in a fully ionized plasma, can assume two values: electrons
$\alpha=e$  and ions  $\alpha=i$. The right-hand side depicts the temporal
variation of the velocity distribution function due to collisions between the
system's particles.

Collisional interactions are the central point of the present thesis. However,
the formalism presented in \Cref{noneigenmode}, departs from the weak turbulence
equations, which are the kinetic description of nonlinear, low-amplitude
instabilities in a collisionless plasma. The contributions of the correlations
between \emph{source} fluctuations, which account for collisional interactions
(see.~\cite{klimon,Tigik2015}), are then retrieved with the inclusion of
noneigenmode contributions in the wave and particle kinetic equations\footnote{
  For instance, in \Cref{sec:inc-non}, it will be shown that including the
  noneigenmode fluctuations in the weak turbulence's formal particle kinetic
  equation leads to an expression that accounts for the collisional interactions
  that appear on the right-hand side of equation~\eqref{boltz}.}. Therefore,
at this point, it will be assumed that collisions can be neglected.

%% ~\cite{klimon,Tigik2015}
\section{The Vlasov-Maxwell system of equations}
% \label{vlasov-appro}
% When the time necessary for the plasma to establish statistical equilibrium
% is much greater than the characteristic time of the process being studied,
% collisions can be neglected. In this section, is presented a brief discussion
% about the necessary set of assumptions that allow us to disregard the effect
% of collisions in dilute plasmas. In the sequence we present the complete set
% of self-consistent equations for a collisionless plasma.

% \subsection{Plasma parameter and relaxation time}
% \label{plasma-param-relax}
% A plasma is considered diluted when the plasma parameter $g\ll 1$, which
% has the following form
% \begin{equation}
%   \label{ppar}
%   g=\frac{1}{n\lambda_{De}^3},
% \end{equation}
% where $n$ is the number of charges inside the Debye sphere, and $\lambda_{De}$
% is the Debye length, given by Eq. \eqref{debye}. At priory, the inequality
% $g\ll 1$ seems counter intuitive, since it means that you must have a large
% number of charged particles in the Debye sphere. However, if one substitutes
% $\lambda_{De}$ with its expression,
% \begin{align}
%   g&=\frac{4\pi n e^2}{n T_e}\sqrt{\frac{4\pi n e^2}{T_e}}\\
%    &=\frac{4\pi e^2}{T_e}\sqrt{\frac{4\pi n e^2}{T_e}}
% \end{align}

% % Let $\tau_{\rm rel}$ be the time necessary for the plasma to establish
% % statistical equilibrium, i.e, the relaxation time. 

% the relaxation time $\tau_{\rm rel}$ 
% The relaxation time of a plasma is defined
% as follows $\tau_{\rm rel}=\lambda_{De}/v_{th}g$, where $\lambda_{De}$ is
% given by Eq. \eqref{debye}, $v_{th}$ is the thermal velocity, and $g$
% is the plasma parameter.

In the absence of collisions, the plasma dynamics is given by the Vlasov-Maxwell
system, which is composed of the Vlasov movement equation
\begin{equation}
  \label{eq:vlasov}
  \frac{\partial f_{\alpha}({\bf r},{\bf v},t)}{\partial t}
  +{\bf v \cdot \nabla_r}f_{\alpha}({\bf r},{\bf v},t)
  +\frac{\bf F}{m_{\alpha}}
  {\bf \cdot \nabla_v}f_{\alpha}({\bf r},{\bf v},t)=0.
\end{equation}

In the general case, the acting force in Eqs. \eqref{boltz} and \eqref{eq:vlasov}
is given by the Lorentz force added by an external force, which, at kinetic scales,
is also of electromagnetic nature. Therefore, in \emph{cgs} units\footnote{The
  use of cgs units will be consistent through all this work.}, ${\bf F}$ has the
following form
\begin{equation}
  {\bf F}=\sum_{\alpha}q_{\alpha}\left[{\bf E}({\bf r},t)
    +\frac{1}{c}{\bf v \times B}({\bf r},t)\right]
  +{\bf F_{ext}},
  \label{florentz}
\end{equation}
where $q_{\alpha}$ is the charge of the particles of kind $\alpha$, and $c$ is
the speed of light. The quantities ${\bf E}({\bf r},t)$ and ${\bf B}({\bf r},t)$
are, respectively, the electric and magnetic fields, which are described by the
Maxwell equations
\begin{eqnarray}
  \label{max1}
  {\bf \nabla \cdot E}&=& 4\pi\rho\\
  \label{max2}
  {\bf \nabla \cdot B}&=& 0\\
  \label{max3}
    {\bf \nabla \times E}&=& -\frac{1}{c}
   \frac{\partial {\bf B}}{\partial t}\\
  \label{max4}
    {\bf \nabla \times B}&=& \frac{4\pi}{c}{\bf J}
    +\frac{1}{c}\frac{\partial{\bf E}}{\partial t}.
\end{eqnarray}

In a statistical description, the charge density $\rho$ and the current
density ${\bf J}$ depend on the velocity distribution function
\begin{eqnarray}
  \label{qdensity}
    \rho({\bf r},t)
  &=&\sum_{\alpha}q_{\alpha}\int_v
    f_{\alpha}({\bf r},{\bf v},t)d^3v\\
\label{jdensity}
  {\bf J}({\bf r},t)
  &=&\sum_{\alpha}q_{\alpha}\int_v{\bf v}
  f_{\alpha}({\bf r},{\bf v},t) d^3v.
\end{eqnarray}
% where the fields are described by the Maxwell equations
% \begin{eqnarray}
%   \label{max1}
%   {\bf \nabla \cdot E}&=& 4\pi\rho\\
%   \label{max2}
%   {\bf \nabla \cdot B}&=& 0\\
%   \label{max3}
%     {\bf \nabla \times E}&=& -\frac{1}{c}
%    \frac{\partial {\bf B}}{\partial t}\\
%   \label{max4}
%     {\bf \nabla \times B}&=& \frac{4\pi}{c}{\bf J}
%     +\frac{1}{c}\frac{\partial{\bf E}}{\partial t}.
% \end{eqnarray}
% here given in cgs units, as commonly used in the plasma physics literature.


% In a statistical description, the charge density $\rho$
% and the current density ${\bf J}$ depend on the velocity
% distribution function
% \begin{eqnarray}
%   \label{qdensity}
%     \rho({\bf r},t)
%   &=&\sum_{\alpha}q_{\alpha}\int_v
%     f_{\alpha}({\bf r},{\bf v},t)d^3v\\
% \label{jdensity}
%   {\bf J}({\bf r},t)
%   &=&\sum_{\alpha}q_{\alpha}\int_v{\bf v}
%   f_{\alpha}({\bf r},{\bf v},t) d^3v,
% \end{eqnarray}
% where $\alpha$ accounts for all charged species that constitute
% the plasma. In a fully ionized hydrogen plasma, $\alpha=i$ for
% ions and $\alpha=e$ for electrons.

Together, the expressions \eqref{eq:vlasov}, \eqref{max1},
\eqref{max2}, \eqref{max3}, \eqref{max4}, \eqref{qdensity}
and \eqref{jdensity} form a closed self-consistent set
of coupled nonlinear equations, which is responsible for the
description of a myriad of oscillatory processes that may
occur in plasmas and how they affect (and are affected by) the
velocity distribution function, in the absence of collisions.
The solution of such complicated system  will invariably depend
on some degree of approximation.

Considering the extensive variety of nonlinear processes that
might occur in a plasma, the appropriate way to deal with these
equations will depend on intrinsic characteristics of the plasma
and the nature of the phenomenon under consideration. For low
amplitude oscillations in a fully ionized plasma, one may employ
perturbation theory in order to solve the Vlasov equation
\cite{klimo}.

\section{Low amplitude electrostatic oscillations}
We are interested in the presence of high-frequency electrostatic
waves propagating in a homogeneous, fully ionized and unmagnetized
plasma, which is initially in equilibrium. Under such conditions,
we may consider ${\bf B=0}$. Thus, the Vlasov equation takes the
following form
\begin{equation}
  \label{vlasov-elec}
  \frac{\partial f_\alpha}{\partial t}+{\bf v \cdot \nabla}f_\alpha
+\frac{q_{\alpha}}{m_\alpha}\ {\bf E}\ {\bf\cdot \nabla_v}f_\alpha=0,
\end{equation}
where the electric field is given by
\begin{equation}
{\bf \nabla \cdot E}  = 4\pi\ \sum_\alpha q_\alpha 
\int_v  f_\alpha({\bf r},{\bf v},t) d^3v.
\label{leidegauss}
\end{equation}

Assuming low amplitude oscillations, we may approximate these equations
by introducing a small first order perturbation into the initial velocity
distribution function and, assuming ${\bf E}_0=0$, we introduce a small
first order electric field :
\begin{eqnarray}
  \label{fpert}
  f_{\alpha}({\bf r}, {\bf v},t)&=&f_{\alpha0}({\bf v})
	    +f_{\alpha1}({\bf r}, {\bf v},t),\\
  {\bf E}({\bf r},t)&=&{\bf E}_1({\bf r},t).
\label{Epert}
\end{eqnarray}

Until this point, the linear, quasilinear and nonlinear formulations
are identical. For a further evaluation we must decide if substitute
Eqs. \eqref{fpert} and \eqref{Epert} in Eq. \eqref{vlasov-elec} as they
are and then eliminate all nonlinear terms in the Vlasov equation. Or if
we take the average of Eqs. \eqref{fpert} and \eqref{Epert} and use the
averaged quantities to keep some degree of nonlinearity in the movement
equation. The subsequent deduction for each case was carefully detailed
and discussed in Chapters 2 and 3 of Ref. \cite{Tigik2015}. Therefore,
in order to maintain the focus on the objectives of the present work, we
% LFZ180216: Ao invés da frase acima, que tal a seguinte:
% "in order to maintain the focus on the objectives of the present work, we" ?
% Sabrina: OK
reproduce in the next chapter only a brief review of the deduction of
the weak turbulence theory equations. 



\chapter{Weak turbulence theory}
\label{cha:weak-turb}
The weak turbulence theory is what we may call a ``borderline approach''
because, despite the fact that it deals with some more intense
instabilities, we must assure that these fluctuations are still inside
the weak turbulence limit, i.e., that the turbulence is weak enough to be
treated by a perturbative method. Such boundary is defined by a comparison
between the averaged kinetic energy of the particles per volume unity
${\cal E}_{kin}=\bar n m<v^2>/2$, and the energy density ${\cal E}_f$, which
is associated with the fluctuations of the perturbative electric field.
Thus, by ``weak turbulence'' we mean
\begin{equation}
  \label{eq:weak}
  {\cal E}_f \ll {\cal E}_{kin},
\end{equation}
that is,
% LFZ180216: Não fica melhor com vírgula depois de "that is"?
% Sabrina: OK
the energy density of the fluctuations must be much smaller
% LFZ180216: Ou simplesmente "much smaller"?
% Sabrina: OK
than the kinetic energy density of the particles. This condition is
related to the slow wave-growing assumption \cite{david}, which is
central for the further development of nonlinear kinetic equations,
as will be discussed further.

The following discussion is a reduced version of the
comprehensive demonstration presented in Section 3.2 of
Ref. \cite{Tigik2015}. We also use the same notation and
logic sequence. Any deviation from \cite{Tigik2015} will
be explained in a footnote.


\section{Nonlinear kinetic equations}
\label{sec:kin-gen-eq}
Let us consider the propagation of electrostatic oscillations
in a homogeneous, unmagnetized and fully ionized plasma. Thus,
in the absence of collisions, the movement equation is given
by the electrostatic Vlasov equation
\begin{equation}
  \label{vlasov-nl}
  \frac{\partial f_a({\bf r}, {\bf v},t)}{\partial t}
+{\bf v \cdot \nabla}f_a({\bf r}, {\bf v},t)
+\frac{e_a}{m_a}{\bf E}({\bf r},t){\bf \cdot}
\frac{\partial f_a({\bf r},{\bf v},t)}{\partial \bf v}=0,
\end{equation}
where the electric field is given by
\begin{equation}
   {\bf \nabla \cdot E}({\bf r},t)
   =4\pi \hat n \sum_a e_a\int d^3 v f_a({\bf r}, {\bf v},t). 
\end{equation}
In the above equations, $f_a$ is the velocity distribution function,
where the subscript ``$a$'' is the kind of particle: $a=i$ for ions
and $a=e$ for electrons. For an overall neutral plasma, the average
number density of
% LFZ180216: Sugestão "the number density of"
% Sabrina: OK
the ions and the electrons is the same, $\hat n = \hat n_e = \hat n_i$.

Proceeding under the perturbative approach, we assume small
fluctuations and write the velocity distribution function
and the electric field as a sum of an equilibrium zero-order
term and a small first-order fluctuation
\begin{equation}
  \label{pert-nl}
  \begin{split}
      f_a({\bf r}, {\bf v},t)&=F_a({\bf v})+\delta f_a({\bf r}, {\bf v},t)\\
  {\bf E}({\bf r},t)&=\delta {\bf E}({\bf r},t),
  \end{split}
\end{equation}
where the zero-order electric field is assumed to be zero and
$F_a$ is the equilibrium velocity distribution function.

Substituting \eqref{pert-nl} in \eqref{vlasov-nl} we obtain
\begin{equation} 
  \frac{\partial F_a}{\partial t}+\frac{e_a}{m_a}\
  \delta {\bf E \cdot} \frac{\partial F_a}{\partial \bf v}
+\frac{\partial\ \delta f_a}{\partial t}+{\bf v \cdot \nabla} \delta f_a
  +\frac{e_a}{m_a}\ \delta {\bf E \cdot}
  \frac{\partial\ \delta f_a}{\partial \bf v}=0.
\label{turb_ave_vlasov}
\end{equation}

Taking the average\footnote{For more details see subsection 3.1.1
  of Ref. \cite{Tigik2015}.} of Eq. \eqref{turb_ave_vlasov}, we
obtain a kinetic equation for the time evolution of the equilibrium
velocity distribution function
\begin{equation}
\frac{\partial F_a}{\partial t}
=-\frac{e_a}{m_a}\left< \delta {\bf E \cdot}
\frac{\partial\ \delta f_a}{\partial \bf v}\right>.
\label{form_par}
\end{equation}

Subtracting \eqref{form_par} from \eqref{turb_ave_vlasov} we obtain
an equation for the fluctuations
\begin{equation}
\frac{\partial\ \delta f_a}{\partial t}+{\bf v \cdot \nabla} \delta f_a
+\frac{e_a}{m_a}\ \delta {\bf E \cdot} \frac{\partial F_a}{\partial \bf v}
+\frac{e_a}{m_a}\left[\delta {\bf E \cdot} \frac{\partial\ \delta f_a}{\partial \bf v}
-\left< \delta {\bf E \cdot}
\frac{\partial\ \delta f_a}{\partial \bf v}\right>\right]=0.
\label{flut}
\end{equation}

It is interesting to mention that at this stage, Eqs. \eqref{form_par}
and \eqref{flut} are precisely the same as those leading to the
quasilinear approximation.
% LFZ180216: Sugiro "are precisely the same as those leading to the
% quasilinear approximation."
% Sabrina: OK
Thus, at this point, one must choose if the quasilinear approach is
enough, or if a nonlinear formulation is necessary. In the first case
all terms proportional to $\delta {\bf E} \delta f_a$ are neglected in
the equation for the fluctuations\footnote{See subsection 3.1.2 of
  \cite{Tigik2015}.}. We shall keep these terms and proceed with the
nonlinear approach.

Now, we assume the that fluctuations may be decomposed in terms of
the Fourier-Laplace transformation over the fast time-scale of the
fluctuations, while amplitudes of the spectra vary in a slow
time-scale
\begin{equation}
\begin{split}
  \delta f_a({\bf r}, {\bf v},t)
  &=\int d^3k \int_L d \omega\
  \delta f^a_{{\bf k}, \omega}({\bf v}, t)e^{i({\bf k \cdot r}-\omega t)},\\
  \delta f^a_{{\bf k}, \omega}({\bf v}, t)
  &=\frac{1}{(2 \pi)^4} \int d^3 r \int_0^{\infty} dt\
  \delta f_a({\bf r}, {\bf v},t) e^{-i({\bf k \cdot r} - \omega t)},\\
  \delta {\bf E}({\bf r},t)
  &=\int d^3 k \int_L d \omega\ \delta {\bf E}_{{\bf k},\omega}(t)
  e^{i({\bf k \cdot r}-\omega t)},\\
  \delta {\bf E}_{{\bf k},\omega}(t)
  &=\frac{1}{(2 \pi)^4}\int d^3 r \int_0^{\infty} dt\
  \delta {\bf E}({\bf r},t) e^{-i({\bf k \cdot r} - \omega t)},
\end{split}
\end{equation}
% LFZ180216: Na última equação o denominador deve ser (2\pi)^4 ao invés de
% 2\pi^4, não?
% Sabrina: OK
where the integration path is taken along $L$, stretching from
$\omega = -\infty +i\sigma$ to $\omega=\infty+i\sigma$, in which
$\sigma >0$ and $\sigma \rightarrow 0$. We should emphasize that
we assume slow and adiabatic time-dependence for the spectral
amplitudes $\delta f^a_{{\bf k},\omega}({\bf v},t)$ and
$\delta {\bf E}(t)_{{\bf k}, \omega}$ in the above transformation.

The Fourier-Laplace transformation of nonlinear terms, where we
have the product of two functions, is given by the convolution
of these functions
\begin{equation}
\begin{split}
\frac{1}{(2\pi)^4}\int d^3r \int dt\ \delta f_a({\bf r}, {\bf v},t)\
\delta {\bf E}({\bf r},t)\ e^{-i({\bf k \cdot r}-\omega t)}
&=\int d^3 k' \int d \omega '\ \delta f^a_{{\bf k'}, \omega '}\
\delta {\bf E}_{{\bf k-k'},\omega-\omega '}\\
&=\int d^3 k' \int d \omega '\ \delta f^a_{{\bf k-k'}, \omega-\omega '}\
\delta {\bf E}_{{\bf k'},\omega '}.
\end{split}
\label{conv}
\end{equation}

Applying these transformations to the Vlasov-Maxwell system, we
obtain a set of hierarchic equations composed by the formal
particle kinetic equation \cite{Yoon00}
\begin{equation}
\frac{\partial F_a}{\partial t}=-\frac{e_a}{m_a}\frac{\partial}{\partial {\bf v}}
{\bf \cdot}\int d^3k \int d\omega \int d^3k'\int d\omega '\
\left<\delta{\bf E}_{{\bf k'},\omega '}\ \delta f^a_{{\bf k},\omega}\right>
e^{i({\bf k+k'}){\bf \cdot r}-i(\omega+\omega ')t};
\label{eq-cin-par}
\end{equation}
the equation of the fluctuating distribution evolution
\begin{equation}
\begin{split}
  \left(\omega-{\bf k \cdot v}
    +i\frac{\partial}{\partial t}\right)\delta f^a_{{\bf k},\omega}
  &=-i \frac{e_a}{m_a} \delta {\bf E}_{{\bf k},\omega}{\bf \cdot}
  \frac{\partial F_a}{\partial \bf v}
  -i\frac{e_a}{m_a}\frac{\partial}{\partial \bf v}
  {\bf \cdot} \int d^3k' \int d\omega '\\
  \times \bigr[\delta {\bf E}_{{\bf k'},\omega '}
  \delta f^a&_{{\bf k-k'},\omega-\omega '}
    -\left<\delta {\bf E}_{{\bf k'},\omega '}
    \delta f^a_{{\bf k-k'},\omega-\omega '}\right>\bigr];
\end{split}
\label{ev-par-per}
\end{equation}
and the differential form of the Gauss law for the electric field fluctuations
% and the Poisson equation for the electric field fluctuations
% LFZ180216: Aqui há um ponto que às vêzes me incomoda. O uso desse nome é
% frequente, o Peter costuma usá-lo, e "no embalo" eu uso muitas vêzes também.
% Mas a rigor a equação de Poisson é uma equação com o Laplaciano do potencial,
% não com a divergẽncia do campo (essa seria a lei de Gauss). É de se pensar
% se mantemos o costume, ou se vale a pena mudar ...
% (se mexer aqui tem que mudar mais adiante também)
% Sabrina: Pode ser assim?
% LFZ180220: Pode ser, com a mudança mais abaixo acompanhando.
\begin{equation}
   {\bf k \cdot}\delta{\bf E}_{{\bf k},\omega}=-4\pi \hat n i
   \sum_a e_a\int d^3v\ \delta f^a_{{\bf k}, \omega}.
\label{poisson-flut}
\end{equation}

The bracketed terms in Eqs. \eqref{eq-cin-par} and \eqref{ev-par-per}
represent the ensemble averages over the phases of the perturbation. One
may notice that the above set of equations is not closed, since the solution
for $\delta f^a_{{\bf k},\omega}$ requires knowing the two-body correlation
$\left<\delta f^a_{{\bf k},\omega}\delta f^a_{{\bf k'},\omega '}\right>$,
which depends on the tertiary correlation $\left<\delta f^a_{{\bf k},\omega}
  \delta f^a_{{\bf k'},\omega '}\delta f^a_{{\bf k''},\omega ''}\right>$, and
so on.

On the left-hand side of  Eq. \eqref{eq-cin-par} we have retained the
slow adiabatic derivative $i(\partial/\partial t)$. To deal with
that we employ two-step approximation \cite{Yoon00} and redefine
the angular frequency $\omega \rightarrow \omega+i \partial/\partial t$.
Then the equation for the
% LFZ180216: "the perturbed"?
% Sabrina: OK
perturbed distribution may be iteratively
solved up to third order in electric field.  The solution is then
inserted into Eq. \eqref{poisson-flut}
% the perturbed Poisson equation
% LFZ180216: Se o nome Poisson foi mudado anteriormente, deve ser mudado aqui
% também.
% Sabrina: coloquei o número da equação no lugar. Pode ser?
% LFZ180220: OK
and,  under the assumption that there are random phases associated with
the fluctuations, we take the appropriated ensembles averages. This
procedure results in the nonlinear spectral balance equation. At this
point, according to the two-time approximation, the slow time derivative
is reintroduced \cite{YZKS16}. We then obtain the nonlinear
% LFZ190822: Seria melhor evitar a repetição de "This procedure results".
% Quem sabe aqui se escreve algo como "We then obtain the nonlinear
% spectral balance equation\footnote{..." ?
% Sabrina: Sim! Eu nem tinha percebido isso
spectral balance equation\footnote{Equation \eqref{nlebe} is not in the
  same form that appears in Refs. \cite{Yoon00,Tigik2015}. The reason
  for this choice (see Eq. 2.59 of Ref. \cite{YZKS16}), will become
  clear in the next chapter.}
\begin{eqnarray}
  &&\frac{i}{2}\frac{\partial \epsilon({\bf k},\omega)}{\partial \omega}
  \frac{\partial \left<\delta E^2\right>_{{\bf k},\omega}}{\partial t}
  +\mbox{Re}\,\epsilon({\bf k},\omega)\left<\delta E^2 \right>_{{\bf k},\omega}
+i\mbox{Im}\,\epsilon({\bf k},\omega)\left<\delta E^2 \right>_{{\bf k},\omega}
  \nonumber\\
  &&\qquad \qquad \qquad -\frac{2}{\pi}\frac{1}{k^2\epsilon^*({\bf k},\omega)}
     \sum_ae_a^2\int d{\bf v}\,\delta(\omega-{\bf k\cdot v})F_a({\bf v})
  \nonumber\\
  &&=-2\int d{\bf k'} \int d\omega '\
  \biggl\{ \left[\chi^{(2)}({\bf k'},\omega '|{\bf k-k'},
    \omega-\omega ')\right]^2
  \biggl[\frac{\left<\delta E^2\right>_{{\bf k-k'},\omega-\omega '}}
       {\epsilon({\bf k'},\omega ')}
  \nonumber\\
  &&+\frac{\left<\delta E^2\right>_{{\bf k'},\omega '}}
  {\epsilon({\bf k-k'},\omega-\omega ')}\biggr]
     -\bar \chi^{(3)}({\bf k'},\omega|-{\bf k'},-\omega '|{\bf k},\omega)
      \left<\delta E^2\right>_{{\bf k'},\omega'}\biggr\}
  \left<\delta E^2\right>_{{\bf k},\omega}\nonumber\\
  &&+2\int d{\bf k'} \int d\omega '
  \frac{|\chi^{(2)}({\bf k'},\omega '|{\bf k-k'},\omega-\omega ')|^2}
{\epsilon^*({\bf k},\omega)}
\left<\delta E^2\right>_{{\bf k'},\omega '}
     \left<\delta E^2\right>_{{\bf k-k'},\omega-\omega '}
  \label{nlebe}\\
&&-\frac{4}{\pi}\int d{\bf k '} \int d\omega '\
\frac{1}{k^2|\epsilon({\bf k'},\omega ')|^2}
\biggl[ \frac{[\chi^{(2)}({\bf k'},\omega '|{\bf k-k'},\omega-\omega ')]^2}
{\epsilon({\bf k-k'},\omega-\omega ')}
    \left<\delta E^2 \right>_{{\bf k},\omega}
  \nonumber\\
&&-\frac{|\chi^{(2)}({\bf k'},\omega '|{\bf k-k'},\omega-\omega ')|^2}
{\epsilon^*({\bf k},\omega)}
\left<\delta E^2 \right>_{{\bf k-k'},\omega-\omega '}\biggr]
\sum_ae_a^2\int d{\bf v}\, \delta(\omega '-{\bf k \cdot v})F_a({\bf v})
  \nonumber\\
&&-\frac{4}{\pi}\int d{\bf k '} \int d\omega '\
\frac{1}{|{\bf k - k'}|^2|\epsilon({\bf k-k'},\omega-\omega ')^2}
\biggl[ \frac{[\chi^{(2)}({\bf k'},\omega '|{\bf k-k'},\omega-\omega ')]^2}
{\epsilon({\bf k'},\omega ')}\left<\delta E^2 \right>_{{\bf k},\omega}
  \nonumber\\
&&-\frac{|\chi^{(2)}({\bf k'},\omega '|{\bf k-k'},\omega-\omega ')|^2}
{\epsilon^*({\bf k},\omega)}
\left<\delta E^2 \right>_{{\bf k'},\omega '}\biggr]
\sum_ae_a^2\int d{\bf v}\, \delta(\omega-\omega '-{\bf (k-k')\cdot v})F_a({\bf v})
\nonumber
\end{eqnarray}
where on the left-hand side we have the expressions that correspond
to the linear equation and on the right-hand side are the nonlinear
expressions. The term
\begin{equation} 
\epsilon({\bf k},\omega)=1+\sum_a\chi_a({\bf k},\omega)
\end{equation}
is the linear dielectric response function.

In the above equations, $\sum_a \chi_a({\bf k},\omega)$
is the linear dielectric susceptibility,
$\sum_a \chi_a^{(2)}({\bf k}_1,\omega_1|{\bf k}_2,\omega_2)$ is the
second order nonlinear dielectric susceptibility and
$\sum_a \bar \chi_a^{(3)}({\bf k}_1,\omega_1|{\bf k}_2,\omega_2|{\bf k}_3,\omega_3)$
is the partial third-order nonlinear dielectric susceptibility.
For particles of species $a$, the dielectric susceptibilities
are respectively given by
\begin{equation}
\chi_a({\bf k},\omega)=-\frac{4\pi e_a\hat n_a}{k^2}
\int d^3v\ {\bf k \cdot g}_{{\bf k},\omega}F_a,
\end{equation}
\begin{equation}
\begin{split}
\chi_a^{(2)}({\bf k_1},\omega_1|{\bf k_2},\omega_2)
&=-\frac{1}{2}\frac{4\pi ie_a\hat n_a}{k_1k_2|{\bf k_1}+{\bf k_2}|}\\
\times\int d^3v\ {\bf g}_{{\bf k_1}+{\bf k_2}, \omega_1+\omega_2}
\bigl[{\bf k_1}&({\bf k_2}{\bf \cdot g}_{{\bf k_2},\omega_2})
 +{\bf k_2}({\bf k_1}{\bf \cdot g}_{{\bf k_1},\omega_1})\bigr]F_a,
\end{split}
\end{equation}
\begin{equation}
\begin{split}
  \bar \chi_a^{(3)}({\bf k_1},\omega_1|{\bf k_2},\omega_3|{\bf k_3},\omega_3)
  &=-\frac{1}{2}\frac{4\pi i e_a\hat n_a}{k_1k_2k_3|{\bf k_1}+{\bf k_2}+{\bf k_3}|}\\
  \times\int d^3v\
  ({\bf g}_{{\bf k_1}+{\bf k_2}+{\bf k_3}, \omega_1+\omega_2+\omega_3}{\bf \cdot k_1})
  {\bf g}&_{{\bf k_2}+{\bf k_3},\omega_2+\omega_3}{\bf \cdot}
  \bigl[{ \bf k_2}({\bf k_3}{\bf \cdot g}_{{\bf k_3},\omega_3})
  +{\bf k_3}({\bf k_2}{\bf \cdot g}_{{\bf k_2},\omega_2})\bigr]F_a,
\end{split}
\end{equation}
where ${\bf g}_{{\bf k},\omega}$ is a differential operator:
\begin{equation}
\label{op-dif}
  {\bf g}_{{\bf k},\omega}\equiv-\frac{e_a}{m_a}\
   \frac{1}{\omega-{\bf k \cdot v}+i0}\ \frac{\partial}{\partial \bf v}.
\end{equation}

Equation \eqref{nlebe} is a general expression. The usual
approach in plasma kinetic theory is to assume $|\mbox{Im}\
\epsilon({\bf k},\omega)|\ll|\mbox{Re}\ \epsilon({\bf k},\omega)|$.
Then, the imaginary part of \eqref{nlebe} leads to the wave kinetic
equation, while the real part leads to the wave dispersion equation,
whose solutions 
% LFZ190822: Não seria "whose solutions", como estava antes?
%Sabrina: ok
are the wave dispersion relations.

For the generalized particle kinetic equation is used a similar
procedure. In this work, however, we use only the quasilinear
approximation for the particle kinetic equation. Therefore, the
nonlinear particle equation will be omitted here\footnote{The
generalized form can be seen at Refs. \cite{Yoon00,Tigik2015}}.
Then, in the quasilinear approximation, the particle kinetic equation
is given by the following expression\footnote{\Cref{part}
  is not in the same form that appears in Refs. \cite{Yoon00,Tigik2015}.
  The reason for this choice, which is based in Ref. \cite{YZKS16},
  will become clear in the next chapter.} \cite{YZKS16}
\begin{equation}
  \begin{split}
  \frac{\partial F_a}{\partial t}&=\frac{\pi e_a^2}{m_a^2}
  \int d{\bf k}\, \int d\omega\,
  \left( \frac{\bf k}{k}\cdot \frac{\partial}{\partial {\bf v}}\right)
  \delta(\omega -{\bf k \cdot v})\\
  &\times\left[\mbox{Im} \frac{m_a \epsilon({\bf k},\omega)}
    {2\pi^3k \left|\epsilon ({\bf k},\omega) \right|^2 } F_a
  +\left<\delta E^2\right>_{{\bf k},\omega}
\left( \frac{\bf k}{k}\frac{\partial F_a}{\partial \bf v} \right)  \right].
  \end{split}
\label{part}
\end{equation}

\subsection{Wave kinetic equation for linear eigenmodes}
Following the standard weak turbulence theory procedure, we assume
slow wave amplification, which means that
% LFZ180216: "means that"?
% Sabrina: OK
the wave growing process
does not affect the plasma dynamics. Hence, the normal modes
of oscillation are determined only by the linear response of
the plasma, while the interactions among waves and particles
are, in general, described by the nonlinear wave and particle
kinetic equations. Thus, we have a theory that depicts nonlinear
interactions between linear eigenmodes. The dispersion relations
for linear normal modes are given by the solutions of the real
part of the first term in Eq. \eqref{nlebe}
\begin{equation}
  0=\mbox{Re}\ \epsilon({\bf k},\omega)\left<\delta E^2 \right>_{{\bf k},\omega}.
  \label{real-lin}
\end{equation}

Let us assume $\omega=\omega^{\alpha}_{{\bf k},\omega}$ is the solution
of the above expression. Once the dispersion relation has been calculated,
we may write the spectral wave amplitude as follows
\begin{equation}
\left<\delta E^2\right>_{{\bf k},\omega}
=\sum_{\alpha}\bigl[I^{+\alpha}_{\bf k} \delta(\omega-\omega_{\bf k}^{\alpha})
  +I^{-\alpha}_{\bf k} \delta(\omega+\omega_{\bf k}^{\alpha})\bigr],
\label{wave-dens}
\end{equation}
where the $\pm$ signs represent the propagation direction
of their respective normal modes, denoted by $\alpha$. The
eigenmodes will be given by the oscillation modes that satisfy
the following condition
\begin{equation}
\label{eigenmode}
\epsilon({\bf k,\pm\, \omega_{{\bf k},\omega}^\alpha}) \approx 0.
\end{equation}

In an unmagnetized plasma, there are two possible electrostatic
eigenmodes: the  Langmuir $(\alpha= L)$ and ion-sound $(\alpha=S)$
waves. Therefore, after careful considerations and extensive
algebraic manipulations, the following kinetic equation for these
waves is obtained\footnote{\Cref{wave-lin} is not in the same form
  that appears in Refs. \cite{Yoon00,Tigik2015}. The reason for this
  choice, which is based in Ref. \cite{YZKS16}, will become clear in
  the next chapter.}\cite{Yoon00,Yoon05a}
\begin{equation}
\begin{split}
  \frac{\partial I^{\alpha}_{\bf k}}{\partial t}
    =&-\frac{2 \mbox{Im}\ \epsilon({\bf k}, \sigma \omega_{\bf k}^{\alpha})}
    {\epsilon '({\bf k},\sigma \omega_{\bf k}^{\alpha})}I^{\alpha}_{\bf k}
    +\sum_{a=e,i}\frac{4e^2}{k^2[\epsilon '({\bf k},\sigma\omega_{\bf k}^\alpha)]^2}
    \int d{\bf v}\, \delta(\sigma\omega_{\bf k}^\alpha-{\bf k \cdot v})
    f_a({\bf v})\\
    -&\sum_{\alpha,\beta}\ \sum_{\sigma '=\pm 1}
    \int d{\bf k '}\ A_{\alpha,\beta}({\bf k,k'})
    I_{\bf k'}^{\beta}\ I_{\bf k}^{\alpha}
    -\sum_{a=e,i}\frac{16 e_a^2}{\epsilon '({\bf k},\sigma\omega_{\bf k}^\alpha)}
    \sum_{\sigma '=\pm 1}\sum_{\beta=L,S}\\
    \times&\int d{\bf v}\int d{\bf k '}\,
    \frac{|\chi^{(2)}({\bf k'},\sigma '\omega_{\bf k'}^{\beta}|{\bf k-k'},
      \sigma \omega_{\bf k}^\alpha-\sigma '\omega_{\bf k'}^{\beta})|^2}
    {|{\bf k-k'}|^2|\epsilon({\bf k-k'},\sigma \omega_{\bf k}^\alpha
      -\sigma '\omega_{\bf k'}^{\beta})|^2}\\
    \times& \left[ \frac{I_{\bf k}^{\sigma \alpha}}
      {\epsilon({\bf k '},\sigma '\omega_{\bf k'}^\beta)}
    -\frac{I_{\bf k'}^{\sigma '\beta}}
    {\epsilon '({\bf k },\sigma \omega_{\bf k}^\alpha)}\right]
   \delta[\sigma \omega_{\bf k}^\alpha-\sigma '\omega_{\bf k'}^\beta
  -({\bf k-k'}){\bf \cdot v}] f_a({\bf v})\\
    -&4\pi \sum_{\alpha,\beta,\gamma}\ \sum_{\sigma ''= \pm 1}\int d{\bf k '}\
    \frac{|\chi^{(2)}({\bf k'},\sigma '\omega_{\bf k'}^\beta|{\bf k-k'},
      \sigma ''\omega_{\bf k-k'}^\gamma)|^2}
       {\epsilon '({\bf k},\sigma \omega_{\bf k}^{\alpha})}\\
       \times& \biggl(\frac{I_{\bf k-k'}^{\gamma}\ I_{\bf k}^{\alpha}}
       {\epsilon '({\bf k'},\sigma ' \omega_{\bf k'}^{\beta})}
       +\frac{I_{\bf k'}^{\beta}\ I_{\bf k}^{\alpha}}
       {\epsilon '({\bf k-k'},\sigma ''\omega_{\bf k-k'}^{\gamma})}
       -\frac{I_{k'}^{\beta}\ I_{\bf k-k'}^{\gamma}}
       {\epsilon '({\bf k},\sigma \omega_{\bf k}^{\alpha})}\biggr)
       \delta\bigl(\sigma \omega_{\bf k}^{\alpha}-\sigma ' \omega_{\bf k'}^{\beta}
       -\sigma '' \omega_{\bf k-k'}^{\gamma}\bigr),
       \label{wave-lin}
\end{split}
\end{equation}
where the coefficient $A_{\alpha,\beta}({\bf k,k'})$ is given by
\begin{equation}
\begin{split}
  &A_{\alpha,\beta}({\bf k,k'})
  =\frac{4}{\epsilon '({\bf k},\sigma \omega_{\bf k}^{\alpha})}
  \mbox{Im}\biggl(2\bigl[\chi^{(2)}({\bf k'},\sigma ' \omega_{\bf k'}^{\beta}|
    {\bf k-k'},\sigma \omega_{k}^{\alpha}-\sigma ' \omega_{\bf k'}^{\beta})\bigr]^2\\
    \times&{\cal P}\frac{1}{\epsilon({\bf k-k'},\sigma \omega_{\bf k}^{\alpha}
        -\sigma ' \omega_{\bf k'}^{\beta})}
       -\bar \chi^{(3)}({\bf k'},\sigma ' \omega_{\bf k'}^{\beta}|-{\bf k'},
       -\sigma ' \omega_{\bf k'}^{\beta}|{\bf k},
       \sigma \omega{\bf k}^{\alpha})\biggr),
       \label{coef-esp}
\end{split}
\end{equation}
and $\mathcal{P}$ is the principal value of the integral where
the coefficient belongs. In both expressions the following
short-hand notation has been used
\begin{equation}
\epsilon '({\bf k},\sigma \omega_{\bf k}^{\alpha})
=\frac{\partial \mbox{Re}\ \epsilon ({\bf k},\sigma \omega_{\bf k}^{\alpha})}
{\partial \sigma \omega_{\bf k}^{\alpha}}.
\end{equation}

On the right hand side of \eqref{wave-lin}, in the first two terms
we have the quasilinear wave-particle interaction, responsible for
the induced and spontaneous emission processes. The third and fourth
terms are nonlinear wave-particle interactions accounting for the
induced and spontaneous scattering processes. The last term is the
nonlinear wave-wave interaction describing the induced and spontaneous
three wave-decay processes. 


\section{Weak turbulence equations for Langmuir and ion-sound waves}
The further evaluation of Eq. \eqref{wave-lin} demands a careful
analysis of the resonance conditions for each process and will not
be addressed here\footnote{A comprehensive description of this
  process can be seen in subsection 3.2.3 and section 3.3 of
  \cite{Tigik2015}.}. In the following equations we show the complete
wave kinetic equation for both $(L)$ and $(S)$ waves, in which the term
describing the effects of spontaneous emission is already included.

For $(L)$ waves we have:
\begin{displaymath}
\frac{\partial}{\partial t}\frac{I_{\bf k}^{\sigma L}}{\mu_{\bf k}^L}
=\mu_{\bf k}^L\,\frac{\omega_{pe}^2}{k^2}\int d^3v\;
\delta(\sigma\omega_{\bf k}^L-{\bf k}\cdot{\bf v})
\biggl(\hat{n}\,e^2\,F_e({\bf v})+\pi\,(\sigma\omega_{\bf k}^L)\,
\;{\bf k}\cdot\frac{\partial F_e({\bf v})}{\partial{\bf v}}
\,\frac{I_{\bf k}^{\sigma L}}{\mu_{\bf k}^L}\biggr)
\end{displaymath}
\begin{displaymath}
-\,\pi\sigma\mu_{\bf k}^L\,\omega_{\bf k}^L
\,\frac{e^2}{2T_e^2}\sum_{\sigma ',\sigma ''}\int d^3 k'\;
\frac{\mu_{{\bf k}'}^L\,\mu_{{\bf k}-{\bf k}'}^S
\,({\bf k}\cdot{\bf k}')^2}{k^2\,k'^2\,|{\bf k}-{\bf k}'|^2}
\biggl(\sigma '\omega_{{\bf k}'}^L\,
\frac{I_{{\bf k}-{\bf k}'}^{\sigma ''S}}{\mu_{{\bf k}-{\bf k}'}^S}
\frac{I_{\bf k}^{\sigma L}}{\mu_{\bf k}^L}
\end{displaymath}
\begin{equation}
+\,\sigma ''\omega_{{\bf k}-{\bf k}'}^L
\,\frac{I_{{\bf k}'}^{\sigma 'L}}{\mu_{{\bf k}'}^L}
\frac{I_{\bf k}^{\sigma L}}{\mu_{\bf k}^L}
-\sigma\omega_{\bf k}^L\,\frac{I_{{\bf k}'}^{\sigma 'L}}
{\mu_{{\bf k}'}^L}\frac{I_{{\bf k}-{\bf k}'}^{\sigma ''S}}
{\mu_{{\bf k}-{\bf k}'}^S}\biggr)
\;\delta(\sigma\omega_{\bf k}^L-\sigma '\omega_{{\bf k}'}^L
-\sigma ''\omega_{{\bf k}-{\bf k}'}^S)
\label{L,1}
\end{equation}
\begin{displaymath}
+\,\sigma\omega_{\bf k}^L\,\frac{e^2}{m_e^2\,\omega_{pe}^2}
\sum_{\sigma '}\int d^3 k'\int d^3v\;
\frac{\mu_{\bf k}^L\,\mu_{{\bf k}'}^L\,
({\bf k}\cdot{\bf k}')^2}{k^2\,k'^2}\;\delta[\sigma\omega_{\bf k}^L
-\sigma '\omega_{{\bf k}'}^L-({\bf k}-{\bf k}')\cdot{\bf v}]
\end{displaymath}
\begin{displaymath}
\times\,\biggl[\frac{\hat{n}\,e^2}{\omega_{pe}^2}
\,\biggl(\sigma\omega_{\bf k}^L\,\frac{I_{{\bf k}'}^{\sigma 'L}}
{\mu_{{\bf k}'}^L}
-\sigma '\omega_{{\bf k}'}^L\,
\frac{I_{\bf k}^{\sigma L}}{\mu_{\bf k}^L}
\biggr)\,[F_e({\bf v})+F_i({\bf v})]
+\pi\,\frac{m_e}{m_i}\,\frac{I_{{\bf k}'}^{\sigma 'L}}{\mu_{{\bf k}'}^L}
\frac{I_{\bf k}^{\sigma L}}{\mu_{\bf k}^L}\;({\bf k}-{\bf k}')
\cdot\frac{\partial F_i({\bf v})}{\partial{\bf v}}\biggr].
\end{displaymath}

In the first line on the right-hand side of \eqref{L,1}, one may recognize
the quasilinear resonance condition $\delta(\sigma\omega_{\bf k}^L
-{\bf k}\cdot{\bf v})$, which is responsible for the spontaneous
emission process and the induced emission process, respectively.
In the second and third lines we have the three wave-decay process,
which can be recognized by the wave-wave resonance condition 
$\delta(\sigma\omega_{\bf k}^L-\sigma '\omega_{{\bf k}'}^L
-\sigma ''\omega_{{\bf k}-{\bf k}'}^S)$ at the end of the third line.
Inside the parenthesis, the first two terms depicts the induced
wave-decay process, and the third term describes the spontaneous
wave-decay process. In the last two lines we have the equations
for the induced and spontaneous scattering processes. At the end
of the fourth line is the nonlinear wave-particle resonance
condition $\delta[\sigma\omega_{\bf k}^L-\sigma '\omega_{{\bf k}'}^L
-({\bf k}-{\bf k}')\cdot{\bf v}]$  and, in the last line, are the
terms of spontaneous and induced scattering, respectively.

For $(S)$ waves we have the same effects, in the same order:
\begin{displaymath}
\frac{\partial}{\partial t}\frac{I_{\bf k}^{\sigma S}}{\mu_{\bf k}^S}
=\mu_{\bf k}^S\,\frac{\omega_{pe}^2}{k^2}
\int d^3v\;\delta(\sigma\omega_{\bf k}^S-{\bf k}\cdot{\bf v})
\biggl[\hat{n}\,e^2\,[F_e({\bf v})+F_i({\bf v})]
\end{displaymath}
\begin{displaymath}
+\,\pi\,(\sigma\omega_{\bf k}^L)\,\biggl({\bf k}\cdot
\frac{\partial F_e({\bf v})}{\partial{\bf v}}
+\frac{m_e}{m_i}\;{\bf k}\cdot
\frac{\partial F_i({\bf v})}{\partial{\bf v}}\biggr)
\,\frac{I_{\bf k}^{\sigma S}}{\mu_{\bf k}^S}\biggr]
\end{displaymath}
\begin{displaymath}
-\,\pi\sigma\omega_{\bf k}^L\,\frac{e^2}{4T_e^2}
\sum_{\sigma ',\sigma ''}\int d^3k'\;
\frac{\mu_{\bf k}^S\,\mu_{{\bf k}'}^L\,\mu_{{\bf k}-{\bf k}'}^L
\,[{\bf k}'\cdot({\bf k}-{\bf k}')]^2}{k^2\,k'^2\,|{\bf k}-{\bf k}'|^2}
\biggl(\sigma '\omega_{{\bf k}'}^L\,
\frac{I_{{\bf k}-{\bf k}'}^{\sigma ''L}}{\mu_{{\bf k}-{\bf k}'}^L}
\frac{I_{\bf k}^{\sigma S}}{\mu_{\bf k}^S}
\end{displaymath}
\begin{equation}
+\,\sigma ''\omega_{{\bf k}-{\bf k}'}^L
\,\frac{I_{{\bf k}'}^{\sigma 'L}}{\mu_{{\bf k}'}^L}
\frac{I_{\bf k}^{\sigma S}}{\mu_{\bf k}^S}
-\sigma\omega_{\bf k}^L\,
\frac{I_{{\bf k}'}^{\sigma 'L}}{\mu_{{\bf k}'}^L}
\frac{I_{{\bf k}-{\bf k}'}^{\sigma ''L}}
{\mu_{{\bf k}-{\bf k}'}^L}\biggr)
\;\delta(\sigma\omega_{\bf k}^S-\sigma '\omega_{{\bf k}'}^L
-\sigma ''\omega_{{\bf k}-{\bf k}'}^L)
\label{S,1}
\end{equation}
\begin{displaymath}
+\,\sigma\omega_{\bf k}^L\,\frac{e^2}{m_e^2\,\omega_{pe}^2}
\sum_{\sigma '}\int d^3k'\int d^3v\;
\frac{\mu_{\bf k}^S\,\mu_{{\bf k}'}^S\,
({\bf k}\cdot{\bf k}')^2}{k^4\,k'^4\,\lambda_{De}^4}
\;\delta[\sigma\omega_{\bf k}^S-\sigma '\omega_{{\bf k}'}^S
-({\bf k}-{\bf k}')\cdot{\bf v}]
\end{displaymath}
\begin{displaymath}
\times\,\biggl[\frac{\hat{n}\,e^2}{\omega_{pe}^2}
\,W_{{\bf k},{\bf k}'}\,\biggl(\sigma\omega_{\bf k}^L\,
\frac{I_{{\bf k}'}^{\sigma 'S}}{\mu_{{\bf k}'}^S}
-\sigma '\omega_{{\bf k}'}^L\,
\frac{I_{\bf k}^{\sigma S}}{\mu_{\bf k}^S}\biggr)
\;[F_e({\bf v})+F_i({\bf v})]
\end{displaymath}
\begin{displaymath}
+\,\pi\,\frac{m_e}{m_i}\biggl(W_{{\bf k},{\bf k}'}
+\sigma\,\sigma '\,\frac{k'}{k}\biggr)
\,\frac{I_{{\bf k}'}^{\sigma 'S}}{\mu_{{\bf k}'}^S}
\frac{I_{\bf k}^{\sigma S}}{\mu_{\bf k}^S}\;({\bf k}-{\bf k}')
\cdot\frac{\partial F_i({\bf v})}{\partial{\bf v}}\biggr].
\end{displaymath}
In the scattering term we have
\begin{equation}
  \label{W}
W_{{\bf k},{\bf k}'}=\biggl(1+\frac{1}{\xi^2}\biggr)^2
\frac{1}{|{\bf k}-{\bf k}'|^4\,\lambda_{De}^4\,
|\epsilon_\parallel({\bf k}-{\bf k}',
\sigma\omega_{\bf k}^S-\sigma '\omega_{{\bf k}'}^S)|^2},
\end{equation}
where
\begin{equation}
  \label{eps-par}
\epsilon_\parallel({\bf k}-{\bf k}',
\sigma\omega_{\bf k}^S-\sigma '\omega_{{\bf k}'}^S)
=1+\frac{2\,({\bf k}\cdot{\bf k}'-\sigma\,\sigma\,k\,k')}
{|{\bf k}-{\bf k}'|^2\,(k-\sigma\,\sigma '\,k')^2\,\lambda_{De}^2}
\end{equation}
\begin{displaymath}
+\,i\,\biggl(\frac{\pi}{2}\frac{m_e}{m_i}\biggr)^{1/2}
\biggl[\exp\biggl(-\frac{m_e}{m_i}\frac{\xi}{2}\biggr)
+\biggl(\frac{m_i}{m_e}\frac{T_e^3}{T_i^3}\biggr)^{1/2}
\exp\biggl(-\frac{T_e}{T_i}\frac{\xi}{2}\biggr)\biggr],
\end{displaymath}
with
\begin{equation}
  \label{xi}
\xi=\frac{(\sigma\,k-\sigma ' k')^2}{|{\bf k}-{\bf k}'|^2}.
\end{equation}

Though Eq. \eqref{S,1} depicts the full wave kinetic equation for
ion-sound waves, the scattering term has always been neglected and
will not be considered in this work as well. The reason for that is
that the scattering of ion-sound waves caused by other ion-sound
waves evolves in a very slow time-scale. Indeed, we are interested
in describing collisional processes, which act in a long evolution
time. However, one must take one step at a time, and we do not
discard including ion-sound scattering in future analysis.  


% LFZ180216: Nas aplicações esse termo de espalhamento tem sido sempre
% desprezado. Há algum comentário a esse respeito em algum lugar? Suponho que
% haja nos trabalhos anexados, mas talvez seja bom dizer algo a respeito
% aqui, não?
% Sabrina: Acrescentei. Que tu achas?
% LFZ180220: Acho que ficou bom. Só mudei uma ou duas palavras. Achei que
% as duas frase finais podem ser unidas. OK?
% Sabrina: OK



The particle kinetic equation is given by
\begin{equation}
\begin{split}
\frac{\partial F_a({\bf v})}{\partial t}&=\frac{\pi e_a^2}{m_a^2}
\sum_\sigma\sum_{\alpha=L,S}\int d^3k\ \biggl(\frac{\bf k}{k}
\cdot\frac{\partial}{\partial{\bf v}}\biggr)\ \mu_{\bf k}^\alpha
\delta(\sigma\omega_{\bf k}^\alpha-{\bf k}\cdot{\bf v})\\
&\times\biggl(\frac{m_a}{4\pi^2}\frac{\sigma\omega_{\bf k}^L}{k}
F_a({\bf v})+\frac{I_{\bf k}^{\sigma\alpha}}{\mu_{\bf k}^\alpha}
\frac{\bf k}{k}\cdot\frac{\partial F_a({\bf v})}{\partial{\bf v}}\biggr),
\end{split}
\label{Fa,1}
\end{equation}
where $a=i$ for ions and $a=e$ for electrons, and $\alpha=L,S$.

The following approximations and definitions were used to write
the above equations
\begin{displaymath}
\frac{1}{\epsilon '_\parallel({\bf k},\sigma\omega_{\bf k}^L)}
=\frac{\sigma\mu_{\bf k}^L\,\omega_{\bf k}^L}{2},\qquad
\frac{1}{\epsilon '_\parallel({\bf k},\sigma\omega_{\bf k}^S)}
=\frac{\sigma\mu_{\bf k}^S\,\omega_{\bf k}^L}{2},\qquad
\end{displaymath}
\begin{displaymath}
\mu_{\bf k}^L=1,\qquad
\mu_{\bf k}^S=|k|^3\lambda_{De}^3\biggl(\frac{m_e}{m_i}\biggr)^{1/2}
\biggl(1+\frac{3T_i}{T_e}\biggr)^{1/2}.
\end{displaymath}
% LFZ190822: Ao final da expressão acima havia um ponto "." e corrigistes
% colocando uma vírgula. Mas acho que era para ser ponto mesmo, pois a frase
% acaba ali.
% Sabrina: ok
\chapter{Weak turbulence for collisional plasmas}
\label{noneigenmode}
In \Cref{cha:weak-turb} we made a brief review of the standard weak
turbulence theory procedure. Such procedure takes into account only
the contributions from the linear eigenmodes. The basic premise of this
conventional approach claims that only frequencies $\omega=\omega_{\bf k}$
that satisfy the dispersion relations are considered significant, even if
the general electrostatic fluctuations $\left<\delta E^2\right>_{{\bf k},\omega}$
are characterized by all ${\bf k}$ and $\omega$. Therefore, the fluctuations
with 
% LFZ190822: Acho que aqui seria "with", não ''which" (i.e., "the fluctuations 
% with $\omega\ne\omega_{\bf k}$, which characterize the ...")
% Sabrina: de fato.
$\omega \ne \omega_{\bf k}$, which characterize the noneigenmodes,
are ignored. In Ref. \cite{YZKS16} the authors present a generalization
of this customary approach, where, besides the traditional inclusion of
the normal modes, it is also taken into account the contribution from the
noneigenmode fluctuations. The result is a generalized expression for the
spectral energy density of the waves, with some remarkable consequences
for the wave and particle kinetic equations.

The generalized spectral energy density, which contains the noneigenmode
contribution, is the central factor of this new kinetic turbulence theory
for collisional plasmas. When it is substituted in the nonlinear terms of
the nonlinear spectral balance equation, three new terms appear in the wave
kinetic equation. The first term is an extension for the spontaneous
scattering equation. The second term is a new, rigorous expression for the
collisional damping rate of plasma waves. The third term depicts a
hitherto unknown process of electrostatic radiation emission, in the eigenmode
frequency range, due to particle scattering. This underlying electrostatic
form of braking radiation was
% LFZ190822: Esse "then" me parece desnecessário.
% Sabrina: ok
called \emph{electrostatic bremsstrahlung}
\cite{YZKS16}. For the particle kinetic equation, the application of this
generalized expression is responsible for the arising of the Balescu-Lenard
collision integral, without any \emph{ad hoc} addition.

In this chapter, we provide a basic outline of the elaborated demonstration
imparted by the authors of \cite{YZKS16}. We start by showing how the
noneigenmodes contribution is included in the spectral energy density of the
waves. Then we show that the inclusion of the noneigenmodes does not affect
the quasilinear wave kinetic equation. In the next two subsections, we
exhibit, without entering in details, the effects of the new contribution
to the particle kinetic equation and the nonlinear particle kinetic equation.



\section{Inclusion of noneigenmodes}
\label{sec:inc-non}
Let us start considering the linear part of the spectral balance equation,
which corresponds to the left-hand side of Eq. \eqref{nlebe}
\begin{equation}
  \label{lin}
  \begin{split}
  \frac{1}{2}\frac{\partial \,\mbox{Re}\,\epsilon({\bf k},\omega)}{\partial \omega}
  \frac{\partial \left< \delta E^2 \right>_{{\bf k},\omega}}{\partial t}
  +&\,\mbox{Re}\,\epsilon({\bf k},\omega)
  \left< \delta E^2 \right>_{{\bf k},\omega}
  +i\, \mbox{Im}\,\epsilon({\bf k},\omega)
  \left< \delta E^2 \right>_{{\bf k},\omega}\\
  =&\frac{2}{\pi}\frac{1}{k^2\epsilon^{*}({\bf k},\omega)}
  \sum_a e^2_a\, \int d{\bf v}\,\delta(\omega-{\bf k \cdot v})f_a({\bf v}),
\end{split}
\end{equation}
where the dielectric constant is given by
\begin{equation}
  \label{dielec}
  \epsilon({\bf k},\omega)=\sum_a \frac{4\pi e^2_a}{m_ak^2}\int d{\bf v}\,
  \frac{{\bf k \cdot}\partial f_a/\partial{\bf v}}{\omega-{\bf k \cdot v}+i0}.
\end{equation}

Then, taking the real part of \eqref{lin}, we have
\begin{equation}
  \label{real}
  \mbox{Re}\, \epsilon({\bf k},\omega)\left[
  \left\langle \delta E^2 \right\rangle_{{\bf k},\omega}
  -\frac{2}{\pi}\frac{1}{k^2|\epsilon({\bf k},\omega)|^2}
  \sum_a e_a^2 \int d{\bf v}\, \delta(\omega-{\bf k \cdot v})
  f_a({\bf v})\right]=0.
\end{equation}

In the above equation we have two possibilities for the wave-number
frequency space. If one is interested in the region of $({\bf k},\omega)$
in which $|\epsilon({\bf k},\omega)|^2\ne 0$, then the left- and right-hand
sides of \eqref{real} may be balanced by writing
\begin{equation}
  \label{balance}
  \left\langle \delta E^2 \right\rangle_{{\bf k},\omega}
  =\left\langle \delta E^2 \right\rangle_{{\bf k},\omega}^0,
\end{equation}
where
\begin{equation}
  \label{field-fluc}
  \left\langle \delta E^2 \right\rangle_{{\bf k},\omega}^0
  \equiv \frac{2}{\pi}\frac{1}{k^2|\epsilon({\bf k},\omega)|^2}
  \sum_a e_a^2 \int d{\bf v}\, \delta(\omega-{\bf k \cdot v})
  f_a({\bf v}).
\end{equation}

It is clear that the solution of \eqref{balance} depends on the denominator
$|\epsilon({\bf k},\omega)|^2$ not being zero. For eigenmodes, however,
$\epsilon({\bf k},\omega_{\bf k}^\alpha)=0$, meaning the denominator remains
nonzero only if $\omega \ne \omega_{\bf k}^\alpha$, i.e., if $\omega$ does
not satisfy the dispersion relation. Hence, the electric field fluctuation
$\left< \delta E^2 \right>_{{\bf k},\omega}^0$ must represent noneigenmodes
contribution. As exposed in \Cref{sec:kin-gen-eq} the standard theory is
concerned only with the normal oscillation modes, which are the solutions
of the dispersion relations, calculated in the vicinity of the zeros of
$\mbox{Re}\,\epsilon({\bf k},\omega)$.


Therefore, we have two distinct situations. One is the situation in which
collective modes are not important, and the electric field energy can be
expressed by Eq. \eqref{balance} with the electric field fluctuation
$\left< \delta E^2 \right>_{{\bf k},\omega}^0$ given by \eqref{field-fluc}.
The other is standard situation, in which collective modes are important,
and the wave electric field is determined by
\begin{equation}
  \label{collective}
  \begin{split}
  \left< \delta E^2 \right>_{{\bf k},\omega}
  -\left< \delta E^2 \right>_{{\bf k},\omega}^0
  &=\sum_{\sigma=\pm 1}\sum_{\alpha=L,S}
    I_{\bf k}^{\sigma \alpha}\delta(\omega-\sigma \omega_{\bf k}^\alpha)\\
  \mbox{Re}\, \epsilon({\bf k},\sigma\omega_{\bf k}^\alpha)&=0.
\end{split}
\end{equation}

The theory that we are 
% LFZ180216: "The theory that we are ..."?
% Sabrina: OK
analyzing is a generalization of the standard weak
turbulence theory, in which both the eigenmode and noneigenmode are
included by defining the following quantity
\begin{equation}
  \label{Psi}
  \Psi_{{\bf k},\omega}\equiv\left< \delta E^2 \right>_{{\bf k},\omega}
  -\left< \delta E^2 \right>_{{\bf k},\omega}^0.
\end{equation}

Thus, the function $\Psi_{{\bf k},\omega}$ becomes the total eigenfunction
of Eq. \eqref{real}:
\begin{equation}
  \label{total}
  \mbox{Re}\, \epsilon({\bf k},\omega)\Psi_{{\bf k}, \omega}=0,
\end{equation}
where $\Psi_{{\bf k},\omega}$ can be expressed as
\begin{equation}
  \label{Psi-eigen}
  \Psi_{{\bf k},\omega}=\sum_{\sigma=\pm 1}\sum_{\alpha=L,S}
  I_{\bf k}^{\sigma \alpha}\delta(\omega-\sigma \omega_{\bf k}^\alpha),
\end{equation}
with the eigenvalue $\sigma \omega_{\bf k}^\alpha$ satisfying
\begin{equation}
  \mbox{Re}\, \epsilon({\bf k},\sigma\omega_{\bf k}^\alpha)=0.
\end{equation}

The procedure described above encompasses the two possibilities under
analysis in this work by generalizing Eq.~\eqref{wave-dens} to take into
account the noneigenmode contribution, given by Eq.~\eqref{field-fluc},
along with Eq.~\eqref{Psi-eigen} for the collective modes that satisfy
the eigenmode condition, resulting in the following expression
\begin{equation}
  \label{gen-dens}
  \left< \delta E^2 \right>_{{\bf k},\omega}
  =\left< \delta E^2 \right>_{{\bf k},\omega}^0
  +\sum_{\sigma=\pm 1}\sum_{\alpha=L,S}
  I_{\bf k}^{\sigma \alpha}\delta(\omega-\sigma \omega_{\bf k}^\alpha).
\end{equation}

\subsection{Absence of noneigenmode contribution to quasilinear wave
  kinetic equation}
Let us start by including the noneigenmode term in the quasilinear
wave kinetic equation. Substituting Eq. \eqref{gen-dens} in the
imaginary part of Eq. \eqref{lin}, we have
\begin{equation}
  \begin{split}
 &\frac{\partial\, \mbox{Re}\,\epsilon({\bf k},\omega)}{\partial \omega}
  \frac{\partial \left< \delta E^2 \right>_{{\bf k},\omega}^0}{\partial t}
  + 2\, \mbox{Im}\,\epsilon({\bf k},\omega)
  \left< \delta E^2 \right>_{{\bf k},\omega}^0+\sum_{\sigma=\pm 1}\sum_\alpha\biggl[
    \frac{\partial \mbox{Re}\, \epsilon({\bf k},\sigma\omega_{\bf k}^\alpha)}
    {\partial(\sigma\omega_{\bf k}^\alpha)}
    \frac{\partial I_{\bf k}^{\sigma \alpha}}{\partial t}\\
  &+2\,\mbox{Im}\,\epsilon({\bf k},\sigma\omega_{\bf k}^\alpha)
    I_{\bf k}^{\sigma \alpha}\biggr]\delta(\omega-\sigma\omega_{\bf k}^\alpha)
=\mbox{Im}\,\frac{4}{\pi k^2\epsilon^{*}({\bf k},\omega)}
  \sum_a e^2_a\, \int d{\bf v}\,\delta(\omega-{\bf k \cdot v})f_a({\bf v}).
\end{split}
\end{equation}

Making use of
\begin{equation}
  \label{principal}
  \begin{split}
    \frac{1}{\epsilon({\bf k},\omega)}
    &=\mathcal{P}  \frac{1}{\epsilon({\bf k},\omega)}
    -\sum_{\sigma=\pm 1}\sum_{\alpha=L,S}
    \frac{i\pi \delta(\omega-\sigma\omega_{\bf k}^\alpha)}{\epsilon '({\bf k},\omega)}\\
    \frac{1}{\epsilon^*({\bf k},\omega)}
    &=\mathcal{P}\frac{1}{\epsilon^*({\bf k},\omega)}
    +\sum_{\sigma=\pm 1}\sum_{\alpha=L,S}
    \frac{i\pi \delta(\omega-\sigma\omega_{\bf k}^\alpha)}{\epsilon '({\bf k},\omega)},
  \end{split}
\end{equation}
where
\begin{displaymath}
  \epsilon '({\bf k},\omega)=\partial\,
  \mbox{Re}\, \epsilon({\bf k},\omega)/\partial \omega,
\end{displaymath}
we obtain
\begin{equation}
  \label{noneigenmode-ql}
  \begin{split}
 \epsilon '({\bf k},\omega)
  &\frac{\partial \left< \delta E^2 \right>_{{\bf k},\omega}^0}{\partial t}
  + \mbox{Im}\,\frac{4}{\pi k^2\epsilon^*({\bf k},\omega)}
  \sum_ae_a^2 \int d{\bf v}\, \delta(\omega-{\bf k \cdot v})f_a({\bf v})\\
  &+\sum_{\sigma=\pm 1}\sum_\alpha\left[
    \epsilon '({\bf k},\sigma\omega_{\bf k}^\alpha)
    \frac{\partial I_{\bf k}^{\sigma \alpha}}{\partial t}
    +2\,\mbox{Im}\,\epsilon({\bf k},\sigma\omega_{\bf k}^\alpha)
    I_{\bf k}^{\sigma \alpha}\right]\delta(\omega-\sigma\omega_{\bf k}^\alpha)\\
  &=\mbox{Im}\,\mathcal{P} \frac{4}{\pi k^2\epsilon^{*}({\bf k},\omega)}
  \sum_a e^2_a\, \int d{\bf v}\,\delta(\omega-{\bf k \cdot v})f_a({\bf v})\\
  & +\sum_{\sigma=\pm 1}\sum_\alpha
  \frac{4\delta(\omega-{\bf k \cdot v})}
  {k^2\epsilon '({\bf k},\sigma\omega_{\bf k}^\alpha)}
  \sum_a e^2_a\, \int d{\bf v}\,\delta(\omega-{\bf k \cdot v})f_a({\bf v}).
\end{split}
\end{equation}

According to the definition \eqref{field-fluc}, the argument
of the dielectric constant excludes the eigenmodes. Hence, the
second term on the left-hand side of Eq. \eqref{noneigenmode-ql}
is implicitly taken with the principal value, which means it
cancels out the first term of the right-hand side. Thus, by assuming
$\partial \left< \delta E^2 \right>_{{\bf k},\omega}^0/\partial t=0$,
and removing the common factor $\sum_{\sigma=\pm 1}\sum_\alpha
\delta(\omega-\sigma\omega_{\bf k}^\alpha)$, we are left with
\begin{equation}
  \label{eigenmode-ql}
  \begin{split}
    \epsilon '({\bf k},\sigma\omega_{\bf k}^\alpha)
    \frac{\partial I_{\bf k}^{\sigma \alpha}}{\partial t}
    +2\,\mbox{Im}\,\epsilon({\bf k},\sigma\omega_{\bf k}^\alpha)
    I_{\bf k}^{\sigma \alpha}
  =\frac{4}{k^2\epsilon '({\bf k},\sigma\omega_{\bf k}^\alpha)}
  \sum_a e^2_a\, \int d{\bf v}\,\delta(\omega-{\bf k \cdot v})f_a({\bf v}).
\end{split}
\end{equation}
And, rearranging Eq. \eqref{eigenmode-ql}, we obtain
\begin{equation}
  \begin{split}
\label{eigenmode-ql2}
    \frac{\partial I_{\bf k}^{\sigma \alpha}}{\partial t}
    =-\frac{2\,\mbox{Im}\,\epsilon({\bf k},\sigma\omega_{\bf k}^\alpha)}
    {\epsilon '({\bf k},\sigma\omega_{\bf k}^\alpha)}I_{\bf k}^{\sigma \alpha}
  +\frac{4}{k^2[\epsilon '({\bf k},\sigma\omega_{\bf k}^\alpha)]^2}
  \sum_a e^2_a\, \int d{\bf v}\,\delta(\omega-{\bf k \cdot v})f_a({\bf v}),
\end{split}
\end{equation}
that is equivalent to the first two terms of Eq.~\eqref{wave-lin},
which describe the effects of spontaneous and induced emissions,
respectively. Both processes are determined by the linear
 wave-particle resonance conditions
 $\sigma \omega_{\bf k}^\alpha-{\bf k\cdot v}=0$.
Therefore, we may conclude that noneigenmodes do not alter the
quasilinear wave kinetic equation.

\subsection{Noneigenmode contribution to particle kinetic equation}
For the particle kinetic equation, however, the outcome is different.
In this case, the inclusion of noneigenmode field fluctuations leads to
the rigorous Balescu-Lenard collision integral, resulting in the merge
of two essential formulations of the plasma kinetic theory, without any
\emph{ad hoc} addition. The procedure starts by inserting the generalized wave
electric field intensity, given by Eq. \eqref{gen-dens}, in Eq. \eqref{part}
\begin{equation}
  \begin{split}
  \frac{\partial f_a}{\partial t}&=\frac{\pi e_a^2}{m_a^2}
  \int d{\bf k}\, \int d\omega\,
  \left( \frac{\bf k}{k}\cdot \frac{\partial}{\partial {\bf v}}\right)
  \delta(\omega -{\bf k \cdot v})
  \biggl[\mbox{Im}\, \mathcal{P} \frac{m_a}
    {2\pi^3k\, \epsilon^*({\bf k},\omega)} f_a({\bf v})\\
    &+\frac{2}{\pi k^2|\epsilon({\bf k},\omega)|^2}
    \sum_b e_b^2\int d{\bf v'} \delta(\omega-{\bf k\cdot v'})
    f_b({\bf v '})\left( \frac{\bf k}{k}\cdot
    \frac{\partial f_a({\bf v})}{\partial {\bf v}}\right)\\
  &+\sum_\sigma\sum_\alpha\frac{\pi m_a\,\delta(\omega-\sigma\omega_{\bf k}^\alpha)}
      {2\pi^3 k\, \epsilon '({\bf k},\sigma\omega_{\bf k}^\alpha)}f_a({\bf v})
      +\sum_\sigma\sum_\alpha I_{\bf k}^{\sigma \alpha}\,
      \delta(\omega-\sigma\omega_{\bf k}^\alpha)
      \left( \frac{\bf k}{k}\cdot
      \frac{\partial f_a({\bf v})}{\partial {\bf v}}\right)\biggr].
  \end{split}
\label{part-eig}
\end{equation}

Proceeding with the $\omega$ integration, we redefine the velocity
distribution function by taking the ambient density factor out
\begin{equation}
  \label{Fa}
f_a({\bf v})= \hat n F_a({\bf v}), \quad \int d^3v\ F_a({\bf v})=1,
\end{equation}
% LFZ180216: Talvez seja desnecessário o espaçamento extra antes do d^3v. O
% espaço parece tão grande ...
% Sabrina: OK
and, making explicit use of the definition for
$\mbox{Im}\,\epsilon({\bf k},\omega)$:
\begin{displaymath}
  \mbox{Im}\epsilon({\bf k},\omega)
  =-\pi \sum_\alpha \frac{4\hat n \pi e_a^2}{m_a k^2} \int d{\bf v}\,
  {\bf k \cdot}\frac{\partial F_a}{\partial {\bf v}}\delta(\omega-{\bf k\cdot v}),
\end{displaymath}
we obtain the following form of the generalized particle kinetic
equation including both the eigenmode and noneigenmode contributions
\cite{YZKS16}
\begin{equation}
\begin{split}
  \frac{\partial F_a({\bf v})}{\partial t}
  &=\sum_b\frac{2e_a^2e_b^2n_b}{m_a}\frac{\partial }{\partial v_i}
  \int d{\bf k}\ \int d{\bf v'}
  \frac{k_ik_j}{k^4}\frac{\delta({\bf k \cdot v - k \cdot v'})}
    {|\epsilon({\bf k, k\cdot v})|^2}\\
  &\times \left(\frac{\partial }{\partial v_j} -\frac{m_a}{m_b}
    \frac{\partial }{\partial v'_j}\right)
  F_a({\bf v})F_b({\bf v'})\\
  &+\frac{\pi e_a^2}{m_a^2}\sum_\sigma\sum_{\alpha=L,S}
  \int d{\bf k} \biggl(\frac{\bf k}{k}
\cdot\frac{\partial}{\partial{\bf v}}\biggr)
\delta(\sigma\omega_{\bf k}^\alpha-{\bf k}\cdot{\bf v})\\
&\times\biggl(\frac{m_a}{2\pi^2k\epsilon '({\bf k},\sigma\omega_{\bf k}^L)}
F_a({\bf v})+I_{\bf k}^{\sigma\alpha}
\frac{\bf k}{k}\cdot\frac{\partial F_a({\bf v})}{\partial{\bf v}}\biggr),
\end{split}
\label{Fa-gen}
\end{equation}
where the first term on the right-hand side is the contribution
of the noneigenmode inclusion in the particle kinetic equation
and, as mentioned before, corresponds to
% LFZ180216: Que tal "and corresponds to", em lugar de "it is"?
% Sabrina: OK
the Balescu-Lenard collision integral.
The other term is the usual quasilinear particle equation, which
contains the velocity friction term, related with the spontaneous
emission process, and the velocity space diffusion term.


In Eq. \eqref{Fa-gen}, the linear response $|\epsilon({\bf k},
{\bf k\cdot v})|^2$ in the denominator of the Balescu-Lenard
integral can be approximated under the assumption that the most
important contributions to the noneigenmode fluctuations is in
the region around $v=0$ , i.e., the region with the highest
concentration of particles in the velocity distribution function.
Therefore, we may treat these various angular frequency arguments
as if they have the basic form of ${\bf k \cdot v}\approx 0$ and
approximate the collision integral as
\begin{equation}
  \label{approx}
  \epsilon({\bf k},{\bf k \cdot v})\approx
  \epsilon({\bf k},0)=1+\sum_a \frac{2 \omega_{pa}^2}{k^2v_{T_a}^2}
  =1+\frac{2 \omega_{pe}^2}{k^2v_{Te}^2}\left( 1+\frac{T_e}{T_i} \right),
\end{equation}
where $\omega_{pe}=\sqrt{4 \pi n_e e^2/m_e}$ is the plasma frequency,
$v_{Te}= \sqrt{2T_e/m_e}$ is the electron thermal velocity, and $T_e$
and $T_i$ are, respectively, the electron and ion temperature expressed
in units of energy. This leads to
\begin{equation}
\begin{split}
  \frac{\partial F_a({\bf v})}{\partial t}
  &=\sum_b\frac{2e_a^2e_b^2n_b}{m_a}\frac{\partial }{\partial v_i}
  \int d{\bf k}\ \int d{\bf v'}\,
  \frac{k_ik_j\lambda_{De}^4\delta({\bf k \cdot v - k \cdot v'})}
    {|1+T_e/T_i+k^2 \lambda_{De}^2|^2}\\
  &\times \left(\frac{\partial }{\partial v_j} -\frac{m_a}{m_b}
    \frac{\partial }{\partial v'_j}\right)
  F_a({\bf v})F_b({\bf v'})\\
  &+\frac{\pi e_a^2}{m_a^2}\sum_\sigma\sum_{\alpha=L,S}
  \int d{\bf k} \biggl(\frac{\bf k}{k}
\cdot\frac{\partial}{\partial{\bf v}}\biggr)
\delta(\sigma\omega_{\bf k}^\alpha-{\bf k}\cdot{\bf v})\\
&\times\biggl(\frac{m_a}{2\pi^2k\epsilon '({\bf k},\sigma\omega_{\bf k}^L)}
F_a({\bf v})+I_{\bf k}^{\sigma\alpha}
\frac{\bf k}{k}\cdot\frac{\partial F_a({\bf v})}{\partial{\bf v}}\biggr),
\end{split}
\label{Fa-gen-approx}
\end{equation}
where $\lambda_{De}=\sqrt{T_e/4\pi n_ee^2}$ is the Debye length.

The collision integral also has a further approximation where,
assuming $2[\omega_{pe}/kv_{Te}]^2\ll 1$, the linear response
function can be written as $\epsilon({\bf k},{\bf k \cdot v})
\approx 1$. In such case, the wave number range of noneigenmode
fluctuations must be restricted to the short wavelength regime,
$k^2\lambda_{De}^2\gg 1$. This approximation leads to the well
known Landau collision integral, which is the approximation used
in Refs. \cite{Tigik2016a,Tigik2017a}. A very detailed discussion
about the limits and assumptions of the weak coupling approximation,
which leads to the Landau collision integral,
% LFZ180216: Acho que vai uma vígula aqui, depois de "integral"
% Sabrina: OK
 can be seen in Chapter
4 of Ref. \cite{Tigik2015}.
In Ref. \cite{Tigik2015} the Balescu-Lenard collision integral was
obtained through the Klimontovich statistical formalism
\cite{klimo,klimon}, which is also the starting point for
the derivation of the weak turbulence theory \cite{YZKS16} and the
basis of the simplified approach presented in \Cref{sec:kin-gen-eq}.
The implementation depicted in Ref. \cite{Tigik2015} did not make
any mention about the connection of the eigenmode and noneigenmode
contributions to each term of the particle kinetic equation. However,
the final result was exactly the same, which means that
% LFZ180216: "means that"?
% Sabrina: OK
the real novelty
here are not the equations, but it is this eigenmode/noneigenmode
relationship with collective and collisional processes, respectively.
The truly innovative outcome regarding new equations and the inclusion
of noneigenmode fluctuations will appear in the wave dynamics, as will
become clear in the next section.


\section{Noneigenmode contribution to nonlinear wave
  kinetic equation}
\label{sec:non-non}
For the nonlinear wave kinetic equation, the implementation is far more
extensive and complicated than the two previous cases. In this text, we
provide a brief explanation of the method, which is not very different
from the quasilinear case, though it is way more complex, and present some
critical remarks regarding the inclusion of noneigenmodes in the nonlinear
wave equation. It is worth to mention again, that a more detailed description
of the procedure
% LFZ190822: Talvez fique melhor "... a more detailed description of the
% procedures for inclusion of the ..."
% Sabrina: ok
for the inclusion of the noneigenmode contribution in the weak turbulence's
equations can be seen in \cite{YZKS16}.

Now, let us consider all terms in Eq. \eqref{nlebe}. On the
left-hand side, we have the quasilinear equations that, as
discussed before, are not altered by the inclusion of the
noneigenmode fluctuations. Thus, we can call the left-hand
side as ``$NL$'', and work only with the nonlinear part, on
right-hand side of \eqref{nlebe}, and substitute the total
electric field fluctuation, given by Eq. \eqref{gen-dens},
in these terms. Moreover, we are interested in how the
noneigenmode contribution affects the emissions in the
eigenmode frequency range, i.e., $\omega=\sigma \omega_{\bf k}$.
Therefore, all the noneigenmode fluctuations $\left<\delta E^2
\right>_{{\bf k}, \omega}^0$, with superscript $0$ and subscript
${\bf k},\omega$ to be more specific, can be neglected at this
point since, by definition, they are excluded from the region
that satisfies the $\omega=\sigma\omega_{\bf k}$ condition.
The noneigenmode fluctuations with different arguments
($\left< \delta E^2\right>_{{\bf k'},\omega '}^0$ and
$\left< \delta E^2\right>_{{\bf k-k'},\omega-\omega '}^0$),
though, must be retained. After all these considerations,
we have a generalized form of Eq. \eqref{nlebe}:
\begin{align}
  \mbox{NL}
 =-&2\int d{\bf k'} \int d\omega '\
     \biggl\{ \left[\chi^{(2)}({\bf k'},\omega '|{\bf k-k'},
     \omega-\omega ')\right]^2
     \nonumber\\
  &\times\biggl[\frac{\left< \delta E^2 \right>_{{\bf k-k'},\omega-\omega '}^0
     +\sum_{\sigma ''}\sum_{\gamma} I_{\bf k -k'}^{\sigma '' \gamma}
     \delta(\omega-\omega ' -\sigma '' \omega_{\bf k -k'}^\gamma)}
     {\epsilon({\bf k'},\omega ')}
  \label{nlebe-gen}\\
  &+\frac{\left< \delta E^2 \right>_{{\bf k'},\omega '}^0
  +\sum_{\sigma '}\sum_{\beta}
  I_{\bf k'}^{\sigma ' \beta}\delta(\omega '-\sigma ' \omega_{\bf k'}^\beta)}
     {\epsilon({\bf k-k'},\omega-\omega ')}\biggr]
     -\bar \chi^{(3)}({\bf k'},\omega|-{\bf k'},-\omega '|{\bf k},\omega)
     \nonumber\\
     &\times\biggl[ \left< \delta E^2 \right>_{{\bf k'},\omega '}^0
	+\sum_{\sigma '}\sum_{\beta}I_{\bf k'}^{\sigma ' \beta}
	\delta(\omega '-\sigma ' \omega_{\bf k'}^\beta) \biggr]\biggr\}
	\sum_{\sigma}\sum_{\alpha}
  I_{\bf k}^{\sigma \alpha}\delta(\omega-\sigma \omega_{\bf k}^\alpha)
	\nonumber\\
  +&2\int d{\bf k'} \int d\omega '
  \frac{|\chi^{(2)}({\bf k'},\omega '|{\bf k-k'},\omega-\omega ')|^2}
     {\epsilon^*({\bf k},\omega)}\biggl[ \left< \delta E^2 \right>_{{\bf k'},\omega '}^0
  +\sum_{\sigma '}\sum_{\beta}
     I_{\bf k'}^{\sigma ' \beta}\delta(\omega '-\sigma ' \omega_{\bf k'}^\beta) \biggr]
  \nonumber\\
  &\times \biggl[  \left< \delta E^2 \right>_{{\bf k-k'},\omega-\omega '}^0
     +\sum_{\sigma"}\sum_{\gamma} I_{\bf k -k'}^{\sigma" \gamma}
     \delta(\omega-\omega ' -\sigma" \omega_{\bf k -k'}^\gamma) \biggr]
     \nonumber\\
-&\frac{4}{\pi}\int d{\bf k '} \int d\omega '\
\frac{1}{k^2|\epsilon({\bf k'},\omega ')|^2}
\biggl\{ \frac{[\chi^{(2)}({\bf k'},\omega '|{\bf k-k'},\omega-\omega ')]^2}
{\epsilon({\bf k-k'},\omega-\omega ')}\sum_{\sigma}\sum_{\alpha}
   I_{\bf k}^{\sigma \alpha}\delta(\omega-\sigma \omega_{\bf k}^\alpha)
   \nonumber\\
  &-\frac{|\chi^{(2)}({\bf k'},\omega '|{\bf k-k'},\omega-\omega ')|^2}
   {\epsilon^*({\bf k},\omega)}
     \biggl[ \left< \delta E^2 \right>_{{\bf k-k'},\omega-\omega '}^0
     +\sum_{\sigma"}\sum_{\gamma}I_{\bf k -k'}^{\sigma" \gamma}
     \delta(\omega-\omega ' -\sigma" \omega_{\bf k -k'}^\gamma)\biggr]\biggr\}
    \nonumber
\end{align}
\begin{align}
  &\times\sum_ae_a^2\int d{\bf v}\, \delta(\omega '-{\bf k \cdot v})F_a({\bf v})
  \nonumber\\
-&\frac{4}{\pi}\int d{\bf k '} \int d\omega '\
\frac{1}{|{\bf k - k'}|^2|\epsilon({\bf k-k'},\omega-\omega ')^2}
\biggl\{ \frac{[\chi^{(2)}({\bf k'},\omega '|{\bf k-k'},\omega-\omega ')]^2}
   {\epsilon({\bf k'},\omega ')}\nonumber\\
  &\times\sum_{\sigma}\sum_{\alpha}
  I_{\bf k}^{\sigma \alpha}\delta(\omega-\sigma \omega_{\bf k}^\alpha)
-\frac{|\chi^{(2)}({\bf k'},\omega '|{\bf k-k'},\omega-\omega ')|^2}
{\epsilon^*({\bf k},\omega)}  \nonumber\\
&\times 
   \biggl[\left< \delta E^2 \right>_{{\bf k'},\omega '}^0
   +\sum_{\sigma '}\sum_{\beta}I_{\bf k'}^{\sigma ' \beta}
   \delta(\omega '-\sigma ' \omega_{\bf k'}^\beta)\biggr]\biggr\}
   \sum_a e_a^2\int d{\bf v}\,
   \delta(\omega-\omega '-{\bf (k-k')\cdot v})F_a({\bf v}).\nonumber
\end{align}

In the above equation, the noneigenmode fluctuations are represented
by the terms  $\left<\delta E^2 \right>_{{\bf k'}, \omega '}^0$, and
$\left< \delta E^2\right>_{{\bf k-k'},\omega-\omega '}^0$. It is important
to reinforce that if one is interested only in nonlinear terms related
to collective and spontaneous fluctuations in the eigenmode frequency
range, then all of these terms might be ignored. In such case, we are
back to the standard weak turbulence theory.

After an extensive algebraic manipulation that involves making explicit
use of definition \eqref{field-fluc} in the noneigenmode field fluctuations,
carrying out the $\omega '$ integrations, then using Eq. \eqref{principal}
to decompose the denominators
$1/\epsilon({\bf k-k'},\sigma \omega_{\bf k}^\alpha-{\bf k'\cdot v})$
and $1/\epsilon^*({\bf k},\omega)$ into principal parts and imaginary terms,
and taking the imaginary part of the resulting equation, we finally obtain
the noneigenmode corrections to the nonlinear wave kinetic equation:
\begin{equation}
  \label{corr}
  \begin{split}
    \mbox{corr}&=-\sum_{\sigma '}\sum_\beta\sum_a \,8\pi e_a^2
    \int d{\bf k'}\int d{\bf v}\,
    \frac{|\chi^{(2)}({\bf k'},\sigma '\omega_{\bf k'}^\beta|{\bf k-k'},
      ({\bf k-k'}){\bf \cdot v})|^2}
    {|{\bf k-k'}|^2|\epsilon[{\bf k-k'},{\bf k-k'},({\bf k-k'}){\bf \cdot v}]|^2}\\
    &\times \left[ \frac{I_{\bf k}^{\sigma \alpha}}
      {\epsilon '({\bf k'},\sigma '\omega_{\bf k'}^\beta)}
      -\frac{I_{\bf k'}^{\sigma '\beta}}
	{\epsilon '({\bf k},\sigma \omega_{\bf k}^\alpha)}\right] 
    \delta[\sigma\omega_{\bf k}^\alpha-\sigma '\omega_{\bf k'}^\beta
    -({\bf k-k'}){\bf \cdot v}]f_a({\bf v})\\
    &-\sum_a \frac{4 e_a^2}{\pi}
    \int d{\bf k '}\int d{\bf v}\,\mbox{Im}\biggl[ \mathcal{P}\,
    \frac{2[\chi^{(2)}({\bf k'},{\bf k' \cdot v}|{\bf k-k'},
      \sigma\omega_{\bf k}^\alpha-{\bf k'\cdot v})]^2}
    {\epsilon({\bf k-k'},\sigma \omega_{\bf k}^\alpha-{\bf k'\cdot v})}\\
    &-\bar \chi^{(3)}({\bf k'},{\bf k' \cdot v}|{\bf -k'},{\bf -k'\cdot v}
    |{\bf k},\sigma\omega_{\bf k}^\alpha)\biggr]
    \frac{I_{\bf k}^{\sigma \alpha}}{k'^2|\epsilon({\bf k'},{\bf k'\cdot v})|^2}
    f_a({\bf v})\\
    &+\sum_a\sum_b \frac{24 e_a^2 e_b^2}{\pi}\int d{\bf k '}
    \int d{\bf v} \int d{\bf v'}\,
    \frac{|\chi^{(2)} \left[ {\bf k'},{\bf k' \cdot v'}|
     {\bf k-k'},({\bf k-k'}){\bf \cdot v} \right]|^2}{k'^2|{\bf k-k'}|^2
   |\epsilon({\bf k'},{\bf k'\cdot v'})|^2
 |\epsilon({\bf k-k'},({\bf k-k'}){\bf \cdot v})|^2}\\
    &\times \frac{\delta \left[ \sigma\omega_{\bf k}^\alpha-{\bf k \cdot v}
	+{\bf k' \cdot}({\bf v-v'})\right]}
    {\epsilon '({\bf k},\sigma\omega_{\bf k}^\alpha)}f_a({\bf v}) f_b({\bf v'}).
  \end{split}
\end{equation}

At this point, we have to define an approximation for the various response
functions with the angular frequency replaced by ${\bf k \cdot v}$,
appearing in the denominator of the correction terms. To do so, we must
remember that waves are not subjected to Debye screening, therefore we
cannot assume $k^2\lambda_{De}^2\gg 1$, that would lead to the same
approximation $\epsilon({\bf k},0) \approx 1 $, which reduces the
Balescu-Lenard collision integral to the Landau collision integral.
Instead, we make use of Eq. \eqref{approx}, $\epsilon({\bf k},{\bf k \cdot v})\approx
\epsilon({\bf k},0) = 1+(2 \omega_{pe}^2/k^2v_{Te}^2)\left( 1+T_e/T_i \right)$,
which is the same we applied to the collision integral in the previous
subsection. Thus, applying this approximation and then adding the
right-hand side of \eqref{corr} to the right-hand side of Eq.
\eqref{wave-lin}, we obtain a generalized wave kinetic equation,
which includes both the eigenmode and noneigenmode fluctuations:
\begin{eqnarray}
  \frac{\partial I^\alpha_{\bf k}}{\partial t}
    &=&-\frac{2 \mbox{Im}\ \epsilon({\bf k}, \sigma \omega_{\bf k}^\alpha)}
    {\epsilon '({\bf k},\sigma \omega_{\bf k}^\alpha)}I^\alpha_{\bf k}
    +\sum_a\frac{4e^2}{k^2[\epsilon '({\bf k},\sigma\omega_{\bf k}^\alpha)]^2}
    \int d{\bf v}\, \delta(\sigma\omega_{\bf k}^\alpha-{\bf k \cdot v})f_a({\bf v})
	\nonumber\\
      &&-4\pi \sum_{\alpha,\beta,\gamma}\ \sum_{\sigma ',\sigma ''}\int d{\bf k '}\
    \frac{|\chi^{(2)}({\bf k'},\sigma '\omega_{\bf k'}^\beta|{\bf k-k'},
      \sigma ''\omega_{\bf k-k'}^\gamma)|^2}
    {\epsilon '({\bf k},\sigma \omega_{\bf k}^{\alpha})}
    \nonumber\\
      &&\times \biggl(\frac{I_{\bf k-k'}^{\gamma}\ I_{\bf k}^{\alpha}}
       {\epsilon '({\bf k'},\sigma ' \omega_{\bf k'}^{\beta})}
       +\frac{I_{\bf k'}^{\beta}\ I_{\bf k}^{\alpha}}
       {\epsilon '({\bf k-k'},\sigma ''\omega_{\bf k-k'}^{\gamma})}
       -\frac{I_{k'}^{\beta}\ I_{\bf k-k'}^{\gamma}}
       {\epsilon '({\bf k},\sigma \omega_{\bf k}^{\alpha})}\biggr)
       \delta\bigl(\sigma \omega_{\bf k}^{\alpha}-\sigma ' \omega_{\bf k'}^{\beta}
	 -\sigma '' \omega_{\bf k-k'}^{\gamma}\bigr)
    \nonumber\\
    &&-\sum_{\alpha,\beta}\ \sum_{\sigma '}
    \int d{\bf k '}\ A_{\alpha,\beta}({\bf k,k'})
    I_{\bf k'}^{\beta}\ I_{\bf k}^{\alpha}
       -\sum_a\,\frac{16 e_a^2}{\epsilon '({\bf k},\sigma\omega_{\bf k}^\alpha)}
       \sum_{\sigma '}\sum_{\beta}
    \nonumber\\
    &&\times\int d{\bf v}\int d{\bf k '}\,
    \frac{|\chi^{(2)}({\bf k'},\sigma '\omega_{\bf k'}^{\beta}|{\bf k-k'},
      \sigma \omega_{\bf k}^\alpha-\sigma '\omega_{\bf k'}^{\beta})|^2}
    {|{\bf k-k'}|^2|\epsilon({\bf k-k'},\sigma \omega_{\bf k}^\alpha
      -\sigma '\omega_{\bf k'}^{\beta})|^2}
    \nonumber\\
    &&\times \left[ \frac{I_{\bf k}^{\sigma \alpha}}
      {\epsilon({\bf k '},\sigma '\omega_{\bf k'}^\beta)}
    -\frac{I_{\bf k'}^{\sigma '\beta}}
    {\epsilon '({\bf k },\sigma \omega_{\bf k}^\alpha)}\right]
   \delta[\sigma \omega_{\bf k}^\alpha-\sigma '\omega_{\bf k'}^\beta
   -({\bf k-k'}){\bf \cdot v}] f_a({\bf v})
   \nonumber\\
    &&\textcolor{darkblue}{
       -\sum_a \,\frac{16 e_a^2}{\epsilon '({\bf k},\sigma\omega_{\bf k}^\alpha)}
       \sum_{\sigma '}\sum_\beta\,
    \int d{\bf k'}\int d{\bf v}\,
       \frac{|{\bf k-k'}|^2 \lambda_{De}^4
       |\chi^{(2)}({\bf k'},\sigma '\omega_{\bf k'}^\beta|{\bf k-k'},0)|^2}
    {|1+T_e/T_i+|{\bf k-k'}|^2 \lambda_{De}^2|^2}}
    \nonumber\\
    &&\textcolor{darkblue}{\times \left[ \frac{I_{\bf k}^{\sigma \alpha}}
      {\epsilon '({\bf k'},\sigma '\omega_{\bf k'}^\beta)}
      -\frac{I_{\bf k'}^{\sigma '\beta}}
	{\epsilon '({\bf k},\sigma \omega_{\bf k}^\alpha)}\right] 
    \delta[\sigma\omega_{\bf k}^\alpha-\sigma '\omega_{\bf k'}^\beta
    -({\bf k-k'}){\bf \cdot v}]f_a({\bf v})}
	 \nonumber\\
      &&\textcolor{skobeloff}{-\sum_a \frac{8 e_a^2}{\pi}
	 \int d{\bf k '}\int d{\bf v}\,
	 \frac{k'^2\lambda_{De}^4}{|1+T_e/T_i+k'^2\lambda_{De}^2|^2}}
	 \nonumber\\
    &&\textcolor{skobeloff}{
     \times\mbox{Im}\biggl[ \mathcal{P}\,
    \frac{2[\chi^{(2)}({\bf k'},0|{\bf k-k'},\sigma\omega_{\bf k}^\alpha)]^2}
    {\epsilon({\bf k-k'},\sigma \omega_{\bf k}^\alpha)}
       -\bar \chi^{(3)}({\bf k'},0|{\bf -k'},0|{\bf k},\sigma\omega_{\bf k}^\alpha)\biggr]
       I_{\bf k}^{\sigma \alpha}f_a({\bf v})}
      \nonumber\\
 &&\textcolor{softred}{+\sum_a\sum_b
     \frac{48 e_a^2 e_b^2}{\pi[\epsilon '({\bf k},\sigma\omega_{\bf k}^\alpha)]^2}
     \int d{\bf k '} \int d{\bf v} \int d{\bf v'}\,
     \frac{k'^2|\chi^{(2)} \left( {\bf k'},0|{\bf k-k'},0 \right)|^2}
     {|1+T_e/T_i+k^2 \lambda_{De}^2|^2}}
    \nonumber\\
  &&\textcolor{softred}{\times \frac{|{\bf k-k'}|^2\lambda_{De}^8}
     {|1+T_e/T_i+|{\bf k-k'}|^2 \lambda_{De}^2|^2} 
     \delta \left[ \sigma\omega_{\bf k}^\alpha-{\bf k \cdot v}
	+{\bf k' \cdot}({\bf v-v'})\right] f_a({\bf v}) f_b({\bf v'})},
       \label{wave-gen}
\end{eqnarray}
where the coefficient $A_{\alpha,\beta}({\bf k},{\bf k'})$ is given by
Eq. \eqref{coef-esp}.

Above, looking at the first noneigenmode term, in blue, one may notice
the extreme similarity with its previous expression, which describes the
spontaneous scattering process. In fact, this term is nothing but a
\emph{correction} to the spontaneous scattering process. Colored in
dark-green is the new, rigorous expression for the \emph{collisional
  damping}. The pink equation in the end, depicts the so-far unknown
\emph{electrostatic bremsstrahlung} process. The final expressions for
the three new contributions to the wave kinetic equation are given below.

\subsubsection{Correction for the spontaneous scattering}
The blue term in Eq. \eqref{wave-gen}, which depicts the
correction for the spontaneous scattering term is:
\begin{equation}
  \label{corrLS}
  \begin{split}
     \frac{\partial I_{\bf k}^{\sigma L}}{\partial t}\biggr|_{\mbox{\tiny corr}}
    &= - 4\hat n e^2 \sum_{\sigma '}  \int d{\bf k'}\int d{\bf v}\,
       \frac{|{\bf k-k'}|^2 \lambda_{De}^4\,
       |\chi^{(2)}({\bf k'},\sigma '\omega_{\bf k'}^L|{\bf k-k'},0)|^2}
    {|1+T_e/T_i+|{\bf k-k'}|^2 \lambda_{De}^2|^2}\\
    &\times \sigma \omega_{\bf k}^L \left( \sigma '\omega_{\bf k'}^L I_{\bf k}^{\sigma L}
      -\sigma \omega_{\bf k}^L I_{\bf k'}^{\sigma 'L}\right) 
    \delta[\sigma\omega_{\bf k}^L-\sigma '\omega_{\bf k'}^L
    -({\bf k-k'}){\bf \cdot v}]\left[ F_e({\bf v}) +F_i({\bf v})\right],\\
     \frac{\partial I_{\bf k}^{\sigma S}}{\partial t}\biggr|_{\mbox{\tiny corr}}
     &= - 4\hat n e^2\mu_{\bf k}\mu_{\bf k'}
     \sigma \omega_{\bf k}^L \sum_{\sigma '}  \int d{\bf k'}\int d{\bf v}\,
       \frac{|{\bf k-k'}|^2 \lambda_{De}^4\,
       |\chi^{(2)}({\bf k'},\sigma '\omega_{\bf k'}^S|{\bf k-k'},0)|^2}
    {|1+T_e/T_i+|{\bf k-k'}|^2 \lambda_{De}^2|^2}\\
    &\times \left( \sigma '\omega_{\bf k'}^S I_{\bf k}^{\sigma S}
      -\sigma \omega_{\bf k}^S I_{\bf k'}^{\sigma 'S}\right)
    \delta[\sigma\omega_{\bf k}^S-\sigma '\omega_{\bf k'}^S
    -({\bf k-k'}){\bf \cdot v}]\left[ F_e({\bf v}) +F_i({\bf v})\right],
  \end{split}
\end{equation}
where $\mu_{\bf k}=|k|^3\lambda_{De}^3 \left(m_e/m_i\right)^{1/2}
\left( 1+3T_i/T_e \right)^{1/2}$.

After a careful analysis on the second order susceptibility, we have
the following expressions for $L$ and $S$ waves
\begin{equation}
  \begin{split}
    u_{k,k'}^{L\, (\mbox{\tiny corr})}
    &= - \frac{\hat n e^2}{m_e^2\omega_{pe}^4}\sigma \omega_{\bf k}^L
    \sum_{\sigma '}  \int d{\bf k'}\int d{\bf v}\,
       \frac{({\bf k-k'})^2}{k^2 k'^2|1+T_e/T_i+({\bf k-k'})^2 \lambda_{De}^2|^2}\\
    &\times  \left( \sigma '\omega_{\bf k'}^L I_{\bf k}^{\sigma L}
      -\sigma \omega_{\bf k}^L I_{\bf k'}^{\sigma 'L}\right) 
    \delta[\sigma\omega_{\bf k}^L-\sigma '\omega_{\bf k'}^L
    -({\bf k-k'}){\bf \cdot v}]\left[ F_e({\bf v}) +F_i({\bf v})\right],
  \end{split}
\end{equation}
\begin{equation}
  \begin{split}
    u_{k,k'}^{S\, (\mbox{\tiny corr})}
    &= - \frac{\hat n e^2}{m_e^2\omega_{pe}^4}\mu_{\bf k}\mu_{\bf k'}
    \sigma \omega_{\bf k}^L \sum_{\sigma '}  \int d{\bf k'}\int d{\bf v}\,
    \frac{1}{k^2 k'^2\lambda_{De}^4}
    \left(1-\frac{T_e}{T_i} \frac{\bf k \cdot k'}{k'^2} \right)^2
    \left[1+\frac{T_e}{T_i}+({\bf k-k'})^2 \lambda_{De}^2\right]^{-2}\\
    &\times  \left( \sigma '\omega_{\bf k'}^S I_{\bf k}^{\sigma S}
      -\sigma \omega_{\bf k}^S I_{\bf k'}^{\sigma 'S}\right) 
    \delta[\sigma\omega_{\bf k}^S-\sigma '\omega_{\bf k'}^S
    -({\bf k-k'}){\bf \cdot v}]\left[ F_e({\bf v}) +F_i({\bf v})\right].
  \end{split}
\end{equation}

The corrected expression for the spontaneous scattering of
$L$ and $S$ waves are 
\begin{equation}
  \label{corrL}
  \begin{split}
  u_{k,k'}^{L}&=\sigma\omega_{\bf k}^L\,\frac{\hat n e^4}{m_e^2\,\omega_{pe}^4}
\sum_{\sigma '}\int d{\bf k'}\int d{\bf v}\;
\frac{({\bf k}\cdot{\bf k}')^2}{k^2\,k'^2}\;\delta[\sigma\omega_{\bf k}^L
-\sigma '\omega_{{\bf k}'}^L-({\bf k}-{\bf k}')\cdot{\bf v}]\\
&\times\,\bigl(\sigma\omega_{\bf k}^L\,I_{{\bf k}'}^{\sigma 'L}
  -\sigma '\omega_{{\bf k}'}^L\, I_{\bf k}^{\sigma L}\bigr)
  \left[1\textcolor{darkblue}{
  + \frac{1}{|1+T_e/T_i+({\bf k-k'})^2 \lambda_{De}^2|^2}} \right]
  \,[F_e({\bf v})+F_i({\bf v})],
\end{split}
\end{equation}
\begin{equation}
  \label{corrS}
  \begin{split}
  u_{k,k'}^{S}&=\sigma\omega_{\bf k}^L\frac{\hat n e^4}{m_e^2\omega_{pe}^4}
  \frac{\mu_{\bf k}\mu_{{\bf k}'}}{k^2\lambda_{De}^4}\sum_{\sigma '}
  \int \frac{d{\bf k'}}{k'^2}
  \int d{\bf v}\,\delta[ \sigma\omega_{\bf k}^S-\sigma '\omega_{{\bf k}'}^S
  -({\bf k}-{\bf k}')\cdot{\bf v}]
  \left(\sigma\omega_{\bf k}^L\frac{I_{{\bf k}'}^{\sigma 'S}}{\mu_{{\bf k}'}}
  -\sigma '\omega_{{\bf k}'}^L\frac{I_{\bf k}^{\sigma S}}{\mu_{\bf k}}\right)\\
&\times\left[\frac{({\bf k}\cdot{\bf k}')^2}{k^2k'^2}W_{{\bf k},{\bf k}'}
  \textcolor{darkblue}{
    +\frac{\left(1-\frac{T_e}{T_i} \frac{\bf k \cdot k'}{k'^2}\right)^2}
        {|1+T_e/T_i+({\bf k-k'})^2 \lambda_{De}^2|^2}}\right]
[F_e({\bf v})+F_i({\bf v})],
\end{split}
\end{equation}
where the blue term is the noneigenmode contribution.

\subsubsection{Collisional damping for $L$ and $S$ waves}
The collisional damping expression is given by the dark-green
term in Eq. \eqref{wave-gen}:
\begin{equation}
  \label{CD}
  \begin{split}
  \frac{\partial I_{\bf k}^{\sigma L}}{\partial t}\biggr|_{\mbox{\tiny coll}}
    &=-\sum_a \frac{4\hat n e_a^2}{\pi}\sigma \omega_{\bf k}^L
    \int d{\bf k '} \frac{k'^2\lambda_{De}^4}{|1+T_e/T_i+k'^2\lambda_{De}^2|^2}\\
    &\times \int d{\bf v}\,\mbox{Im}\biggl[ \mathcal{P}\,
    \frac{2[\chi^{(2)}({\bf k'},0|{\bf k-k'},\sigma\omega_{\bf k}^L)]^2}
    {\epsilon({\bf k-k'},\sigma \omega_{\bf k}^L)}
    -\bar \chi^{(3)}({\bf k'},0|{\bf -k'},0|{\bf k},\sigma\omega_{\bf k}^L)\biggr]
    F_a({\bf v})\,I_{\bf k}^{\sigma L},\\
   \frac{\partial I_{\bf k}^{\sigma S}}{\partial t}\biggr|_{\mbox{\tiny coll}}
    &=-\sum_a \frac{4\hat n e_a^2}{\pi}\sigma \mu_{\bf k} \omega_{\bf k}^L
    \int d{\bf k '} \frac{k'^2\lambda_{De}^4}{|1+T_e/T_i+k'^2\lambda_{De}^2|^2}\\
    &\times\int d{\bf v}\,\mbox{Im}\biggl[ \mathcal{P}\,
    \frac{2[\chi^{(2)}({\bf k'},0|{\bf k-k'},
      \sigma\omega_{\bf k}^S)]^2}
    {\epsilon({\bf k-k'},\sigma \omega_{\bf k}^S)}
    -\bar \chi^{(3)}({\bf k'},0|{\bf -k'},0|{\bf k},\sigma\omega_{\bf k}^S)\biggr]
    F_a({\bf v})\,I_{\bf k}^{\sigma S}.
  \end{split}
\end{equation}

With the appropriate approximations for the second order and partial
third order susceptibilities carefully analyzed \cite{YZKS16}, we 
% LFZ180216: "analyzed"?
% Sabrina: sim
obtain the following expressions for $L$ and $S$ waves
\begin{eqnarray}
  \label{gammaL}
\gamma_{\bf k}^{\sigma L\,(\mbox{\tiny coll})}
  &=& \sigma \omega_{\bf k}^L
\frac{4\hat{n}e^4\omega_{pe}^2}{T_e^2}\int d{\bf k}'
\frac{({\bf k}\cdot {\bf k'})^2\lambda_{De}^4}{k^2k'^4 
      |\epsilon({\bf k'},\sigma\omega_{\bf k}^L)|^2}\\
&&\times \left(1+\frac{T_e}{T_i}+({\bf k}-{\bf k'})^2\lambda_{De}^2\right)^{-2}
\int d{\bf v}\, {\bf k'}\cdot \frac{\partial F_e({\bf v})}{\partial {\bf v}}
\delta [\sigma\omega_{\bf k}^L-{\bf k'}\cdot{\bf v}]\nonumber\\
\gamma_{\bf k}^{\sigma S\, (\mbox{\tiny coll})}
  &=& \sigma \mu_{\bf k}\omega_{\bf k}^L
\frac{\hat{n}e^4\omega_{pe}^2}{T_e^2}\int d{\bf k}'
\frac{1}{k^2k'^4 |\epsilon({\bf k'},\sigma\omega_{\bf k}^S)|^2}\nonumber\\
&&\times \left(1+\frac{T_e}{T_i}+({\bf k}-{\bf k'})^2\lambda_{De}^2\right)^{-2}
   \left(1+\frac{2T_e}{T_i}\frac{{\bf k}\cdot{\bf k}'}{k^2}\right)
   \label{gammaS}\\
&&\times \int d{\bf v}\, {\bf k'}\cdot \frac{\partial}{\partial{\bf v}}
\left(F_e({\bf v})+\frac{m_e}{m_i}F_i({\bf v})\right)
\delta [\sigma\omega_{\bf k}^S-{\bf k'}\cdot{\bf v}].\nonumber
\end{eqnarray}

\subsubsection{Electrostatic bremsstrahlung for $L$ and $S$ waves}
The electrostatic bremsstrahlung equations are given by the pink term
in Eq. \eqref{wave-gen}:
\begin{equation}
\label{PkL,1}
  \begin{split}
    \frac{\partial I_{\bf k}^{\sigma L}}{\partial t}\biggr|_{\mbox{\tiny brem}}&=\sum_{a,b}
\frac{12n_ee_a^2e_b^2\omega_{pe}^2}{\pi}\int d{\bf k}'\int d{\bf v}\int d{\bf v}'
\,\frac{|\chi^{(2)}({\bf k}',0|{\bf k}-{\bf k}',0)|^2}{k'^2|{\bf k}-{\bf k}'|^2
|\epsilon({\bf k}',0)|^2|\epsilon({\bf k}-{\bf k}',0)|^2}\\
&\times\delta\left[\sigma\omega_{\bf k}^L-{\bf k}\cdot{\bf v}+{\bf k}'\cdot
  ({\bf v}-{\bf v}')\right]F_a({\bf v})F({\bf v}'),\\
\end{split}
\end{equation}
\begin{equation}
     \label{PkS,1}
  \begin{split}
 \frac{\partial}{\partial t}
\frac{I_{\bf k}^{\sigma S}}{\mu_{\bf k}}\biggr|_{\mbox{\tiny brem}}&=\sum_{a,b}
\frac{12n_ee_a^2e_b^2\omega_{pe}^2}{\pi}
\int d{\bf k}'\int d{\bf v}\int d{\bf v}'
\,\frac{|\chi^{(2)}({\bf k}',0|{\bf k}-{\bf k}',0)|^2}{k'^2|{\bf k}-{\bf k}'|^2
|\epsilon({\bf k}',0)|^2|\epsilon({\bf k}-{\bf k}',0)|^2}\\
&\times\delta\left[\sigma\omega_{\bf k}^S-{\bf k}\cdot{\bf v}+{\bf k}'\cdot
  ({\bf v}-{\bf v}')\right]F_a({\bf v})F({\bf v}').
  \end{split}
\end{equation}


In Ref. \cite{YZKS16}, the following approximation for the
second order susceptibility of both $L$ and $S$ waves was
employed
\begin{equation}
  \label{sec-susc-brem}
  \chi^{(2)}({\bf k}',0|{\bf k}-{\bf k}',0)
  =-i \sum_a\frac{e_a}{T_a} \frac{\omega_{pa}^2}{k k' |{\bf k-k'}|}\frac{1}{v_{Ta}^2}
  =\frac{i e}{T_e}\frac{1}{v_{Te}^2}\left( 1-\frac{T_e^2}{T_i^2}
  \frac{\omega_{pe}^2}{k k' |{\bf k-k'}|}\right).
\end{equation}

In the present work, however, we have revisited the subject, and found
that an improved approximation could be employed for the $L$ waves. The
derivation of this alternative approach, where we basically make the
same assumption ${\bf k \cdot v}\approx 0$, ${\bf k'\cdot v'}\approx 0$
and ${\bf k' \cdot v}\approx 0$ but a few steps later, is detailed in
\Cref{appA}. For the $S$ waves we keep the original approximation. 
Then, after applying the respective approximation for the
second order susceptibility for the $L$ and $S$ waves, we have
\begin{equation}
  \label{PkL}
  \begin{split}
      P_{\bf k}^{\sigma L}
  &= \frac{3e^2}{4\pi^3}\frac{1}{(\omega_\bk^L)^2}
    \left(1-\frac{m_e}{m_i}\frac{T_e}{T_i}\right)^2\frac{v_e^4}{k^2}
    \int d{\bf k}'k'^2|{\bf k- k'}|^2\left(1+\frac{T_e}{T_i}
    +({\bf k}-{\bf k}')^2\lambda_D^2\right)^{-2}\\
  &\times\left(1+\frac{T_e}{T_i}
    +{k'}^2\lambda_D^2\right)^{-2} \int d{\bf v}\int d{\bf v}'\,
    \delta[\sigma \omega_{\bf k}^L-{\bf k \cdot v}
    +{\bf k'\cdot}({\bf v}-{\bf v}')]
    \sum_aF_a({\bf v})\sum_bF_b({\bf v}')
  \end{split}
\end{equation}
\begin{equation}
  \label{PkS}
  \begin{split}
      P_{\bf k}^{\sigma S}
  &= \mu_{\bf k}\frac{3e^2T_e}{16\pi^3m_e}
    \biggl(1-\frac{T_e^2}{T_i^2}\biggr)^2\frac{1}{k^2\lambda_D^2}
    \int d{\bf k}'\left(1+\frac{T_e}{T_i}
    +({\bf k}-{\bf k}')^2\lambda_D^2\right)^{-2}\\
  & \times\left(1+\frac{T_e}{T_i}
    +{k'}^2\lambda_D^2\right)^{-2}\int d{\bf v}\int d{\bf v}'\,
    \delta[\sigma\omega_{\bf k}^S-{\bf k \cdot v}
    +{\bf k'\cdot}({\bf v}-{\bf v}')]
    \sum_a F_a({\bf v})\sum_b F_b({\bf v}').
  \end{split}
\end{equation}

\subsection{Generalized wave kinetic equation for $L$ and $S$ waves}
Adding Eqs. \eqref{gammaL} and \eqref{PkL} to Eq. \eqref{L,1},
and Eqs. \eqref{gammaS} and \eqref{PkS} to Eq. \eqref{S,1}, and
then substituting the spontaneous scattering term of both equations
with their respective corrected expressions, \eqref{corrL} and
\eqref{corrS}, we obtain the following generalized wave kinetic
equations for $L$ and $S$ waves, respectively:
\begin{equation}
  \label{L-gen}
  \begin{split}
\frac{\partial I_{\bf k}^{\sigma L}}{\partial t}
&=\frac{\omega_{pe}^2}{k^2}\int d{\bf v}\;
\delta(\sigma\omega_{\bf k}^L-{\bf k}\cdot{\bf v})
\biggl(\hat{n}\,e^2\,F_e({\bf v})+\pi\,(\sigma\omega_{\bf k}^L)\,
\;{\bf k}\cdot\frac{\partial F_e({\bf v})}{\partial{\bf v}}
\,I_{\bf k}^{\sigma L}\biggr)\\
&-\,\pi\sigma\,\omega_{\bf k}^L
\,\frac{e^2}{2T_e^2}\sum_{\sigma ',\sigma ''}\int d{\bf k'}\;
\frac{\mu_{{\bf k}-{\bf k}'}^S
\,({\bf k}\cdot{\bf k}')^2}{k^2\,k'^2\,|{\bf k}-{\bf k}'|^2}
\biggl(\sigma '\omega_{{\bf k}'}^L\,
\frac{I_{{\bf k}-{\bf k}'}^{\sigma ''S}}{\mu_{{\bf k}-{\bf k}'}^S}
I_{\bf k}^{\sigma L}\\
&+\,\sigma ''\omega_{{\bf k}-{\bf k}'}^L
\,I_{{\bf k}'}^{\sigma 'L}\,I_{\bf k}^{\sigma L}
-\sigma\omega_{\bf k}^L I_{{\bf k}'}^{\sigma 'L}
\frac{I_{{\bf k}-{\bf k}'}^{\sigma ''S}}
{\mu_{{\bf k}-{\bf k}'}^S}\biggr)
\;\delta(\sigma\omega_{\bf k}^L-\sigma '\omega_{{\bf k}'}^L
-\sigma ''\omega_{{\bf k}-{\bf k}'}^S)\\
&+\,\sigma\omega_{\bf k}^L\,\frac{e^2}{m_e^2\,\omega_{pe}^2}
\sum_{\sigma '}\int d{\bf k'}\int d{\bf v}\;
\frac{({\bf k}\cdot{\bf k}')^2}{k^2\,k'^2}\;\delta[\sigma\omega_{\bf k}^L
-\sigma '\omega_{{\bf k}'}^L-({\bf k}-{\bf k}')\cdot{\bf v}]\\
&\times\biggl\{\frac{\hat{n}\,e^2}{\omega_{pe}^2}
\bigl(\sigma\omega_{\bf k}^LI_{{\bf k}'}^{\sigma 'L}
  -\sigma '\omega_{{\bf k}'}^L I_{\bf k}^{\sigma L}\bigr)
  \left[1+ \frac{1}{|1+T_e/T_i+({\bf k-k'})^2 \lambda_{De}^2|^2} \right]
[F_e({\bf v})+F_i({\bf v})]\\
&+\pi\,\frac{m_e}{m_i}\,I_{{\bf k}'}^{\sigma 'L}
I_{\bf k}^{\sigma L}\;({\bf k}-{\bf k}')
\cdot\frac{\partial F_i({\bf v})}{\partial{\bf v}}\biggr\}
+2 I_{\bf k}^{\sigma L} \gamma_{\bf k}^{\sigma L}+ P_{\bf k}^{\sigma L},
  \end{split}
\end{equation}
\begin{equation}
  \label{S-gen}
  \begin{split}
\frac{\partial}{\partial t}\frac{I_{\bf k}^{\sigma S}}{\mu_{\bf k}^S}
&=\mu_{\bf k}^S\,\frac{\omega_{pe}^2}{k^2}
\int d{\bf v}\;\delta(\sigma\omega_{\bf k}^S-{\bf k}\cdot{\bf v})
\biggl[\hat{n}\,e^2\,[F_e({\bf v})+F_i({\bf v})]\\
&+\,\pi\,(\sigma\omega_{\bf k}^L)\,\biggl({\bf k}\cdot
\frac{\partial F_e({\bf v})}{\partial{\bf v}}
+\frac{m_e}{m_i}\;{\bf k}\cdot
\frac{\partial F_i({\bf v})}{\partial{\bf v}}\biggr)
\,\frac{I_{\bf k}^{\sigma S}}{\mu_{\bf k}^S}\biggr]\\
&-\,\pi\sigma\omega_{\bf k}^L\,\frac{e^2}{4T_e^2}
\sum_{\sigma ',\sigma ''}\int d{\bf k'}\;
\frac{\mu_{\bf k}^S[{\bf k}'\cdot({\bf k}-{\bf k}')]^2}{k^2k'^2|{\bf k}-{\bf k}'|^2}
\biggl(\sigma '\omega_{{\bf k}'}^L\,
I_{{\bf k}-{\bf k}'}^{\sigma ''L}
\frac{I_{\bf k}^{\sigma S}}{\mu_{\bf k}^S}\\
&+\,\sigma ''\omega_{{\bf k}-{\bf k}'}^L
\,I_{{\bf k}'}^{\sigma 'L}\frac{I_{\bf k}^{\sigma S}}{\mu_{\bf k}^S}
-\sigma\omega_{\bf k}^LI_{{\bf k}'}^{\sigma 'L}
I_{{\bf k}-{\bf k}'}^{\sigma ''L}\biggr)
\delta(\sigma\omega_{\bf k}^S-\sigma '\omega_{{\bf k}'}^L
-\sigma ''\omega_{{\bf k}-{\bf k}'}^L)\\
&+\frac{e^2}{m_e^2\omega_{pe}^2}\frac{\mu_{\bf k}^S\mu_{{\bf k}'}^S}{\lambda_{De}^4}
\sigma\omega_{\bf k}^L\sum_{\sigma '}\int \frac{d{\bf k'}}{k^2k'^2}\int d{\bf v}\;
\delta[\sigma\omega_{\bf k}^S-\sigma '\omega_{{\bf k}'}^S
-({\bf k}-{\bf k}')\cdot{\bf v}]\\
&\times\,\biggl\{\biggl[\frac{\hat{n}\,e^2}{\omega_{pe}^2}
\frac{({\bf k}\cdot{\bf k}')^2}{k^2\,k'^2}W_{{\bf k},{\bf k}'}
+\frac{\left(1-\frac{T_e}{T_i} \frac{\bf k \cdot k'}{k'^2}\right)^2}
{|1+T_e/T_i+({\bf k-k'})^2 \lambda_{De}^2|^2}\biggr]\\
&\times\biggl(\sigma\omega_{\bf k}^L\,
\frac{I_{{\bf k}'}^{\sigma 'S}}{\mu_{{\bf k}'}^S}
-\sigma '\omega_{{\bf k}'}^L\,
\frac{I_{\bf k}^{\sigma S}}{\mu_{\bf k}^S}\biggr)
\,[F_e({\bf v})+F_i({\bf v})]\\
&+\frac{m_e}{m_i}\frac{\pi({\bf k}\cdot{\bf k}')^2}{k^2\,k'^2}
\biggl(W_{{\bf k},{\bf k}'}+\sigma\,\sigma '\frac{k'}{k}\biggr)
\,\frac{I_{{\bf k}'}^{\sigma 'S}}{\mu_{{\bf k}'}^S}
\frac{I_{\bf k}^{\sigma S}}{\mu_{\bf k}^S}({\bf k}-{\bf k}')
\cdot\frac{\partial F_i({\bf v})}{\partial{\bf v}}\biggr\}
+2 I_{\bf k}^{\sigma S} \gamma_{\bf k}^{\sigma S}+ P_{\bf k}^{\sigma S},
\end{split}
\end{equation}
where the coefficient $W$ is given by Eq. \eqref{W}.



\chapter{Weakly turbulent processes in the presence of
  collisional interactions}
\label{main}
The generalization introduced in \cite{YZKS16} and summarized in
\Cref{noneigenmode}, is an innovative approach, in which the contribution
of the usually neglected noneigenmode fluctuations is taken into account
along with the well-known eigenmodes contribution. The outcome of this new
formulation is a first principles theory that combines both collective
and non-collective processes, assuming the propagation of electrostatic
oscillations in unmagnetized plasmas, without any \emph{ad hoc} addition.
Such theory is the basis of the present research, in which the
% LFZ190822: Não seria "whose", como estava antes?
% Sabrina: troquei para "in which"
primary objective
is performing numerical analysis of the generalized weak turbulence equations,
and then investigate its applications on the description and interpretation
of space plasma phenomena. More specifically, the solar plasma, where the
results of the first two studies \cite{Tigik2016b,Tigik2017a} have shown a
promising perspective. 

From the three new terms presented in the last chapter, we analyzed two:
the collisional damping \cite{Tigik2016b} and electrostatic bremsstrahlung
\cite{Tigik2017a}. The third term is a correction for the spontaneous
scattering effect, which is a nonlinear process. The first study, however,
analyses only the intensity of the collisional damping spectrum for $L$ and
$S$ waves by numerically integrating Eqs. \eqref{gammaL} and \eqref{gammaS},
respectively. Regarding the second study, the time evolution analysis was
made using quasilinear approximation, which does not involve nonlinear
wave-particle interaction, like the scattering effect. Besides that, a
small correction to a nonlinear term does not seem to justify a whole study
dedicated only to it.

In this chapter, we discuss some essential aspects regarding the results
obtained in both studies mentioned above. The results are presented in the
form of appended papers, using chronological order, in \Cref{coll-damp,sec:gen-supr}.




\section{Collisional damping rates for plasma waves}
\label{coll-damp}
In this first analysis, presented in Ref. \cite{Tigik2016b}, we have
numerically integrated Eqs. \eqref{gammaL}
% LFZ190822: O artigo está anexado, mas acho que mesmo assim vale dizer qual
% é o artigo na lista de referências. Algo como
% "In this first analysis, presented in Ref. \cite{Tigik...}, 
% we have numerically integrated Eqs. \eqref{gammaL} for ..."
% Sabrina: ok
for $L$ waves, and \eqref{gammaS} for $S$ waves, considering Maxwellian
velocity distribution function for the particles, without including the
new effects in the time evolution subroutine. The main objective here
was to comprehend important aspects of this so-far unknown expressions,
how they relate to other equivalent plasma processes and with the
heuristic Spitzer formula, for collisional damping. The main question
was: ``Are they equivalent?'', more specifically, ``Are their magnitudes,
at least, in the same order of magnitude?''. The answer was no. We found
out that the Spitzer formula overestimates the collisional damping effect
in several orders of magnitude in comparison with the new expression.
Further, the insignificance of the collisional damping became even more
evident when we compared its damping rate with the collisionless (Landau)
damping rate in the same figure.

The following paper is fairly self-sustained regarding the theory and
methods applied. All important equations are there, including their
normalized form and the normalization constants. 
% Any doubt regarding the origin of the new equations is filled by \Cref{noneigenmode}.


\includepdf[pages=2-6]{Tigik2016b.pdf}
\section{Generation of suprathermal electrons by collective processes
  in collisional plasmas}
\label{sec:gen-supr}
For the analysis of the electrostatic bremsstrahlung, the simple integration
of Eqs. \eqref{PkL} and \eqref{PkS}
was not very clarifying since we did not have any equivalent effect for
comparison. So we had to perform the time evolution of the system and analyze
the outcome. However, in order to put in evidence the action of the new effect,
is useful to start with the simplest possible situation. Therefore, instead of
using the complete set of nonlinear weak turbulence equations given by
Eqs. \eqref{L-gen} and \eqref{S-gen}, we restrict our analysis to the
quasilinear formalism, which includes single particle spontaneous emission
and wave-particle induced emission, and add the collisional damping and
electrostatic bremsstrahlung effects. The particle kinetic equation is
given by the usual quasilinear equation, where we included the Landau
collision integral (the same used in \cite{Tigik2016a}). After a long
evolution period, the system reaches a steady state where the electron
velocity distribution strongly resembles a core-halo distribution. An
interesting and surprising detail is that this suprathermal electron
population grows even in the presence of binary collisions. In the next
two subsections we provide some helpful information regarding the equations
and methods employed in \cite{Tigik2017a}.

Furthermore, an extra result is presented in \Cref{appB}, where
we compare the stationary state of the $L$ waves spectrum obtained
after the time evolution (using the results discussed in \cite{
  Tigik2016a}), with the asymptotic Langmuir spectrum generated by
a theoretical core-halo distribution, composed $95\%$ by a Maxwellian
core and $5\%$ by a kappa tail, in the presence and in the absence of
bremsstrahlung emission, for several values of the $\kappa$ index. It is
shown that the asymptotic spectra obtained in the case of distributions
with low $\kappa$ indexes are less affected by the electrostatic
bremsstrahlung emission than the spectra generated in the presence of
higher $\kappa$ indexes, and in the Maxwellian case. These findings
suggest that core-halo distributions may represent the asymptotic
equilibrium when the electrostatic bremsstrahlung is taken into account.

\subsection{Dimensionless equations}
In order to simplify the numerical analysis, it is convenient to write
the equations in terms of dimensionless variables. Thus, let us start
by some important definitions
\begin{align*}
  z\equiv \frac{\omega}{\omega_{pe}},
  & \quad \tau\equiv \omega_{pe}t, \quad
    {\bf q}\equiv \frac{{\bf k}v_e}{\omega_{pe}}, \quad
    {\bf u}\equiv \frac{\bf v}{v_e},\\
  \mu_{\bf q}^L
  = 1,&\quad\mu_{\bf q}^S = \frac{q^3}{2^{3/2}}
	\,\sqrt{\frac{m_e}{m_i}}\;\biggl(1+\frac{3T_i}{T_e}\biggr)^{1/2},
\end{align*}
\begin{align*}
  \Phi_a({\bf u})=  v_e^3 F_a({\bf u}),& \quad
    \calE_{\bf q}^{\sigma\alpha}= \frac{(2\pi)^2 g}{m_ev_e^2}
    \frac{I_{\bf k}^{\sigma\alpha}}{\mu_{\bf k}^\alpha},\quad
    P_{\bf k}^\alpha=\frac{m_ev_e^2}{(2\pi)^2g}\omega_{pe}P_{\bf q}^\alpha,\\
\end{align*}
where $g=1/[2^{3/2}\,(4\pi)^2\,\hat{n}\,\lambda_{De}^3]$ is the
plasma parameter, $\lambda_{De}=T_e/(4\pi\hat{n}e^2)=v_e^2/(2\omega_{pe}^2)$,
$v_e=(2T_e/m_e)^{1/2}$ is the electron thermal velocity and $T_e$ is the
electron temperature.

In terms of the normalized variables and quantities, the equations
for $L$ and $S$ waves, including the contributions from non-eigenmode
fluctuations, become
\begin{equation}
  \begin{split}
    \frac{\partial\calE_{\bf q}^{\sigma L}}{\partial\tau}
  &=\mu_{\bf q}^L\,\frac{\pi}{q^2}\int d{\bf u}\;
    \delta(\sigma z_{\bf q}^L-{\bf q}\cdot{\bf u})\\
    &\times\biggl(g\,\Phi_e({\bf u})+(\sigma z_{\bf q}^L)
    {\bf q}\cdot\frac{\partial \Phi_e({\bf u})}{\partial{\bf u}}
    \calE_{\bf q}^{\sigma L}\biggr)
    +2\calE_{\bf q}^{\sigma L}\gamma_{\bf q}^{\sigma L}+P_{\bf q}^{\sigma L},
      \end{split}
 \label{EpsL}
\end{equation}
\begin{equation}
  \begin{split}
    \frac{\partial{\cal E}_{\bf q}^{\sigma S}}{\partial \tau}
  &=\mu_{\bf q}^S\,\frac{\pi}{q^2}
    \int d{\bf u}\;
    \delta(\sigma z_{\bf q}^S-{\bf q}\cdot{\bf u})
    \biggl[g[\Phi_e({\bf u})+\Phi_i({\bf u})]\\
    &+\,(\sigma z_{\bf q}^L)\,\biggl({\bf q}\cdot
    \frac{\partial \Phi_e({\bf u})}{\partial{\bf u}}
    +\frac{m_e}{m_i}\;{\bf q}\cdot
    \frac{\partial \Phi_i({\bf u})}{\partial{\bf u}}\biggr)
    \,\calE_{\bf q}^{\sigma S}\biggr]
    +2\calE_{\bf q}^{\sigma S}\gamma_{\bf q}^{\sigma S}+P_{\bf q}^{\sigma S},
\end{split}
 \label{EpsS}
\end{equation}
where $\gamma_{\bf q}^{\sigma L}$ and $\gamma_{\bf q}^{\sigma S}$ describe
the collisional damping rate for $L$ and $S$ waves, respectively
\begin{equation}
  \label{GqL}
  \begin{split}
    \gamma_{\bf q}^{\sigma L}&=
    \frac{2gz_{\bf q}^L}{q^2}\int d{\bf q}'
    \frac{({\bf q}\cdot {\bf q'})^2}{{q'}^4
    |\epsilon({\bf q'},z_{\bf q}^L)|^2}
    \left(1+\frac{T_e}{T_i}
    +\frac{({\bf q}-{\bf q'})^2}{2}\right)^{-2}\\
  &\times\int d{\bf u}\,{\bf q'}\cdot
    \frac{\partial \Phi_e({\bf u})}{\partial {\bf u}}
    \delta(z_{\bf q}^L-{\bf q'}\cdot{\bf u}),
  \end{split}
\end{equation}
\begin{equation}
  \label{GqS}
  \begin{split}
    \gamma_{\bf q}^{\sigma S}
    &=\frac{2gz_{\bf q}^L}{q^2}
    \int\frac{d{\bf q}'}{{q'}^4|\epsilon({\bf q'},z_{\bf q}^S)|^2}
    \left(1+\frac{T_e}{T_i}+\frac{({\bf q}-{\bf q'})^2}{2}\right)^{-2}
    \left(1+\frac{2T_e}{T_i}\frac{{\bf q}\cdot{\bf q}'}{q^2}\right)\\
  &\times \int d{\bf u}\, {\bf q'}
    \cdot\frac{\partial}{\partial {\bf u}}
    \left(\Phi_e({\bf u})+\frac{m_e}{m_i}
    \Phi_i({\bf u})\right)\delta(z_{\bf q}^S
    -{\bf q'}\cdot{\bf u}),
  \end{split}
\end{equation}
and $P_{\bf q}^{\sigma L}$ and $P_{\bf q}^{\sigma S}$ terms depict the
normalized electrostatic bremsstrahlung equations, given by
\begin{equation}
  \begin{split}
    P_{\bf q}^{\sigma L}
  &= \frac{12e^2}{\pi^3}
    \frac{1}{q^2}\frac{1}{(z_{\bf q}^L)^2}
    \left(1-\frac{m_e}{m_i}\frac{T_e}{T_i}\right)^2
    \frac{\omega_{pe}^2}{v_e}
    \int d{\bf q'}\,q'^2|{\bf q-q'}|^2\\
  &\times \left[2\left(1+\frac{T_e}{T_i}\right)+q'^2\right]^{-2}
    \left[2\left(1+\frac{T_e}{T_i}\right) +|{\bf q-q'}|^2\right]^{-2}
    \label{PqL}\\
  &\times \int d{\bf u}\int d{\bf u'}
    \delta \left[ \sigma z_{\bf q}^L-{\bf q\cdot u}
    +{\bf q'}\cdot ({\bf u-u'})\right]
    \sum_a \Phi_a({\bf u})\sum_b \Phi_b({\bf u'}),
 \end{split}
\end{equation}
\begin{equation}
  \begin{split}
  P_{\bf q}^{\sigma S}
  &= \mu_{\bf q}\frac{3e^2T_e}{16\pi^3m_e}
    \biggl(1-\frac{T_e^2}{T_i^2}\biggr)^2
    \frac{32}{q^2}\frac{\omega_{pe}^2}{v_e^3}\int d{\bf q}'\\
  &\times\left[2
    \left(1+\frac{T_e}{T_i}\right)+{q'}^2\right]^{-2}
    \left[2 \left(1+\frac{T_e}{T_i}\right)
    +({\bf q}-{\bf q}')^2\right]^{-2}
    \label{PqS}\\
  &\times\int d{\bf u}\int d{\bf u}'\,
    \delta[\sigma z_{\bf q}^L-{\bf q}'\bm{\cdot}{\bf u}'
    -({\bf q}-{\bf q}') \bm {\cdot}{\bf u}]
    \sum_a\Phi_a({\bf u})\sum_b\Phi_b({\bf u}').
  \end{split}
\end{equation}

The dimensionless form of the particle kinetic equation has the
following form
\begin{equation}
  \begin{split}
    \frac{\partial \Phi_a({\bf u})}{\partial \tau}
    &= \frac{e_a^2}{e^2}\frac{m_e^2}{m_a^2}
    \sum_\sigma\sum_{\alpha=L,S}\int d{\bf q}\;\biggl(\frac{\bf q}{q}
    \cdot\frac{\partial}{\partial{\bf u}}\biggr)\,\mu_{\bf q}^\alpha
    \;\delta(\sigma z_{\bf q}^\alpha-{\bf q}\cdot{\bf u})\\
    & \times\,\biggl(g\frac{m_a}{m_e}\frac{\sigma z_{\bf q}^L}{q}
    \,\Phi_a({\bf u})+\calE_{\bf q}^{\sigma\alpha}
    \;\frac{\bf q}{q}\cdot\frac{\partial \Phi_a({\bf u})}
    {\partial{\bf u}}\biggr)+\sum_b\theta_{ab}(\Phi_a,\Phi_b),
    \label{Feu-evol}
  \end{split}
\end{equation}
where $\theta_{ab}(\Phi_a,\Phi_b)$ is the linearized Landau
collision integral, given by
\begin{equation}
  \begin{split}
     \theta_{ab}(\Phi_a,\Phi_b) &= \Gamma_{ab}\left\{
     \frac{\partial}{\partial{\bf u}_a}\cdot
     \left(2\frac{m_a}{m_b}\Psi(u_{ab})
     \frac{{\bf u}_a}{u^3_a}\Phi_a\right)\right.\\
  &+\frac{\partial}{\partial{\bf u}_a}\cdot
     \left\{\left[\left(\Phi(u_{ab})
     -\frac{1}{2 u_{ab}^2}\Psi(u_{ab})\right)
     \frac{\partial^2u_a}{\partial{\bf u}_a
     \partial{\bf u}_a}\right]\cdot
     \frac{\partial \Phi_a}{\partial{\bf u}_a}\right\}
    \label{eq:coll-lin} \\
  &+\left. \frac{\partial}{\partial{\bf u}_a}\cdot
     \left[\left(\frac{1}{u_{ab}^2}\Psi(u_{ab})
     \frac{{\bf u}_a{\bf u}_a}{u_a^3}\right)\cdot
   \frac{\partial \Phi_a}{\partial{\bf u}_a}\right]\right\},
\end{split}
\end{equation}
where $u_{ab}\equiv u_a (v_{T_a}/v_{T_b})$, $u_a=v_a/v_{T_a}$,
with $v_{T_a}$ and $v_{T_b}$ representing the thermal velocity of
the particles of species $a$ and $b$, respectively. The quantity
$\Psi(x)\equiv\Phi(x)-x\Phi'(x)$ is an auxiliary function \cite{
  Gaffey1976}, in which $\Phi(x) \equiv \frac{2}{\sqrt{\pi}}
\int_0^x e^{-t^2}dt$ is the error function and $\Phi'(x)=
\frac{2}{\sqrt{\pi}}e^{-x^2}$ is its derivative. The factor
$\Gamma_{ab}$ is given by $\Gamma_{ab}=2\pi g Z_b^2 \ln\Lambda $,
where $g=1/[2^{3/2}(4\pi)^2\hat n \lambda_{De}^3]$ is the plasma
parameter, $\ln \Lambda$ is Coulomb logarithm, and $Z_{b=e,i}$ is
the charge number of ions and electrons, with $Z_e=1$.
The dispersion relations for $L$ and $S$ waves in dimensionless
form are given by
\begin{eqnarray*}
  z_{\bf q}^L = \left(1+\frac{3}{2}q^2\right)^{1/2},
 \quad   z_{\bf q}^S = \frac{q\, A}{(1+q^2/2)^{1/2}}, \quad
    A= \frac{1}{\sqrt{2}} \left(\frac{m_e}{m_i}\right)^{1/2}
     \left(1+3\frac{T_i}{T_e}\right)^{1/2}.
\end{eqnarray*}

The above set of integro-differential equations was numerically solved in
2D wave-number space and 2D velocity space, respectively. For the wave kinetic
equations, it was used a fourth-order Runge-Kutta method and, for the partial
differential equation describing the electron dynamics, we employed the
splitting algorithm. In both cases we used fixed time step $\Delta \tau=0.1$.
The wave number space grid has been assumed with $71\times 71$ points in
$k_{\perp}$ and $k_{\parallel}$, with $0<k_{\perp}v_e/\omega_{pe}<0.6$ and
$0<k_{\parallel}v_e/\omega_{pe}<0.6$. In the velocity space, the grid
configuration was  $71\times 141$ for $v_{\perp}/v_e$ and $v_{\parallel}/v_e$,
with velocity range $0<v_{\perp}/v_e<16$ and $-16<v_{\parallel}/v_e<16$.


\includepdf[pages=-]{Tigik2017a.pdf}

\chapter{Final remarks}
\label{cha:final}
In this thesis, we have presented the results of the first analyses 
% LFZ190822: Talvez seja mais correto dizer "of the first analyses ...", que
% é plural, enquanto que "analysis" é singular.
% Sabrina: ok
of a new
theory, depicted in \Cref{noneigenmode}, which combines collective processes
(eigenmodes) and collisional interactions (noneigenmodes), for Langmuir and
ion-sound waves, without any \emph{ad hoc} addition. The importance of this
formalism, became clear right in the first study, presented in \Cref{coll-damp},
where it was shown that the collisional damping rate calculated with the heuristic
Spitzer formula, that is independent of the wave-number, is highly overestimated
when compared with the rigorous expression, which has to be integrated in the
wave-number space. The Spitzer approximation is largely applied to the modeling
and interpretation of several solar phenomena observed by space-probes \cite{
  Vinas2000,Hannah2009,ZVST11,KRB12,KF13,Reid2014,Reid2015}, a fact that
emphasizes the necessity of the development of a mathematically rigorous
theory able to describe hybrid situations involving collective oscillations
and collisional interactions.

The second study, discussed in \Cref{sec:gen-supr}, focused on the
consequences of including the hitherto unknown effect of electrostatic
bremsstrahlung in the time evolution of a Maxwellian electron velocity
distribution, under the quasilinear regime. The unexpected outcome of this
analysis revealed an underlying process responsible for taking the electron
velocity distribution out of its initial equilibrium state, leading it to
a new quasi-equilibrium state that is satisfied by a velocity distribution
composed by a Maxwellian core with a suprathermal tail. Such velocity
distributions are pervasively observed in space plasmas \cite{Parker1958a,
  Vasyliunas68,Coroniti1974,FABMG75,Lin1998,Horne2005,Clark2015,Padovani2015,
  deSoria2016,Padovani2017,Deca2017}. However, until these days, despite
several attempts of explaining how they form and why space plasmas are
particularly prone to develop a suprathermal tail while retaining the
thermal core, no existing theory is capable of describing self-consistently
its origin and evolution. Again, the relevance of this research becomes
evident.

The ubiquity of inverse power-law velocity distribution in space plasmas
indicates the action of a fundamental physical process that might -or
might not- be the electrostatic bremsstrahlung. For instance, the possibility
of the formation of suprathermal electron velocity distributions in a mildly
collisional environment, as demonstrated in \cite{Tigik2017a}, has a positive
correlation with the velocity filtration model, which relies on the existence
of a high energy electron population in the solar transition region, located
between the chromosphere and the solar corona, to explain the temperature
inversion observed in the solar corona \cite{Scudder92a,Scudder92b,Anderson94,
  Scudder94,Scudder96,DS99,DS03,Teles15}.

% As a final remark, i
It is important to emphasize the fundamental aspect of the
theory presented in Ref. \cite{YZKS16}. The rigorous analysis performed during
this doctoral research, suggests that collisional interactions involving charged
particles enfold physical processes that go beyond the Coulomb collisions
effects. An enthralling prospect of this study lies in the possibility of these
underlying kinetic processes being connected to unexplained phenomena observed
in space plasmas. On that regard, the electrostatic weak turbulence theory for
collisional plasmas is a first step towards a more accurate description of the
microscopic physics of plasmas. Further development should extend this formalism
to include electromagnetic waves and, in the long term, the influence of ambient
magnetic field. 

These efforts on improving the theoretical description of kinetic processes in
plasmas and numerically analyzing the equations are just in time with current
technological progress in the resolution capability of spacecraft measurement
equipment. A great example of this new trend is NASA's Magnetospheric Multiscale
(MMS) mission that investigates the physics of magnetic reconnection in the
Earth's magnetosphere. Equipped with high-precision instruments \cite{Pollock2016,
  Torbert2016}, MMS was capable of directly measure wave-particle energy
exchange process in the ion kinetic scale \cite{Gershman2017}, for the first
time. Therefore, it is becoming evident that the kinetic scale is the next
frontier in observational plasma physics, and this will require a more accurate
and comprehensive theory.




\begin{appendix}
\chapter{Improved approximation for the second order susceptibility}
\label{appA}
\setcounter{equation}{0}
\renewcommand{\theequation}{A\arabic{equation}}
In this appendix, we present two different approximations applied to the
second-order susceptibility in the electrostatic bremsstrahlung equation
for Langmuir waves. Initially, for comparison, we show the original
simplification used in Ref.~\cite{YZKS16}. In the sequence, we introduce
an improved approximation for the same expression, highlighting the main
differences. This new version, though, is only suitable for fast-waves,
which is the case of Langmuir oscillations, but does not apply to the case
% LFZ190822: Aqui me parece que ficaria melhor dizer This new version, though,
% is only suitable for fast waves, which is the case of Langmuir oscillations,
% but does not apply to the case of ion-sound waves. does not apply
% Sabrina: ok
of ion-sound waves. Moreover, the results for ion-sound waves are quite
well behaved with the original approximation, which is the same that was
initially used for $L$ waves in Ref.~\cite{YZKS16}.
% LFZ190822: Acho que aqui cabe continuar dizendo "... for $L$ waves in
% Ref.~\cite{YZKS16}."
% Sabrina: ok
Therefore, $S$ waves will be considered only in
% LFZ190822: Acho melhor "will be considered", em vez de "will take part"
\Cref{sec:orig-appr}, while \Cref{sec:alt-approx} will be directed solely
% LFZ190822: Sugiro "... will be directed solely to the use in expressions
% related to $L$ waves.", em lugar de "the $L$ waves expression."
% Sabrina: acho que se eu usar "expressions" vai dar a impressão de que
% essa aproximação vai ser usada em outra equação que não seja a do
% electrostatic bremsstrahlung. Não lembro se a gente chegou a usar essa
% aproximação na expressão do amortecimento colisional. Mas acho que não.
to the  expression for $L$ waves.

\section{Original approximation}
\label{sec:orig-appr}
The initial expressions for $L$ and $S$ waves, still in terms of the
second order susceptibility are given by:
\begin{equation}
\label{PkL-A}
  \begin{split}
    \frac{\partial I_{\bf k}^{\sigma L}}{\partial t}\biggr|_{\mbox{\tiny brem}}&=\sum_{a,b}
\frac{12n_ee_a^2e_b^2\omega_{pe}^2}{\pi}\int d{\bf k}'\int d{\bf v}\int d{\bf v}'
\,\frac{|\chi^{(2)}({\bf k}',0|{\bf k}-{\bf k}',0)|^2}{k'^2|{\bf k}-{\bf k}'|^2
|\epsilon({\bf k}',0)|^2|\epsilon({\bf k}-{\bf k}',0)|^2}\\
&\times\delta\left[\sigma\omega_{\bf k}^L-{\bf k}\cdot{\bf v}+{\bf k}'\cdot
  ({\bf v}-{\bf v}')\right]F_a({\bf v})F({\bf v}'),
\end{split}
\end{equation}
\begin{equation}
     \label{PkS-A}
  \begin{split}
 \frac{\partial}{\partial t}
\frac{I_{\bf k}^{\sigma S}}{\mu_{\bf k}}\biggr|_{\mbox{\tiny brem}}&=\sum_{a,b}
\frac{12n_ee_a^2e_b^2\omega_{pe}^2}{\pi}
\int d{\bf k}'\int d{\bf v}\int d{\bf v}'
\,\frac{|\chi^{(2)}({\bf k}',0|{\bf k}-{\bf k}',0)|^2}{k'^2|{\bf k}-{\bf k}'|^2
|\epsilon({\bf k}',0)|^2|\epsilon({\bf k}-{\bf k}',0)|^2}\\
&\times\delta\left[\sigma\omega_{\bf k}^S-{\bf k}\cdot{\bf v}+{\bf k}'\cdot
  ({\bf v}-{\bf v}')\right]F_a({\bf v})F({\bf v}').
  \end{split}
\end{equation}

Assuming that the most important contribution comes from the region where
${\bf k}\cdot {\bf v}\simeq 0$, ${\bf k'}\cdot{\bf v'}\simeq 0$, and
${\bf k'}\cdot {\bf v}\simeq 0$,
\begin{equation}
\begin{split}
\chi_a^{(2)}&[{\bf k'},{\bf k'\cdot v'}|{\bf k-k'},({\bf k-k'})\cdot{\bf v}]
\simeq-\frac{2\pi ie_a^3}{k k'|{\bf k-k'}|m_a^2}\frac{m_a^2}{T_a^2}\int d{\bf v''}\\
&\left[-\frac{T_a}{m_a}\frac{{\bf k'}\cdot({\bf k-k'})}
{(-{\bf k}\cdot{\bf v''}+i0)}
\frac{({\bf k-k'})\cdot{\bf v''}}
   {[-({\bf k-k'})\cdot{\bf v''}+i0]^2} \right.
   + \frac{1}{(-{\bf k}\cdot{\bf v''}+i0)}
\frac{({\bf k'}\cdot{\bf v''})(({\bf k-k'})\cdot{\bf v''})}
{(-({\bf k-k'})\cdot{\bf v''}+i0)}\\
& -\frac{T_a}{m_a} \frac{({\bf k-k'})\cdot {\bf k'}}{(-{\bf k}\cdot{\bf v''}+i0)}
\frac{{\bf k'}\cdot{\bf v''}}{(-{\bf k'}\cdot{\bf v''}+i0)^2} \left.
+ \frac{1}{(-{\bf k}\cdot{\bf v''}+i0)}\frac{(({\bf k-k'})\cdot{\bf v''})
({\bf k'}\cdot{\bf v''})}{(-{\bf k'}\cdot{\bf v''}+i0)}
\right] f_a({\bf v''})
\end{split}
\end{equation}
\begin{equation}
\begin{split}
\chi_a^{(2)}&[{\bf k'},{\bf k'\cdot v'}|{\bf k-k'},({\bf k-k'})\cdot{\bf v}]
\simeq-\frac{2\pi ie_a^3}{k k'|{\bf k-k'}|m_a^2}\frac{m_a^2}{T_a^2}\int d{\bf v''}\\
&\times \left[-\frac{T_a}{m_a}
  \frac{{\bf k'}\cdot({\bf k-k'})}{(-{\bf k}\cdot{\bf v''}+i0)}
\frac{({\bf k-k'})\cdot{\bf v''}}
{(-({\bf k-k'})\cdot{\bf v''})
[-({\bf k-k'})\cdot{\bf v''}+i0]} \right.
 + \frac{1}{(-{\bf k}\cdot{\bf v''}+i0)}
\frac{({\bf k'}\cdot{\bf v''})(({\bf k-k'})\cdot{\bf v''})}
{[-({\bf k-k'})\cdot{\bf v''}+i0]}\\
&-\frac{T_a}{m_a} \frac{({\bf k-k'})\cdot {\bf k'}}
{(-{\bf k}\cdot{\bf v''}+i0)}
\frac{{\bf k'}\cdot{\bf v''}}
{(-{\bf k'}\cdot{\bf v''})(-{\bf k'}\cdot{\bf v''}+i0)}
\left. + \frac{1}{(-{\bf k}\cdot{\bf v''}+i0)}
\frac{(({\bf k-k'})\cdot{\bf v''})
({\bf k'}\cdot{\bf v''})}
{(-{\bf k'}\cdot{\bf v''}+i0)}
\right] f_a({\bf v''})
\end{split}
\end{equation}

Simplifying, and writing the coefficient in front of the expression in a
different way,
\begin{equation}
\begin{split}
\chi_a^{(2)}[{\bf k'},&{\bf k'\cdot v'}|{\bf k-k'},({\bf k-k'})\cdot{\bf v}]
\simeq-i\frac{e_a}{T_a}\frac{\omega_{pa}^2}{k k'|{\bf k-k'}|}\frac{1}{v_{a}^2}\\
&\times\int d{\bf v''}\,\biggl\{\frac{T_a}{m_a}
\frac{{\bf k'}\cdot({\bf k-k'})}{{\bf k}\cdot{\bf v''}}
\left[\frac{1}{({\bf k-k'})\cdot{\bf v''}}
+\frac{1}{({\bf k'})\cdot{\bf v''}}\right]\\
&+\underbrace{ \frac{\bf k'\cdot v''}{\bf k\cdot v''} 
+ \frac{({\bf k}-{\bf k'}){\bf \cdot v''}}{\bf k\cdot v''}}_{=1}
\biggr\} F_a({\bf v''}) + i\,{\rm Im}\chi^{(2)},
\end{split}
\end{equation}
\begin{equation}
  \begin{split}
\chi_a^{(2)}[{\bf k'},&{\bf k'\cdot v'}|{\bf k-k'},({\bf k-k'})\cdot{\bf v}]
\simeq-i\frac{e_a}{T_a}\frac{\omega_{pa}^2}{k k'|{\bf k-k'}|}\frac{1}{v_{a}^2}\\
&\times\int d{\bf v''}\,\left[\frac{T_a}{m_a}
\frac{{\bf k'}\cdot({\bf k-k'})}{({\bf k-k'})\cdot{\bf v''}}+1
\right] F_a({\bf v''})
\end{split}
\end{equation}
where we have used $f_a= \hat{n} F_a$.

The term containing ${\bf v''}$ vanishes, for distribution function which
depends only on the absolute value of the velocity, like the Maxwellian 
distribution. Therefore,
\begin{equation}
\chi_a^{(2)}[{\bf k'},{\bf k'\cdot v'}|{\bf k-k'},({\bf k-k'})\cdot{\bf v}]
\simeq-i\frac{e_a}{T_a}\frac{\omega_{pa}^2}{k k'|{\bf k-k'}|}\frac{1}{v_{a}^2}
\end{equation}

Defining $e_i=e$, $e_e=-e$ and assuming $n_e=n_i=\overline{n}$, we have
\begin{equation}
\begin{split}
\label{chi2,5}
\chi^{(2)}[{\bf k'},{\bf k'\cdot v'}|{\bf k-k'},({\bf k-k'})\cdot{\bf v}]
&= \sum_a\chi_a^{(2)}[{\bf k'},{\bf k'\cdot v'}|{\bf k-k'},({\bf k-k'})\cdot{\bf v}]\\
&\simeq  i e\frac{1}{k k'|{\bf k-k'}|}
\left(\frac{\omega_{pe}^2}{T_e v_{e}^2}-\frac{\omega_{pi}^2}{T_i v_{i}^2}\right)\\
&=i e\frac{\omega_{pe}^2}{k k'|{\bf k-k'}|}
\left(\frac{m_e}{T_e 2T_e}-\frac{m_e}{m_i}\frac{m_i}{T_i 2 T_i}\right)\\
&=i\frac{e}{T_e}\frac{\omega_{pe}^2}{k k'|{\bf k-k'}|}\frac{1}{v_{e}^2}
\left(1-\frac{m_e}{m_i}\frac{2T_e^2}{m_e}\frac{m_i}{2T_i^2}\right)\\
&=i\frac{e}{T_e}\frac{\omega_{pe}^2}{k k'|{\bf k-k'}|}\frac{1}{v_{e}^2}
\left(1-\frac{T_e^2}{T_i^2}\right).
\end{split}
\end{equation}

Finally we have the original approximation:
\begin{equation}
\chi^{(2)}[{\bf k'},{\bf k'\cdot v'}|{\bf k-k'},({\bf k-k'})\cdot{\bf v}]
=i\frac{e}{2T_e}\frac{1}{k k'|{\bf k-k'}|\lambda_{De}^2}
\left(1-\frac{T_e^2}{T_i^2}\right),
\end{equation}
and the resulting equations
\begin{eqnarray}
\label{bremsstL-A}
P_{\bf k}^{\sigma L} &=& \frac{3e^2T_e}{16\pi^3m_e}
\biggl(1-\frac{T_e^2}{T_i^2}\biggr)^2\frac{1}{k^2\lambda_{De}^2}
\int d{\bf k}'\left(1+\frac{T_e}{T_i}
+{k'}^2\lambda_{De}^2\right)^{-2}\left(1+\frac{T_e}{T_i}
+({\bf k}-{\bf k}')^2\lambda_{De}^2\right)^{-2}
\nonumber\\
&& \times\int d{\bf v}\int d{\bf v}'\,
\delta[\sigma\omega_{\bf k}^L-{\bf k}\bm{\cdot}{\bf v}
+{\bf k}'\bm{\cdot}({\bf v}-{\bf v}')]
\sum_aF_a({\bf v})\sum_bF_b({\bf v}'),
\\
\label{bremsstS-A}
P_{\bf k}^{\sigma S} &=& \mu_{\bf k}\frac{3e^2T_e}{16\pi^3m_e}
\biggl(1-\frac{T_e^2}{T_i^2}\biggr)^2\frac{1}{k^2\lambda_{De}^2}
\int d{\bf k}'\left(1+\frac{T_e}{T_i}
+{k'}^2\lambda_{De}^2\right)^{-2}\left(1+\frac{T_e}{T_i}
+({\bf k}-{\bf k}')^2\lambda_{De}^2\right)^{-2}
\nonumber\\
&& \times\int d{\bf v}\int d{\bf v}'\,
\delta[\sigma\omega_{\bf k}^S-{\bf k}\bm{\cdot}{\bf v}
+{\bf k}'\bm{\cdot}({\bf v}-{\bf v}')]
\sum_aF_a({\bf v})\sum_bF_b({\bf v}').
\end{eqnarray}

\section{Alternative approach}
\label{sec:alt-approx}
From Eq.(4.24) in Ref. \cite{YZKS16} we have
\begin{eqnarray}
\label{Eq4.24}
&&
\sum_{a,b}\frac{48e_a^2e_b^2}{\pi}\frac{1}{[\epsilon'({\bf k},\sigma
\omega_{\bf k}^\alpha)]^2}\int d{\bf k'}\int d{\bf v}\int d{\bf v'}
\delta[\sigma\omega_{\bf k}^\alpha-{\bf k\cdot v}+{\bf k'}\cdot
({\bf v-v'})]\\
&&
\times \frac{|\chi^{(2)}[{\bf k'},{\bf k'\cdot v'}|{\bf k-k'},({\bf k-k'})\cdot
{\bf v}]|^2}{k'^2|{\bf k-k'}|^2|\epsilon({\bf k'},{\bf k'\cdot v'})|^2|
\epsilon[{\bf k-k'},({\bf k-k'})\cdot {\bf v}]|^2}f_a({\bf v}f_b({\bf v'}),
\nonumber
\end{eqnarray}
which depends on the second-order susceptibility, defined as follows,
\begin{eqnarray}
\label{chi2,1}
\chi_a^{(2)} (q_1|q_2)=-\frac{4\pi i e_a}{k_1k_2|{\bf k_1}+{\bf k_2}|}
\int d{\bf v''}\, \alpha_a^{(2)}({\bf v''};q_1|q_2)f_a({\bf v''}),
\end{eqnarray}
with $\alpha_a^{(2)}$ given by,
\begin{eqnarray}
\label{alpha2}
\alpha_a^{(2)}({\bf v''};q_1|q_2)f_a({\bf v''})=\frac{1}{2}\left[
({\bf k_1}\cdot{\bf g}_{q_1+q_2}^a)({\bf k_2}\cdot g_{q_2}^a)
+({\bf k_2}\cdot{\bf g}_{q_1+q_2}^a)({\bf k_1}\cdot g_{q_1}^a)\right],
\end{eqnarray}
and $g_q^a$ given by,
\begin{eqnarray}
\label{goper}
g_{{\bf k},\omega}^a=-\frac{e_a}{m_a}\frac{1}{\omega-{\bf k}\cdot{\bf v''}+i0}
\frac{\partial}{\partial {\bf v''}}.
\end{eqnarray}

Notice that we used variable ${\bf v''}$, since ${\bf v}$ and ${\bf v'}$ 
already appear in Eq. \eqref{chi2,1}.

Therefore, using Eq. \eqref{alpha2} in Eq. \eqref{chi2,1},
\begin{equation}
\begin{split}
\chi_a^{(2)}[{\bf k'},{\bf k'\cdot v'}|{\bf k-k'},({\bf k-k'})\cdot{\bf v}]
=&-\frac{4\pi ie_a}{k k'|{\bf k-k'}|}\frac{1}{2}\int d{\bf v''}\,\biggl[
({\bf k'}\cdot
{\bf g}_{{\bf k},{\bf k'}\cdot {\bf v'}+({\bf k-k'})\cdot {\bf v}}^a)
(({\bf k-k'})\cdot 
{\bf g}_{{\bf k-k'},({\bf k-k'})\cdot {\bf v}}^a)\\
&+(({\bf k-k'})\cdot
{\bf g}_{{\bf k},{\bf k'}\cdot {\bf v'}+({\bf k-k'})\cdot {\bf v}}^a)
({\bf k'}\cdot 
{\bf g}_{{\bf k'},{\bf k'}\cdot {\bf v'}}^a)
\biggr] f_a({\bf v''})
\end{split}
\end{equation}

By using Eq. \eqref{goper},
\begin{equation}
\begin{split}
\chi_a^{(2)}&[{\bf k'},{\bf k'\cdot v'}|{\bf k-k'},({\bf k-k'})\cdot{\bf v}]
=-\frac{2\pi ie_a^3}{k k'|{\bf k-k'}|m_a^2}\int d{\bf v''}\\
&\times\left\{[\frac{{\bf k'}\cdot \partial_{\bf v''}}
{[{\bf k'\cdot v'}+({\bf k-k'})\cdot{\bf v}-{\bf k}\cdot{\bf v''}+i0]}
\frac{({\bf k-k'})\cdot \partial_{\bf v''}}
{[({\bf k-k'})\cdot {\bf v}
-({\bf k-k'})\cdot{\bf v''}+i0]} \right.\\
&\left. + \frac{({\bf k-k'})\cdot \partial_{\bf v''}}
{[{\bf k'}\cdot {\bf v'}+({\bf k-k'})\cdot {\bf v}
-{\bf k}\cdot{\bf v''}+i0]}
\frac{{\bf k'}\cdot \partial_{\bf v''}}
{[{\bf k'\cdot v'}-{\bf k'}\cdot{\bf v''}+i0]}
\right\} f_a({\bf v''})
\end{split}
\end{equation}
\begin{equation}
\begin{split}
\chi_a^{(2)}&[{\bf k'},{\bf k'\cdot v'}|{\bf k-k'},({\bf k-k'})\cdot{\bf v}]
=-\frac{2\pi ie_a^3}{k k'|{\bf k-k'}|m_a^2}\int d{\bf v''}\\
&\times\left\{\frac{
{\bf k'}\cdot({\bf k-k'})}
{[{\bf k'\cdot v'}+({\bf k-k'})\cdot{\bf v}-{\bf k}\cdot{\bf v''}+i0]}
\frac{({\bf k-k'})\cdot \partial_{\bf v''}}
{[({\bf k-k'})\cdot {\bf v}
-({\bf k-k'})\cdot{\bf v''}+i0]^2} \right.\\
& + \frac{1}
{[{\bf k'\cdot v'}+({\bf k-k'})\cdot{\bf v}-{\bf k}\cdot{\bf v''}+i0]}
\frac{{\bf k'}\cdot \partial_{\bf v''}
({\bf k-k'})\cdot \partial_{\bf v''}}
{[({\bf k-k'})\cdot {\bf v}
-({\bf k-k'})\cdot{\bf v''}+i0]}\\
& + \frac{({\bf k-k'})\cdot {\bf k'}}
{[{\bf k'}\cdot {\bf v'}+({\bf k-k'})\cdot {\bf v}
-{\bf k}\cdot{\bf v''}+i0]}
\frac{{\bf k'}\cdot \partial_{\bf v''}}
{[{\bf k'\cdot v'}-{\bf k'}\cdot{\bf v''}+i0]^2}\\
&\left. + \frac{1}
{[{\bf k'}\cdot {\bf v'}+({\bf k-k'})\cdot {\bf v}
-{\bf k}\cdot{\bf v''}+i0]}
\frac{({\bf k-k'})\cdot \partial_{\bf v''}
{\bf k'}\cdot \partial_{\bf v''}}
{[{\bf k'\cdot v'}-{\bf k'}\cdot{\bf v''}+i0]}
\right\} f_a({\bf v''})
\end{split}
\end{equation}

Assuming a Maxwellian velocity distribution function
\begin{equation}
  \begin{split}
    \label{chi2,2}
    \chi_a^{(2)}&[{\bf k'},{\bf k'\cdot v'}|
    {\bf k-k'},({\bf k-k'})\cdot{\bf v}]
    =-\frac{2\pi i e_a^3}{k k'|{\bf k-k'}|m_a^2}
    \frac{m_a^2}{T_a^2}\int d{\bf v''}\\
    &\times\left\{-\frac{T_a}{m_a}\frac{ {\bf k'}
	\cdot({\bf k-k'})}{[{\bf k'\cdot v'}
	+({\bf k-k'})\cdot{\bf v}-{\bf k}\cdot{\bf v''}+i0]}
      \frac{({\bf k-k'})\cdot{\bf v''}}{[({\bf k-k'})
	\cdot {\bf v}-({\bf k-k'})\cdot{\bf v''}+i0]^2} \right.\\
    & + \frac{1}{[{\bf k'\cdot v'}+({\bf k-k'})\cdot{\bf v}
      -{\bf k}\cdot{\bf v''}+i0]}\frac{({\bf k'}\cdot{\bf v''})
      (({\bf k-k'})\cdot{\bf v''})}{[({\bf k-k'})\cdot {\bf v}
      -({\bf k-k'})\cdot{\bf v''}+i0]}\\
    & -\frac{T_a}{m_a} \frac{({\bf k-k'})\cdot {\bf k'}}
    {[{\bf k'}\cdot {\bf v'}+({\bf k-k'})\cdot {\bf v}
      -{\bf k}\cdot{\bf v''}+i0]} \frac{{\bf k'}\cdot{\bf v''}}
    {[{\bf k'\cdot v'}-{\bf k'}\cdot{\bf v''}+i0]^2}\\
    &\left. + \frac{1}{[{\bf k'}\cdot {\bf v'}+({\bf k-k'})
	\cdot {\bf v}-{\bf k}\cdot{\bf v''}+i0]}\frac{(({\bf k-k'})
	\cdot{\bf v''}) ({\bf k'}\cdot{\bf v''})}{[{\bf k'\cdot v'}
	-{\bf k'}\cdot{\bf v''}+i0]} \right\}f_a({\bf v''})
  \end{split}
\end{equation}

Here, instead of making the early assumption that
${\bf k} \cdot {\bf v}\simeq 0$, ${\bf k}' \cdot {\bf v}'\simeq 0$
and ${\bf k}'\cdot {\bf v} \simeq 0$, we take into account
the fact that Eq. \eqref{Eq4.24} contains a delta function:
\begin{equation}
{\bf k'\cdot v'}+({\bf k-k'})\cdot{\bf v}= \sigma\omega_{\bf k}^\alpha.
\end{equation}

With this small change we obtain a quite different expression than that obtained
at the same point in the previous approach:
\begin{equation}
\begin{split}
\chi_a^{(2)}&[{\bf k'},{\bf k'\cdot v'}|{\bf k-k'},({\bf k-k'})\cdot{\bf v}]
=-\frac{2\pi ie_a^3}{k k'|{\bf k-k'}|m_a^2}\frac{m_a^2}{T_a^2} \int d{\bf v''}\\
&\times\left\{-\frac{T_a}{m_a}\frac{{\bf k'}\cdot({\bf k-k'})}
{[\sigma\omega_{\bf k}^\alpha-{\bf k}\cdot{\bf v''}+i0]}
\frac{({\bf k-k'})\cdot{\bf v''}}
{[\sigma\omega_{\bf k}^\alpha-{\bf k'}\cdot {\bf v'}
-({\bf k-k'})\cdot{\bf v''}+i0]^2} \right.\\
& + \frac{1}
{[\sigma\omega_{\bf k}^\alpha-{\bf k}\cdot{\bf v''}+i0]}
\frac{({\bf k'}\cdot{\bf v''})
(({\bf k-k'})\cdot{\bf v''})}
{[\sigma\omega_{\bf k}^\alpha-{\bf k'}\cdot {\bf v'}
-({\bf k-k'})\cdot{\bf v''}+i0]}\\
& -\frac{T_a}{m_a} \frac{({\bf k-k'})\cdot {\bf k'}}
{[\sigma\omega_{\bf k}^\alpha-{\bf k}\cdot{\bf v''}+i0]}
\frac{{\bf k'}\cdot{\bf v''}}
{[\sigma\omega_{\bf k}^\alpha- ({\bf k- k'})\cdot{\bf v'}
-{\bf k'}\cdot{\bf v''}+i0]^2}\\
&\left. + \frac{1}
{[\sigma\omega_{\bf k}^\alpha-{\bf k}\cdot{\bf v''}+i0]}
\frac{(({\bf k-k'})\cdot{\bf v''})
({\bf k'}\cdot{\bf v''})}
{[\sigma\omega_{\bf k}^\alpha- ({\bf k- k'})\cdot{\bf v'}
-{\bf k'}\cdot{\bf v''}+i0]}
\right\} f_a({\bf v''}).
\end{split}
\end{equation}

And it is just after the previous assumption that we assume that the most
important contribution comes from the region where
${\bf k}\cdot {\bf v}\simeq 0$, ${\bf k'}\cdot{\bf v'}\simeq 0$,
and ${\bf k'}\cdot {\bf v}\simeq 0$, then
\begin{equation}
\begin{split}
\label{chi2,2b}
\chi_a^{(2)}&[{\bf k'},{\bf k'\cdot v'}|{\bf k-k'},({\bf k-k'})\cdot{\bf v}]
=-\frac{2\pi ie_a^3}{k k'|{\bf k-k'}|m_a^2}\frac{m_a^2}{T_a^2}\int d{\bf v''}\\
&\times \left\{-\frac{T_a}{m_a}\frac{{\bf k'}\cdot({\bf k-k'})}
{[\sigma\omega_{\bf k}^\alpha-{\bf k}\cdot{\bf v''}+i0]}
\frac{({\bf k-k'})\cdot{\bf v''}}
{[\sigma\omega_{\bf k}^\alpha
-({\bf k-k'})\cdot{\bf v''}+i0]^2} \right.\\
& + \frac{1}
{[\sigma\omega_{\bf k}^\alpha-{\bf k}\cdot{\bf v''}+i0]}
\frac{({\bf k'}\cdot{\bf v''})
(({\bf k-k'})\cdot{\bf v''})}
{[\sigma\omega_{\bf k}^\alpha-({\bf k-k'})\cdot{\bf v''}+i0]}\\
& -\frac{T_a}{m_a} \frac{({\bf k-k'})\cdot {\bf k'}}
{[\sigma\omega_{\bf k}^\alpha-{\bf k}\cdot{\bf v''}+i0]}
\frac{{\bf k'}\cdot{\bf v''}}
{[\sigma\omega_{\bf k}^\alpha-{\bf k'}\cdot{\bf v''}+i0]^2}\\
&\left. + \frac{1}
{[\sigma\omega_{\bf k}^\alpha-{\bf k}\cdot{\bf v''}+i0]}
\frac{(({\bf k-k'})\cdot{\bf v''})
({\bf k'}\cdot{\bf v''})}
{[\sigma\omega_{\bf k}^\alpha
-{\bf k'}\cdot{\bf v''}+i0]}
\right\} f_a({\bf v''})
\end{split}
\end{equation}

For $L$ waves, we can assume that the factor $\sigma\omega_{\bf k}^L$ is 
dominant on the denominators,
\begin{equation}
  \begin{split}
    \chi_a^{(2)}&[{\bf k'},{\bf k'\cdot v'}|{\bf k-k'},({\bf k-k'})\cdot{\bf v}]
    =-\frac{2\pi ie_a^3}{k k'|{\bf k-k'}|m_a^2}\frac{m_a^2}{T_a^2}\int d{\bf v''}\\
    &\times\left[-\frac{T_a}{m_a}[{\bf k'}\cdot({\bf k-k'})]
      \frac{({\bf k-k'})\cdot{\bf v''}}
      {(\sigma\omega_{\bf k}^L)^3}+ \frac{({\bf k'}\cdot{\bf v''})
        (({\bf k-k'})\cdot{\bf v''})}
      {(\sigma\omega_{\bf k}^L)^2} \right.\\
    &\left. -\frac{T_a}{m_a} [({\bf k-k'})\cdot {\bf k'}]
      \frac{{\bf k'}\cdot{\bf v''}}
      {(\sigma\omega_{\bf k}^L)^3}
      + \frac{(({\bf k-k'})\cdot{\bf v''})
        ({\bf k'}\cdot{\bf v''})}
      {(\sigma\omega_{\bf k}^L)^2}
    \right] f_a({\bf v''})
    \label{dom-freq}
  \end{split}
\end{equation}

Due its asymmetry, the terms, which are linear in ${\bf v''}$, will vanish
when integrated, remaining
\begin{equation}
\begin{split}
\chi_a^{(2)}[{\bf k'},{\bf k'\cdot v'}|{\bf k-k'},({\bf k-k'})\cdot{\bf v}]
=-\frac{2\pi ie_a^3}{k k'|{\bf k-k'}|m_a^2}\frac{m_a^2}{T_a^2}
\int d{\bf v''}\,\left[2\frac{({\bf k'}\cdot{\bf v''})
(({\bf k-k'})\cdot{\bf v''})}
{(\sigma\omega_{\bf k}^L)^2} 
\right] f_a({\bf v''}).
\end{split}
\end{equation}

As an approximation, we write
\begin{equation}
\begin{split}
\chi_a^{(2)}[{\bf k'},{\bf k'\cdot v'}|{\bf k-k'},({\bf k-k'})\cdot{\bf v}]
&=-\frac{2\pi ie_a^3}{k k'|{\bf k-k'}|m_a^2}\frac{m_a^2}{T_a^2}
2\frac{k'|{\bf k-k'}|}{(\sigma\omega_{\bf k}^L)^2} 
\int d{\bf v''}\,|{\bf v''}|^2 f_a({\bf v''})\\
&=-\frac{2\pi ie_a^3}{k m_a^2}\frac{m_a^2}{T_a^2}
\frac{2}{(\sigma\omega_{\bf k}^L)^2} \frac{2T_a}{m_a}\overline{n}\\
&=-i\frac{e_a}{m_a}\frac{\omega_{pa}^2}{k(\sigma\omega_{\bf k}^L)^2} 
\frac{m_a^2}{T_a^2}\frac{2T_a}{m_a}\nonumber
\end{split}
\end{equation}
which leads to
\begin{equation}
\label{chi2,3}
\chi_a^{(2)}[{\bf k'},{\bf k'\cdot v'}|{\bf k-k'},({\bf k-k'})\cdot{\bf v}]
\simeq -i4\frac{e_a}{m_a}\frac{\omega_{pa}^2}{(\omega_{\bf k}^L)^2} 
\frac{1}{kv_{a}^2}
\end{equation}

Defining $e_i=e$ and $e_e=-e$, and assuming $n_e=n_i=\overline{n}$, we have
\begin{equation}
\chi^{(2)}[{\bf k'},{\bf k'\cdot v'}|{\bf k-k'},({\bf k-k'})\cdot{\bf v}]
= \sum_a
\chi_a^{(2)}[{\bf k'},{\bf k'\cdot v'}|{\bf k-k'},({\bf k-k'})\cdot{\bf v}]
\end{equation}
\begin{equation}
\begin{split}
\sum_a
\chi_a^{(2)}[{\bf k'},{\bf k'\cdot v'}|{\bf k-k'},({\bf k-k'})\cdot{\bf v}]
\simeq & -4i\frac{e}{k(\omega_{\bf k}^L)^2}
\left(\frac{\omega_{pe}^2}{m_ev_e^2}-\frac{\omega_{pi}^2}{m_iv_i^2}\right)\\
=& -4i\frac{e}{k(\omega_{\bf k}^L)^2}
\left(\frac{\omega_{pe}^2}{m_ev_e^2}-\frac{(m_e/m_i)\omega_{pe}^2}{m_iv_i^2}\right)\\
=& -4ie\frac{\omega_{pe}^2}{k(\omega_{\bf k}^L)^2}\frac{1}{m_ev_e^2}
\left(1-\frac{m_e^2}{m_i^2}\frac{v_e^2}{v_i^2}\right)\\
=& -4ie\frac{\omega_{pe}^2}{k(\omega_{\bf k}^L)^2}\frac{m_e}{m_e2T_e}
\left(1-\frac{m_e^2}{m_i^2}\frac{2T_e}{m_e}\frac{m_i}{2T_i}\right)\\
=&-i\frac{2e}{T_e}\frac{\omega_{pe}^2}{k(\omega_{\bf k}^L)^2}
\left(1-\frac{m_e}{m_i}\frac{T_e}{T_i}\right)
\end{split}
\end{equation}
\begin{equation}
\label{chi2,4}
\chi^{(2)}[{\bf k'},{\bf k'\cdot v'}|{\bf k-k'},({\bf k-k'})\cdot{\bf v}]
\simeq i\frac{2e}{T_e}\frac{\omega_{pe}^2}{k(\omega_{\bf k}^L)^2}
\left(1-\frac{m_e}{m_i}\frac{T_e}{T_i}\right)
\end{equation}

Therefore, the resulting equation for the electrostatic bremsstrahlung
of $L$ waves in the new approximation is
\begin{equation}
  \label{PkL-A}
  \begin{split}
      P_{\bf k}^{\sigma L}
  &= \frac{3e^2}{4\pi^3}\frac{1}{(\omega_\bk^L)^2}
    \left(1-\frac{m_e}{m_i}\frac{T_e}{T_i}\right)^2\frac{v_e^4}{k^2}
    \int d{\bf k}'k'^2|{\bf k- k'}|^2\left(1+\frac{T_e}{T_i}
    +({\bf k}-{\bf k}')^2\lambda_D^2\right)^{-2}\\
  &\times\left(1+\frac{T_e}{T_i}
    +{k'}^2\lambda_D^2\right)^{-2} \int d{\bf v}\int d{\bf v}'\,
    \delta[\sigma \omega_{\bf k}^L-{\bf k \cdot v}
    +{\bf k'\cdot}({\bf v}-{\bf v}')]
    \sum_aF_a({\bf v})\sum_bF_b({\bf v}')
  \end{split}
\end{equation}

Since this approximation assumes that the dominant frequency range on
the denominators is given by the dispersion relation of Langmuir waves
(see Eq. \eqref{dom-freq}), which are high frequency oscillations
near the electron plasma frequency, it is not suitable for use in
% LFZ190822: "on", ou "in"?
% Sabrina: ok
the bremsstrahlung expression for $S$ waves.


\chapter{Asymptotic equilibrium}
\label{appB}
The results presented in Ref. \cite{Tigik2017a} suggest that in the presence
of electrostatic bremsstrahlung emission, the wave-particle system evolves to
a state of asymptotic equilibrium, in which the velocity distribution resembles
the ubiquitous core-halo velocity distribution. To test this assumption, we
search for the steady state solution of the kinetic equation for Langmuir waves,
taking into account the effects of collisional damping and electrostatic
bremsstrahlung, in addition to the spontaneous and induced emission processes,
\begin{equation}
  \frac{\partial {\cal E}_{\bf q}^{\sigma L}}
  {\partial \tau}
  =\mu_{\bf q}^L\frac{\pi}{q^2}
  \int d{\bf u}\;\delta(\sigma
  z_{\bf q}^L-{\bf q}\cdot{\bf u})\biggl(g\,\Phi_e({\bf u})
  +\sigma z_{\bf q}^L\,{\cal E}_{\bf q}^{\sigma L}
  \,{\bf q}\cdot\frac{\partial \Phi_e}{\partial{\bf u}}\biggr)
  +P_{\bf q}^{\sigma L}+2\gamma_{\bf q}^{\sigma L}{\cal E}_{\bf q}^{\sigma L}\approx 0,
  \label{asymp}
\end{equation}

For use in this asymptotic equation, let us assume isotropic distributions for
ions and electrons, which are the summation of Maxwellian and $\kappa_\beta$
distributions. In terms of dimensionless variables,
\begin{equation}
\label{fbeta}
f_{\beta}({\bf u})= \left(1-\frac{n_{\kappa\beta}}{n_e}\right)
\frac{1}{\pi^{3/2} u_{\beta}^3}\exp\left(-\frac{u^2}{u_\beta^2}\right)
+\frac{n_{\kappa\beta}}{n_e}\frac{1}{\pi^{3/2} \kappa_\beta^{3/2}
u_{\beta,\kappa}^3}
\frac{\Gamma(\kappa_\beta+1)}{\Gamma(\kappa_\beta-1/2)}
\left(1+\frac{u^2}{\kappa_\beta u_{\beta,\kappa}^2}
\right)^{-(\kappa_\beta+1)}~.
\end{equation}
where
\begin{equation}
u_{\beta,\kappa}= \frac{v_{\beta,\kappa}}{v_e},\quad
u_\beta=\frac{v_\beta}{v_e},\quad
u_{\beta,\kappa}^2= \frac{\kappa_\beta-3/2}{\kappa_\beta} u_\beta^2.
\end{equation}

The equilibrium is obtained setting expression \eqref{asymp}
equal to zero, leading to
\begin{equation}
{\cal E}_{\bf q}^{\sigma L}=
\frac{g}{2(z_{\bf q}^L)^2}
\frac{\left(1-\frac{n_{\kappa e}}{n_e}\right)I_M^{eL}+\frac{n_\kappa}{n_e}I_1^{eL}
+\frac{1}{\mu_{\bf q}^L g}\frac{q^2}{\pi}
P_q^{\sigma L}}
{\left(1-\frac{n_{\kappa e}}{n_e}\right)I_M^{eL}
+\frac{u_e^2}{u_{e,\kappa}^2}\frac{(\kappa_e+1)}{\kappa_e}
\frac{n_{\kappa e}}{n_e}I_2^{eL}
-\frac{1}{\mu_{\bf q}^L}\frac{q^2}{\pi}
\frac{1}{(\sigma z_q^L)^2}
\gamma_{\bf q}^{\sigma L(coll)}},
\end{equation}
where
\begin{equation}
  I_M^{\beta\alpha}= \int d^3u\; \Phi_{\beta,M}(u)
\delta(\sigma z_{\bf q}^\alpha-{\bf q}\cdot{\bf u}),
\end{equation}
\begin{equation}
  I_1^{\beta\alpha}= \int d^3u\; \Phi_{\beta,\kappa}(u)
  \delta(\sigma z_{\bf q}^\alpha-{\bf q}\cdot{\bf u}),
\end{equation}
\begin{equation}
I_2^{\beta\alpha}= \int d^3u\;
\left(1+\frac{u^2}{\kappa_\beta u_{\beta,\kappa}^2}\right)^{-1}
\Phi_{\beta,\kappa}(u)\delta(\sigma z_{\bf q}^\alpha-{\bf q}\cdot{\bf u}).
\end{equation}

The integrals $I_M^{\beta\alpha}$, $I_1^{\beta\alpha}$ and 
$I_2^{\beta\alpha}$ can be evaluated analytically, resulting in
\begin{eqnarray}
\label{IM,I1,I2}
&&I_M^{\beta\alpha}= \frac{1}{\pi^{1/2} u_{\beta}}
\frac{1}{q}\exp\left(-\frac{(z_q^\alpha/q)^2}{u_{\beta}^2}\right),\\
&&I_1^{\beta\alpha}= \frac{1}{\pi^{1/2} \kappa_\beta^{1/2} u_{\beta,\kappa}}
\frac{\Gamma(\kappa_\beta)}{\Gamma(\kappa_\beta-1/2)}
\frac{1}{q}\left(1+\frac{(z_q^\alpha/q)^2}{\kappa_\beta u_{\beta,\kappa}^2}
\right)^{-\kappa_\beta},\nonumber\\
&&I_2^{\beta\alpha}= \frac{1}{\pi^{1/2} \kappa_\beta^{1/2} u_{\beta,\kappa}}
\frac{\Gamma(\kappa_\beta)}{\Gamma(\kappa_\beta-1/2)}
\frac{\kappa_\beta}{\kappa_\beta+1}
\frac{1}{q}\left(1+\frac{(z_q^\alpha/q)^2}{\kappa_\beta u_{\beta,\kappa}^2}
\right)^{-(\kappa_\beta+1)}\nonumber.
 \end{eqnarray}

Using the expressions given by the equations \eqref{IM,I1,I2}, the
numerator of the expression for the asymptotic wave spectrum becomes
\begin{eqnarray*}
  \left(1-\frac{n_{\kappa e}}{n_e}\right)
  \frac{1}{\pi^{1/2} u_{e}}\frac{1}{q}
  \exp\left(-\frac{(z_q^L/q)^2}{u_{e}^2}\right)
  +\frac{n_{\kappa e}}{n_e}
  \frac{1}{\pi^{1/2} \kappa_e^{1/2} u_{e,\kappa}}
  \frac{\Gamma(\kappa_e)}{\Gamma(\kappa_e-1/2)}
  \frac{1}{q}\left(1+\frac{(z_q^L/q)^2}{\kappa_e u_{e,\kappa}^2}
  \right)^{-(\kappa_e)}
  +\frac{1}{\mu_{\bf q}^L g}\frac{q^2}{\pi}
  P_q^{\sigma L}.
\end{eqnarray*}

Proceeding in the same way, the expression at the denominator
of the expression for the asymptotic wave spectrum becomes
\begin{equation*}
  \begin{split}
   \left(1-\frac{n_{\kappa e}}{n_e}\right)
  &\frac{1}{\pi^{1/2} u_{e}}\frac{1}{q}
  \exp\left(-\frac{(z_q^L/q)^2}{u_{e}^2}\right)
  +\frac{u_e^2}{u_{e,\kappa}^2}
  \frac{n_{\kappa e}}{n_e}
  \frac{1}{\pi^{1/2} \kappa_e^{1/2} u_{e,\kappa}}\\
  &\times\frac{\Gamma(\kappa_e)}{\Gamma(\kappa_e-1/2)}
  \frac{1}{q}\left(1+\frac{(z_q^L/q)^2}{\kappa_e u_{e,\kappa}^2}
  \right)^{-(\kappa_e+1)}
  -\frac{1}{\mu_{\bf q}^L}\frac{q^2}{\pi}
  \frac{T_e}{T_*}\frac{1}{(\sigma z_q^L)^2}
  \gamma_{\bf q}^{\sigma L(coll)}.
 \end{split}
\end{equation*}

Therefore,
\begin{eqnarray}
\label{L,asymp}
{\cal E}_{\bf q}^{\sigma L}&=& 
\frac{g}{2(z_{\bf q}^L)^2}
\biggl[\left(1-\frac{n_{\kappa e}}{n_e}\right)
\exp\left(-\frac{(z_q^L/q)^2}{u_{e}^2}\right)\nonumber\\
&&+\frac{n_{\kappa e}}{n_e}
\frac{1}{(\kappa_e-3/2)^{1/2}}
\frac{\Gamma(\kappa_e)}{\Gamma(\kappa_e-1/2)}
\left(1+\frac{(z_q^L/q)^2}{\kappa_e-3/2}
\right)^{-\kappa_e}+\hat{P}_q^{\sigma L}\biggr]\nonumber\\
&&\times \biggl[\left(1-\frac{n_{\kappa e}}{n_e}\right)
\exp\left(-\frac{(z_q^L/q)^2}{u_{e}^2}\right)\\
&&+\frac{n_{\kappa e}}{n_e}
\frac{1}{(\kappa_e-3/2)^{3/2}}
\frac{\Gamma(\kappa_e+1)}{\Gamma(\kappa_e-1/2)}
\left(1+\frac{(z_q^L/q)^2}{\kappa_e-3/2}
   \right)^{-(\kappa_e+1)}-2\hat{\gamma}_{\bf q}^{\sigma L(coll)}\biggr]^{-1},
   \nonumber
\end{eqnarray}
where
\begin{equation}
  \hat{P}_q^{\sigma L}= \frac{u_e}{\mu_{\bf q}^L g}\frac{q^3}{\pi^{1/2}}
  P_q^{\sigma L},
\end{equation}
\begin{equation}
  \hat{\gamma}_{\bf q}^{\sigma L(coll)}= \frac{u_e}{\mu_{\bf q}^L}
  \frac{q^3}{\pi^{1/2}}\frac{1}{2(\sigma z_q^L)^2}
  \gamma_{\bf q}^{\sigma L(coll)}.
\end{equation}

\autoref{fig5} shows the normalized value of the asymptotic spectrum of
$L$ waves, obtained with the use of Eq. \eqref{L,asymp}, considering a
plasma in which the ion population is described by a Maxwellian velocity
distribution, and the electron population is described by a distribution
as defined in equation \eqref{fbeta}, with $n_{\kappa e}/n_e= 0.05$. That
is, the electron distribution is a core-halo distribution, with $5\%$
of the particles in the halo population. For the evaluation of the
collisional damping and electrostatic bremsstrahlung terms, appearing
in Eq. \eqref{L,asymp}, we used Eqs. \eqref{GqL} and \eqref{PqL}, taking
into account only the Maxwellian distribution, which is a reasonable
approximation  due to the smallness of the halo population. 
\autoref{fig5}(a) shows the spectra obtained considering several values
of the index $\kappa_e$, from $\kappa_e=2.5$ to $\kappa_e=40$. It is seen
that the wave amplitude increases at the region of small values of $q$,
with the increase of $\kappa_e$. It is also seen that in the case of
fairly large value of $\kappa_e$, as $\kappa_e=40$, the spectrum obtained
in the case of a core-halo distribution is very close to the spectrum
obtained in the case of a purely Maxwellian electron distribution, which
can be obtained from equation \eqref{L,asymp} by taking $n_{\kappa e}=0$.
It is also seen that in the case of small values of $\kappa_e$ the
asymptotic spectrum is qualitatively very similar to the spectrum
obtained after the time evolution of the system, taking into account
the new effects of electrostatic bremsstrahlung and collisional damping,
shown in Figure 1 of Ref. \cite{Tigik2017a}. For comparison, we
show in figure \autoref{fig5}(b) the asymptotic spectrum of $L$ waves
which would be obtained by neglecting the new effects, of collisional
bremsstrahlung and collisional damping.
\begin{figure}[h]
  \begin{center}
    %% \hspace{-0.42cm}
    \includegraphics[scale=0.5,angle=270]{IL1D_005kvar.eps}
    \includegraphics[scale=0.5,angle=270]{IL1D_005kvarno-brem.eps}
    \caption{Asymptotic spectrum of Langmuir waves for several values
      of $\kappa$ index and Maxwellian distribution. (a) With electrostatic
    bremsstrahlung emission and collisional damping.
   (b) Without bremsstrahlung emission and collisional damping.}
    \label{fig5}
  \end{center}
\end{figure}

The equations presented in this appendix and, more important, the
ideas about the asymptotic state of the eigenmodes of an unmagnetized
plasma considering several values of kappa coefficient, were developed
in parallel with another work, which slightly deviates from the
main subject of this PhD research, but had a big contribution on the
development of the work presented in Ref. \cite{Tigik2017a}. 
The work in question, whose title
% LFZ190822: Como já comentado em outros lugares, não seria "whose", como 
% estava antes?
% Sabrina: ok
is ``Weakly turbulent plasma processes in the presence of inverse
power-law velocity tail population'' \cite{Tigik2017b}, investigates
the modifications on the spectra of the Langmuir, ion-sound and
transverse waves caused by different kappa indexes in a core-halo
velocity distribution function. The full article can be seen in
\Cref{appC}.


\chapter{Extra publication}
\label{appC}
This publication is mainly related with the results presented in
Ref. \cite{Tigik2017a}.
We can say that most of the ideas about the physical processes occurring
in \cite{Tigik2017a}, regarding the formation of the core-halo velocity
distribution and the fact that the electrostatic bremsstrahlung could be
the underlying process behind the ubiquity of such velocity distribution
in the space environment, came from the analysis of the results presented
in \cite{Tigik2017b}. 

\includepdf[pages=-]{Tigik2017b.pdf}

\chapter{Two-dimensional time evolution of beam-plasma
  instability in the presence of binary collisions}
\label{master}
The paper included in this appendix is the complete text of Ref.
\cite{Tigik2016a}, which summarizes the outcomes of the Master's
dissertation \cite{Tigik2015} that led to the current doctoral
research. This text was included here with the intention to give
the reader a complete view of the work developed during my graduate
studies.



\includepdf[pages=-]{Tigik2016a.pdf}



%% Eu acho que ainda tenho que escrever algo aqui. Vou pensar ainda a respeito.

% \chapter{Klimontovich formalism}
% \label{app-klimo}
% % The Klimontovich formalism is the basis of 
% % This appendix has the objective to complement important concepts discussed in
% % \Cref{noneigenmode}. 
% % In this appendix I present the standard approach that leads to the well known
% % Balescu-Lenard collision integral. The objective here is to put the formalism
% % presented in this thesis in a context where it can be compared with the usual
% % text-book method that employs the Klimontovich statistical formulation. The
% % derivation shown here is basically a version of Chapter 4

% \chapter{Microscopic equations for a fully ionized plasma}
% \label{mic-eq}
% Let us consider a fully ionized hydrogen plasma, where $e_a$ and $m_a$ are,
% respectively the charge and the mass of the particles of species $a$, that
% can be or electrons $e_e\equiv -e$ or protons (ions) $e_i\equiv e$. Denoting
% $N_a$ as the total number of particles of species $a$ in the plasma, the
% neutrality condition of a plasma is given by
% \begin{equation}
%   \sum_ae_aN_a=0.  
% \end{equation}

% The probability distribution function for a N-body system in phase-space,
% $({\bf r, v})$, is given by the Klimontovich function
% \begin{equation}
% N_a({\bf r, v}, t)=\sum_{i=1}^{N_a}
% \delta[{\bf r}-{\bf r}_i^a(t)]
% \delta[{\bf v}-{\bf v}_i^a(t)],
% \end{equation}
% where ${\bf r}_i^a$ is the exact position and ${\bf v}_i^a$ the exact velocity
% of the $i\unit{th}$ particle of species $a$, in a given time $t$.






\end{appendix}


\bibliographystyle{unsrturl}
\bibliography{papers,books}


\end{document}



%%% Local Variables:
%%% mode: latex
%%% TeX-master: t
%%% End:
